\documentclass[12pt]{amsart}
\usepackage{amscd, amsmath, verbatim, enumitem, graphicx, amssymb, mathtools, xfrac, amsthm, amsfonts, amstext, hyperref}
\usepackage[utf8]{inputenc}
\usepackage[backend=biber, citestyle=numeric-comp, natbib=true, sortlocale=en_US, url=false, doi=false, eprint=true, maxbibnames=4]{biblatex}            
%\usepackage[all]{xy}
\usepackage[activate={true,nocompatibility}]{microtype}
\usepackage{tikz-cd}

\renewbibmacro{in:}{\ifentrytype{article}{}{\printtext{\bibstring{in}\intitlepunct}}}
\newbibmacro{string+doi}[1]{\iffieldundef{doi}{\iffieldundef{url}{#1}{\href{\thefield{url}}{#1}}}{\href{http://dx.doi.org/\thefield{doi}}{#1}}}
\DeclareFieldFormat{title}{\usebibmacro{string+doi}{\mkbibemph{#1}}}
\DeclareFieldFormat[article]{title}{\usebibmacro{string+doi}{\mkbibquote{#1}}}
\DeclareFieldFormat[inproceedings]{title}{\usebibmacro{string+doi}{\mkbibquote{#1}}}
\DeclareFieldFormat[thesis]{title}{\usebibmacro{string+doi}{\mkbibquote{#1}}}
\addbibresource{paper.bib}

\textwidth 16cm 
\textheight 22cm 
\headheight 0.5cm 
\evensidemargin 0.3cm 
\oddsidemargin 0.2cm

\numberwithin{equation}{section}
\newcounter{thmcounter} \newcounter{lemmacounter}
\newtheorem{thm}[thmcounter]{Theorem}
\newtheorem{prop}[thmcounter]{Proposition}
\newtheorem{cor}[thmcounter]{Corollary}
\newtheorem{lemma}[lemmacounter]{Lemma}
\theoremstyle{definition}
\newtheorem{rem}[equation]{Remark}
\newtheorem{example}[equation]{Example}
\newtheorem{dfn}[equation]{Definition}
\newtheorem{notation}[equation]{Notation}

\newtheorem*{thm*}{Theorem}
\newtheorem*{lemma*}{Lemma}

\DeclareMathOperator{\K}{K}
\DeclareMathOperator{\SK}{SK}
\DeclareMathOperator{\G}{G}
\DeclareMathOperator{\GL}{GL}
\DeclareMathOperator{\SL}{SL}
\DeclareMathOperator{\Sp}{Sp}
\DeclareMathOperator{\St}{St}
\DeclareMathOperator{\E}{E}
\DeclareMathOperator{\EP}{EP}
\DeclareMathOperator{\Par}{P}
\DeclareMathOperator{\B}{B}
\DeclareMathOperator{\Hh}{H}
\DeclareMathOperator{\U}{U}
\DeclareMathOperator{\X}{X}
\DeclareMathOperator{\Z}{Z}
\DeclareMathOperator{\M}{M}
\DeclareMathOperator{\SR}{SR}
\DeclareMathOperator{\sr}{sr}
\DeclareMathOperator{\shape}{shape}
\DeclareMathOperator{\Rad}{Rad}
\DeclareMathOperator{\Max}{Max}
\DeclareMathOperator{\Spec}{Spec}
\DeclareMathOperator{\Spin}{Spin}
\DeclareMathOperator{\Stab}{Stab}
\DeclareMathOperator{\ASR}{ASR}
\DeclareMathOperator{\asr}{asr}
\DeclareMathOperator{\Ums}{Ums}
\DeclareMathOperator{\Umd}{Umd}
\DeclareMathOperator{\rk}{rk}
\newcommand{\rA}{\mathsf{A}}
\newcommand{\rB}{\mathsf{B}}
\newcommand{\rC}{\mathsf{C}} 
\newcommand{\rD}{\mathsf{D}} 
\newcommand{\rE}{\mathsf{E}}

\makeatletter
\newcommand{\indexbox}[1]{\text{\fboxsep=.1em\fbox{\m@th$\displaystyle#1$}}}
\makeatother

\def\ssub#1{\mathchoice
   {_{\lower2pt\hbox{$\scriptstyle #1$}}}
   {_{\lower2pt\hbox{$\scriptstyle #1$}}}
   {_{\lower1.5pt\hbox{$\scriptscriptstyle #1$}}}
   {_{\!\lower1.5pt\hbox{$\scriptscriptstyle #1$}}}}

\title [Parabolic factorizations of congruence subgroups] {Parabolic factorizations and width of congruence subgroups in Chevalley groups}
\keywords {Chevalley groups, relative subgroups, stability for $\K_1$. {\em Mathematical Subject Classification (2010):} 20G40, 20G35, 19B14}
\author {Sergey Sinchuk, Andrei Smolensky}
\email {sinchukss {\it at} yandex.ru,\ andrei.smolensky {\it at} gmail.com}
\date {\today}

\begin{document}

%\begin{abstract} \end{abstract}

\maketitle

\section{Introduction}\label{sec:intro}
Let $R$ be a commutative unital ring and $\Phi$ be a reduced irreducible root system with a fixed basis of simple roots $\Pi$.
Denote by $G=\G(\Phi, R)$ the split simply-connected Chevalley group of type $\Phi$ over $R$ and by $\E(\Phi, R)$ its \emph{elementary subgroup}, 
i.\,e. the subgroup generated by root unipotents $x_\alpha(\xi)$, $\alpha\in\Phi$, $\xi\in R$ see~\cite{VP, St78, S}.
For an ideal $I \trianglelefteq R$ denote by $\G(\Phi, R, I)$ the \emph{relative Chevalley group} (also called \emph{principal congruence subgroup}) of level $I$, 
i.\,e. the kernel of the map $\G(\Phi, R)\rightarrow\G(\Phi, R/I)$ induced by the canonical projection $R\rightarrow R/I$.

The main goal of the present paper is to study parabolic factorizations of Chevalley groups and their analogues for principal congruence subgroups $\G(\Phi, R, I)$.
By a parabolic factorization we mean a decomposition of $G$ into a product of several parabolic and unipotent subgroups.
To be more specific, we study relative versions of the following three kinds of decompositions:
\begin{itemize}
\item \emph{Gauss decomposition,} i.\,e. a decomposition of the form $\E(\Phi,R)=\B \cdot \U^- \cdot \U$, where $\B$ denotes the standard Borel subgroup,
      and $\U$, $\U^-$ are the unipotent radicals of $\B$ and the opposite Borel subgroup $\B^-$;
\item \emph{Bass---Kolster decomposition} $G = \Par_i \cdot \U_i^- \cdot \U_i \cdot \U_i^-$, where 
      $\Par_i$, $\Par_i^-$ are the opposite maximal parabolic subgroups of $\G$ corresponding to $i$-th simple root of $\Phi$ and $\U_i$, $\U_i^-$ are their unipotent radicals;
\item \emph{Dennis---Vaserstein decomposition} $\E(\Phi, R) = \EP_i \cdot (\U_i^- \cap \U_j^-) \cdot \EP_j$, $i\neq j$,
      where $\EP_i$ denotes the subgroup generated by root unipotents $x_\alpha(\xi)$ contained in $\Par_i$.
\end{itemize}
%TODO: Insert some links to VIP's papers :)

%TODO: Illustrate each statement of the text below with a reference:
There are two main reasons to study parabolic factorizations.
Recall that a number of various factorizations of $\G$ are known over small-dimensional rings.
For example, Bruhat decomposition is the key instrument in the study of linear groups over fields, whereas Gauss decomposition and unitriangular factorizations are useful tools in the study of groups over local rings.
Moreover, for each particular analytic or arithmetic setting there is a special decomposition theorem (Iwasawa, Cartan and Iwahori decompositions, to mention a few).
Unfortunately, neither of these decompositions can be carried over to rings of higher dimension.
The main advantage of parabolic factorizations lies in the fact that they remain true for a broader class of rings described in terms of milder ring-theoretic assumptions formulated in terms of the \emph{stable rank} of the ground ring.
Their another advantage is that they prove useful in the study of asymptotical properties of Chevalley groups. We are going to discuss this theme in more detail below.

Let $G$ be an abstract group with a generating set $X \subset G$. 
We denote by $W(G, X)$ the \emph{width of $G$ with respect to $X$}, i.\,e. the smallest natural number $N$ such that every element of $G$ is a product of at most $N$ elements of $X$.
The problem of finding an estimate for the width of $\G(\Phi, R)$ in terms of elementary generators $x_\alpha(\xi)$ has been extensively studied in a number of works (see e.\,g. \cite{CK83, Ka82, Tavgen91, Mo07, VseUnitrZ1p, VavSmSuUnitrEng}).
Summarizing results of these works, one can hope that such width is finite only for rather narrow classes of rings of number-theoretic nature, e.\,g. localizations of Dedeking rings of arithmetic type.
For example, W.~van der Kallen has shown that the width of $\SL(3,R)$ is infinite already for a Euclidean ring $R=\mathbb{C}[t]$.

On the other hand, much less is known about the width of the relative group $\G(\Phi, R, I)$. 
Set $z_\alpha(s, \xi) = x_{\alpha}(s)^{x_{-\alpha}(\xi)}$, $s\in I$, $\xi\in R$.
It is a classical result of J.~Tits and L.~Vaserstein (see e.\,g.~\cite[Theorem~2]{Va86}) that the relative elementary group $\E(\Phi, R, I)$ is generated by
the set of elements $\mathcal{Z} = \{ z_\alpha(s,\xi) \mid \alpha \in \Phi,\ s\in I,\ \xi \in R \}.$

In fact, one can choose a smaller set of generators for $\E(\Phi, R, I)$. 
Let $\Sigma$ be any special subset of $\Phi$, (e.\,g. the set of simple roots $\Pi$ or the special part of $\Sigma_S$ of some parabolic subset of roots $S$, see section~\ref{sec:rootsys}).
Set $$\mathcal{X} = \{ x_\alpha(s) \mid \alpha\in \Phi,\ s\in I\};\ \ \mathcal{Z}(\Sigma) = \{ z_\alpha(s, \xi) \mid s \in I,\ \xi \in R, \alpha\in \Sigma \} \cup \mathcal{X}.$$ 

Now we are ready to state a recent theorem of A.~Stepanov (see~\cite[Theorem~3.4]{S}).  
\begin{thm*}\label{thm:Stepanov}
Let $\Phi$ be a reduced irreducible root system of rank $\geq 2$ and let $S \subset \Phi$ be any parabolic subset of roots.
Then $\E(\Phi, R, I)$ is generated as an abstract group by $\mathcal{Z}(\Sigma_S)$.
\end{thm*}

Our first application of parabolic factorizations is the following theorem which gives an estimate of the width of $\G(\Phi, R, I)$ in terms of certain subsets of $\mathcal{Z}$.
\begin{thm}\label{thm:width} Let $\Phi$ be an irreducible root system, and let $I$ be an ideal of a commutative ring $R$.
\begin{enumerate}
\item If $\sr(I) = 1$, then $W(\E(\Phi,R,I),\mathcal{Z}(-\Pi))\leqslant 3|\Phi^+|+2\rk(\Phi)-1$;
\item Let $p$ be a prime integer, set $R = \mathbb{Z}[\sfrac{1}{p}]$ and assume $\Phi$ is classical. Then under the assumption of the Generalized Riemann Hypothesis one has
%TODO: Now check the coefficients of \rk(\Phi) in the expression below
\begin{alignat*}{2}
& W(\E(\Phi,R,I), \mathcal{Z}(\Sigma_\ell^-))\leqslant 3|\Phi^+| + 2\rk(\Phi) + 1 & \text{ if } \Phi=\rA_\ell,\rC_\ell, \\
& W(\E(\Phi,R,I), \mathcal{Z}(\Sigma_\ell^-))\leqslant 4|\Phi^+| + \rk(\Phi) + 1 & \text{ if } \Phi=\rB_\ell,\rD_\ell;
\end{alignat*}
\item Let $\mathcal{O}_S$ be a Dedekind ring of arithmetic type embedded into some global field $k$ having a real embedding. 
Assume $\Phi$ is classical of rank $\geqslant2$, then $W(\G(\Phi,\mathcal{O}_S,I), \mathcal{Z}(\Sigma_\ell^-))$ is finite.
\end{enumerate}
\end{thm}
The particular case $\Phi = \rA_\ell$, $R = I = \mathbb{Z}$ of the above theorem is similar in spirit to a recent theorem of U.~Hadad (see ~\cite[Theorem~1.6]{Ha12}).

Another application of parabolic factorizations developed in this paper are subsystem factorizations in terms of subgroups of type $\rA_1$.
The case of a finite field $R=F$ was studied by M.~Liebeck, N.~Nikolov and A.~Shalev in~\cite{LiebNikShaSL2}.
More precisely, for an untwisted Chevalley group it has been proved that $\G(\Phi, F)$ is a product of at most $N=5|\Phi^+|$ copies of $\SL(2, F)$.
In~\cite{VavKovSL2} N.~Vavilov and E.~Kovach noted that Bruhat decomposition immediately implies a bound of $N=3|\Phi^+|$ factors for arbitrary field $F$.
In fact, Gauss decomposition (see~\cite{Sm12}) implies a similar factorization for the elementary subgroup $\E(\Phi, R)$ 
with $3|\Phi^+|$ factors for arbitrary commutative ring of stable rank $1$.

In the linear and symplectic case the above results can be extended to even larger class of rings (which e.\,g. includes all Dedekind rings).
\begin{thm}\label{thm:SL2width}
Assume that one of the following is true:
\begin{itemize} 
\item $I$ is an ideal such that $\sr(I)\leq 2$, $\Phi=\rA_\ell$, $l\geq 2$, $N=3|\Phi^+|+\rk(\Phi) - 3$.
\item $I$ is an ideal such that $\sr(I)\leq 3$, $\Phi=\rC_\ell$, $l\geq 2$, $N=3|\Phi^+|+2\rk(\Phi) - 4$
\end{itemize}
Then $\G(\Phi, R, I)$ can be presented as a product of at most $N$ copies of its subgroups isomorphic to $\G(\rA_1, R, I)$.
\end{thm}

Yet another application of parabolic factorizations concerns subsystem factorizations of Chevalley groups formulated in terms of submaximal subgroups of type $\rA_\ell$.
In case when $R=I$ is a field this question was studied by N.~Nikolov in \cite{NikProdDecomp}. 
\begin{thm*} Let $G$ be a classical Chevalley group (of normal or twisted type) of rank $n$ over a finite field.
Then $G$ equals the product of at most $200$ conjugates of $\SL_n$. \end{thm*}

It is also classically known that over an arbitrary field $F$ the group $\SL(n+1, F)$, $n\geq 3$ is a product of at most $4$ subgroups of type $\SL(n, F)$.
On the other hand, from the classical Dennis---Vaserstein theorem (see~\cite{Va69} and \cite[Lemma~2.1]{ST76})
it follows that $\SL(n+1, R, I)$ is a product of a most $5$ subgroups of type $\SL_n(R, I)$ for any ideal $I$ satisfying $\sr(I)\leqslant n-1$.
Using another Dennis---Vaserstein type factorization we obtain the following result.
\begin{thm} Assume that $\sr(I) = 2$ then the group $\Spin_{2\ell}(R, I)=\G_{sc}(\rD_\ell, R, I)$ is a product of at most $C$ copies of subgroups of type $\rA_{\ell-1}$. \end{thm}


\subsection{Acknowledgements}
Authors of the present paper acknowledge financial support from Russian Science Foundation grant 14-11-00297.

\section{Preliminaries}\label{sec:prelim}
For any collection of subsets $H_1,\ldots, H_n$ of a group $G$ we denote by $H_1\ldots H_n$ their Minkowski set-product,
i.\,e. the set consisting of arbitrary products $h_1\ldots h_n$ of elements $h_i\in H_i$. In particular, the equality
$G = H_1\cdot\ldots\cdot H_n$ means that every element $g\in G$ can be presented as a product $h_1\ldots h_n$ for $h_i\in H_i$.

\subsection{Root systems}\label{sec:rootsys}
Let $\Pi=\{\alpha_1, \ldots, \alpha_l\}$ be a basis of $\Phi$. Denote by $\Phi^+$ and $\Phi^-$ subsets of positive and negative roots with respect to $\Pi$.
Denote by $m_i(\alpha)$ the coefficient of $\beta$ in the expansion of $\alpha$ in $\Pi$, i.\,e. $\alpha = \sum_{i=1}^n m_i(\alpha) \alpha_i$.

A root subset $S\subseteq \Phi$ is called {\it parabolic} (resp. {\it reductive}, resp. {\it special}) if $\Phi=S \cup -S$ (resp. $S = -S$, resp. $S \cap -S=\varnothing$).
Any parabolic subset $S \subseteq \Phi$ can be decomposed into the disjoint union of its reductive and special part, i.e. 
$S = \Sigma_S \sqcup \Delta_S$, where $\Sigma_S \cap (-\Sigma_S) = \varnothing$, $\Delta_S = -\Delta_S$.

We denote by $(\alpha, \beta)$ the scalar product of roots and by $\langle \alpha, \beta\rangle$ the integer $2(\alpha, \beta)/(\alpha, \alpha)$.
Denote by $W(\Phi)$ the subgroup of isometries of $\Phi$ generated by all reflections $\sigma_\alpha$ where $\sigma_\alpha(\beta)=\beta-\langle\alpha,\beta \rangle\cdot \alpha$.
For $S\subseteq \Phi$ denote by $W(S)$ the subgroup of $W(\Phi)$ generated by $\sigma_\alpha$ for $\alpha\in S$.

Let $J\subseteq \Pi$ be a subset of simple roots. 
For a root $\beta = \sum d_i\alpha_i \in \Phi$ set $\shape(J, \beta):=\sum\limits_{i\in J}d_i \alpha_i$.
Set $$\Delta_J = \{\alpha \in \Phi \mid \shape(J, \alpha)=0\},\quad \Sigma^\pm_J = \{\alpha \in \Phi \mid \shape(J, \alpha)\in\Phi^\pm \}.$$
Clearly, $\Delta_J$ is a reductive subset of roots while $\Sigma^\pm_J$ are special subsets.
For two disjoint subsets $I, J\subseteq \Pi$ one has $\Sigma^\pm_{I \cup J} = \Sigma^\pm_I\cup\Sigma^\pm_J$ while $\Delta_{I\cup J} = \Delta_I \cap \Delta_J$.

\begin{lemma}[{\cite[Lemma~1]{ABS}}]\label{lemma:abs}
Let $\alpha, \beta \in \Sigma^\pm_J$ be two roots having the same length such that $\shape(J,\alpha)=\shape(J,\beta)\neq 0$.
Then $\alpha$ and $\beta$ are conjugate under the action of $W(\Delta_J)$.
\end{lemma}

\subsection{Chevalley groups}
Our exposition of Chevalley groups is standard and follows~\cite{Ta, S, St78, VP, Va86}.
We denote by $\G(\Phi, R)$ the simply-connected split Chevalley group of type $\Phi$ over arbitrary commutative ring $R$. For example, for each of the four classical series $\Phi=\rA_\ell, \rB_\ell, \rC_\ell, \rD_\ell$
the group $\G(\Phi, R)$ equals $\SL(\ell+1, R)$, $\Spin(2\ell+1, R)$, $\Sp(2\ell, R)$, $\Spin(2\ell, R)$, respectively. Notice that from the uniqueness theorem of M.~Demazure (see~\cite[Corollaire~5.2]{SGA3}) 
it follows that the exceptional isomorphisms of small-rank groups (which are well-known in the theory of groups over fields) remain valid for Chevalley groups over arbitrary commutative rings.
In particular, there are isomorphisms of groups $\Spin(3,R)\cong\SL(2,R)$, $\Spin(5, R)\cong \Sp(4, R)$, $\Spin(6, R)\cong\SL(4, R)$, $\SL(2, R)\cong \Sp(2, R)$.  

In this paper we work with the irreducible representations of Chevalley groups with the highest weight
$\varpi_1$ (in the case of a classical $\Phi$) or $\varpi_\ell$ (in the case $\Phi=\rE_\ell$, $\ell=6,7$). The former are the natural vector representations of classical groups acting on free modules $V=R^n$ of dimension $n=\ell+1, 2\ell+1, 2\ell,2\ell$ for $\Phi=\rA_\ell,\rB_\ell,\rC_\ell,\rD_\ell$ respectively.
The latter are the microweight representations of dimension $n=27, 56$ for $\ell=6,7$.

%TODO: Rewrite this using better terminology.
For a representation $\pi$ of $\G(\Phi, R)$ on $V$ we denote by $\Lambda=\Lambda(\pi)$ the set of weights of $\pi$ and by $V_\lambda$ the weight subspace corresponding to $\lambda\in\Lambda$.
For any of the representations described above (with the sole exception of $(\rE_8, \varpi_8)$) all weight subspaces $V_\lambda$ are one-dimensional. 
This allows us to choose a basis $E_\pi=\{e_\lambda\}_{\lambda\in\Lambda}$.

It will be convenient to use the standard numbering for the weights of the natural representations of classical groups (cp.~\cite[§1B]{St78}):
\[\begin{array}{cll}
  1,2,\ldots, \ell+1 & \text{in the case} & \Phi =\rA_\ell, \\
  1,2,\ldots \ell, 0, -\ell,\ldots, -2, -1 & \text{in the case} & \Phi =\rB_\ell, \\
  1,2,\ldots \ell, -\ell,\ldots, -2, -1 & \text{in the cases}   & \Phi =\rC_\ell, \rD_\ell. \\
\end{array}\]
For example, we write $1$ instead of $\varpi_1$, $2$ instead of $\varpi_1-\alpha_1$ etc.

For a weight $\lambda\in \Lambda$ we denote by $(-)_\lambda\colon V\to R$ the coordinate function corresponding to $\lambda$.
In other words,  $v_\lambda$ is the coefficient of the basis vector $e_\lambda$ in the expansion of $v$ in $E_\pi$, i.e. $v=\sum_{\lambda\in \Lambda} v_\lambda e_\lambda$.
We denote by $v^+$ the basis vector corresponding to the highest weight.
For classical $\Phi$ we have $v^+=e_1$.

\subsection{Elementary subgroup}\label{sec:elementary}
Recall from~\cite{St78, VP} that for $\alpha\in \Phi$, $\xi\in R$ one can define certain elements $x_{\alpha}(\xi)$ of $\G_{sc}(\Phi, R)$ called {\it elementary root unipotents}.
These elements satisfy the well-known Steinberg relations:
\begin{align}
& \phantom{\left[{}\vphantom{x_\beta(t)}\right.}
x_\alpha(s) x_\alpha(t) = x_\alpha(s+t), \label{rel:add}\\
& \left[x_\alpha(s),  x_\beta(t)\right] = \prod x_{i\alpha + j\beta}\left(N_{\alpha\beta ij}\, s^i t^j\right),\quad \alpha\neq-\beta, \quad N_{\alpha\beta ij}\in\mathbb{Z}. \label{rel:CCF}
\end{align} 
The indices $i$, $j$ in the above formula range over all positive natural numbers such that $i\alpha + j\beta\in\Phi$.
The integers $N_{\alpha\beta ij}$ are called {\it structure constants} of the Chevalley group $\G(\Phi,R)$ and depend only on $\Phi$.

For $\varepsilon\in R^*$ set $w_\alpha(\varepsilon) = x_\alpha(\varepsilon) x_{-\alpha}(-\varepsilon^{-1}) x_{\alpha}(\varepsilon).$
If $\rk(\Phi)\geqslant 2$ the following relation is a consequence of \ref{rel:add}--\ref{rel:CCF}:
\begin{equation}\label{rel:R3}
w_\alpha(\varepsilon) x_{\beta}(\xi) w_\alpha(\varepsilon)^{-1} =
x_{\sigma\ssub{\alpha}\beta} \left(\eta_{\alpha, \beta}\cdot\varepsilon^{-\langle\beta,\alpha \rangle}\xi\right),\quad \varepsilon\in R^*,\ \xi\in R.
\end{equation}
where coefficients $\eta_{\alpha, \beta} = \pm 1$ (see~\cite[\S13]{VP} for more details on $\eta_{\alpha, \beta}$).

Define semisimple root elements $h_\alpha(\varepsilon)$, $\alpha\in\Phi$, $\varepsilon\in R^*$ as $h_\alpha(\varepsilon)=w_\alpha(\varepsilon)w_\alpha(-1)$. Then
\begin{equation}\label{rel:R6}
h_\alpha(\varepsilon)x_\beta(\xi)h_\alpha(\varepsilon)^{-1} = x_\beta\left(\varepsilon^{\langle\beta,\alpha\rangle}\xi\right),\quad \varepsilon\in R^*,\ \xi\in R.
\end{equation}

\begin{rem} Let $\lambda_1, \lambda_2 \in \Lambda(\pi)$ be a pair of weights of a representation $\pi$ of a classical group such that $\lambda_1-\lambda_2\in \Phi$.
In this situation it will be convenient for us to write $x_{\lambda_1,\lambda_2}(\xi)$ instead of $x_{\lambda_1-\lambda_2}(\xi)$.
For example, for $\Phi=\rA_\ell$ we have $x_{1,2}(\xi)=x_{1-2}(\xi)=x_{\varpi_1 - \varpi_1 + \alpha_1}(\xi) = x_{\alpha_1}(\xi)$. \end{rem}

Let $I\leq R$ be an ideal. Denote by $\E(\Phi, I)$ the subgroup of $\G_{sc}(\Phi, R)$ generated by the elementary root unipotents $x_\alpha(\xi)$, $\alpha\in\Phi$, $\xi\in I$.
In the special case $I=R$ the subgroup $\E(\Phi, R)$ is called the {\it elementary subgroup} of the Chevalley group $\G_{sc}(\Phi, R)$.
Taddei's theorem asserts that $\E(\Phi, R)$ is a normal subgroup of $\G_{sc}(\Phi, R)$ provided $\Phi$ is an irreducible root system of rank $\geqslant 2$.

We denote by $\E(\Phi, R, I)$ the normal closure of $\E(\Phi, I)$ in $\E(\Phi, R)$, in other words $\E(\Phi, R, I) = \E(\Phi, I)^{\E(\Phi, R)}$.
Denote by $z_\alpha(s, \xi)$ the element $x_\alpha(s)^{x\ssub{-\alpha}(\xi)}$.

Since elementary subgroup $\E(\Phi,R,I)$ is normal in $\G(\Phi,R,I)$, one can define the $\K_1$-functor by $\K_1(\Phi,R,I)=\G(\Phi,R,I)/\E(\Phi,R,I)$. When $I=R$, we write $\K_1(\Phi,R)$ for $\K_1(\Phi,R,R)$. In some cases it is known that $\K_1(\Phi,R,I)$ is trivial. For example, $\SK_1(\ell+1,R)=\K(\rA_\ell,R)=1$ for any ring of stable rank $1$ (see Section~\ref{sec:stability-conditions}), and for other root systems some stronger assumption is required such as $\asr(R)=1$ or being semilocal. For a Euclidean ring $\K_1$ is trivial for every root system.

Let $k$ be a global field, $S$ be a set of places of $k$, $\mathcal{O}_S$ be the Dedekind ring of arithmetic type defined by $S$ and $I$ be an ideal of $\mathcal{O}_S$.
\begin{thm*}[{\cite[Theorem~3.6]{BassMilnorSerre} and \cite[Corollary~4.5]{Ma69}}]
Assume that the field $k$ has a real embedding. Then $\K_1(\Phi,\mathcal{O}_S,I)=1$.
\end{thm*}

%The following result of L.~Vaserstein and J.~Tits describes a set of generators of $\E(\Phi, R, I)$ (see e.\,g.~\cite[Theorem~2]{Va86}).
%\begin{thm}\label{theorem:Tits-Vaserstein} Let $\Phi$ be an irreducible root system of rank $\geq 2$, then the relative elementary subgroup $\E(\Phi, R, I)$
%is generated by the elements $z_\alpha(s, \xi)$, $\alpha\in \Phi$, $s\in I$, $\xi\in R$. \end{thm}

%In fact, one can choose a more economical set of generators for $\E(\Phi, R, I)$ (see~\cite[Theorem~3.4]{S}).
%\begin{thm}\label{theorem:Stepanov}
%Let $\Phi$ be a reduced irreducible root system of rank $\geq 2$ and let $S$ be any parabolic subset of roots of $\Phi$ with the special part $\Sigma_S$.
%Then $\E(\Phi, R, I)$ is generated by the union of the following two families of elements:
%\begin{itemize}
%\item $x_{\alpha}(s)$, where $s\in I$, $\alpha\in\Phi$;
%\item $z_\alpha(s,\xi)$, where $s\in I$, $\xi\in R$, $\alpha\in\Sigma_S$.
%\end{itemize}
%\end{thm}
 
For a special subset of roots $S\subseteq \Phi$ we denote by $\U(S, I)$ the subgroup spanned by all $x_{\alpha}(s)$ for $s\in I$, $\alpha\in S$.
The subgroup $\U(\Sigma_J, I)$ is normalized by $\E(\Delta_J, R)$, hence the Minkowski product set $\EP_J(R, I) = \E(\Delta_J, R, I) \cdot \U(\Sigma_J, I)$ is a subgroup. 
The following two equalities will be referred to in the sequel as {\it Levi decomposition}: 
\[ \EP_J(R, I) = \U(\Sigma_J, I) \cdot \E(\Delta_J, R, I) = \E(\Delta_J, R, I) \cdot \U(\Sigma_J, I). \]

The following lemma is a relative version of the classical result sometimes called Chevalley---Matsumoto decomposition.
\begin{lemma}\label{lemma:Chevalley-Matsumoto}
Let $\pi$ be the fundamental representation of $\G_{sc}(\Phi, R)$ with the highest weight $\varpi_s$.
Assume that $g\in \G_{sc}(\Phi, R, I)$ is such that $(g\cdot v^+)_{\varpi\ssub{s}}=1$, then 
\[ g \in \U(\Sigma_s^-, I) \cdot \G_{sc}(\Delta_s, R, I) \cdot \U(\Sigma_s^+, I). \]
\end{lemma}
\begin{proof}
In the absolute case ($I=R$) the statement of the lemma is contained in~\cite[Theorem~1.3]{St78}. Thus we can write $g=u_1\cdot g'\cdot u_2$, where $u_1\in\U(\Sigma_s^-,R)$, $u_2\in\U(\Sigma_s^+,R)$ and $g'\in\G(\Delta_s,R)$. Since $g$ lies in $\G(\Phi,R,I)$, the vector $gv^+$ is $I$-unimodular. On the other hand, $g'u_2v^+=v^+$, so $u_1\in\G(\Phi,R,I)\cap\U(\Sigma_s^-,R)=\U(\Sigma_s^-,I)$. The matrix $g'u_2$ is an element of $\left(\G(\Delta_s,R)\U(\Sigma_s^+,R)\right)\cap\G(\Phi,R,I)$, and by Levi decomposition $u_2\in\U(\Sigma_s^+,I)$, $g'\in\G(\Delta_s,R,I)$.
\end{proof}

Denote by $\Hh(\Phi,R)$ the subgroup generated by all $h_\alpha(\varepsilon)$, $\alpha\in\Phi$, $\varepsilon\in R^*$. Then set
\[ \Hh(\Phi,R,I) = \Hh(\Phi,R)\cap\G(\Phi,R,I)=\langle h_\alpha(\varepsilon),\ \alpha\in\Phi,\ \varepsilon\in R^*\cap(1+I)\rangle. \]
\begin{lemma}\label{lemma:rel-tor-elementary}
$\Hh(\Phi,R,I)\leqslant\E(\Phi,R,I)$.
\end{lemma}
\begin{proof}
Set $\varepsilon=1+s$, $s\in I$ and write
\begin{multline*}
h_\alpha(1+s) = x_\alpha\left(-1\middle)\, x_{-\alpha}\middle(-s\middle)\, x_\alpha\middle((1+s)^{-1}\middle)\, x_{-\alpha}\middle(s(1+s)\right) = \\
= x_\alpha\left((1+s)^{-1}-1\middle)\, z_{-\alpha}\middle(-s,(1+s)^{-1}\middle)\, x_{-\alpha}\middle(s(1+s)\right).
\end{multline*}
It remains to note that $(1+s)^{-1}\in 1+I$ and so all the factors lie in $\E(\Phi,R,I)$.
\end{proof}

\begin{lemma}[{\cite[Corollary~3.3]{S}}]\label{lemma:Stepanov-ideal}
Let $\Phi$ be a root system of rank $\geqslant2$, let $R$ be a commutative ring and $I\trianglelefteq R$ be its ideal.
If $\Phi\neq\rC_\ell$ then $\E\left(\Phi,R,I^2\right)\leqslant\E(\Phi,I)$, otherwise $\E\left(\Phi,R,II^{\indexbox{2}}\right)\leqslant\E(\Phi,I)$.
\end{lemma}
Here $I^{\indexbox{2}}$ denotes the ideal generated by squares $a^2$, where $a\in I$.
Recall also that $I^2$ is generated by products $ab$ for all $a,b\in I$.
Clearly, $II^{\indexbox{2}}$ is generated by elements of the form $a^2b$ for $a,b\in I$.


\section{Stability conditions}\label{sec:stability-conditions}
In this section we define stability conditions used in the statements of our decomposition theorems from section~\ref{sec:factorizations}.
First, we recall the notion of the stable rank of an ideal introduced by L.~Vaserstein in~\cite{Va69, Va71}.
As we will be mainly concerned with applications to Chevalley groups, our exposition of stable ranks is neccessarily limited to commutative rings.
The main goal of the subsection~\ref{sec:rel-asr} is to define the relative version of the absolute stable rank condition which generalizes the definition of~\cite{EO, MKV}.
Finally, in the end of this section we state several technical lemmas describing the action of certain unipotent radicals on unimodular columns under stability conditions.

\subsection{Relative stable rank}
Recall that a row $a\in{}^n\!R$ is called \emph{$I$-unimodular} if it is congruent to $e_1$ modulo $I$ and its components $a_1,a_2,\ldots,a_n$ generate $R$ as an ideal.
A column $b \in R^n$ is called $I$-unimodular if its transpose $b^t$ is an $I$-unimodular row.
We denote the set of all $I$-unimodular rows (resp. columns) by $\Umd(n, I)$ (resp. $\Ums(n, I)$).
When $R=I$ we refer to $R$-unimodular rows and columns as simply unimodular.

It is not hard to show that for an $I$-unimodular row $a$ there exists an $I$-unimodular column $b$ such that $ab=1$ (see~\cite[\S~2]{Va69}).

An $I$-unimodular row $a=(a_1,\ldots a_{n+1})$ is called \emph{stable} if one can choose $b_1,\ldots,b_n\in I$ such that the row $(a_1+a_{n+1}b_1,\ldots,a_n+a_{n+1}b_n)$ is also $I$-unimodular. 
We say that $I$ satisfies stable range condition $\SR_n(I)$ if any $I$-unimodular row of length $n+1$ is stable.
It is easy to see that $\SR_n(I)$ implies $\SR_m(I)$ for any $m\geqslant n$. 
It is also clear that condition $\SR_n(I)$ does not depend on the choice of the ambient ring $R$.
By definition, the {\it stable rank} of $I$ (denoted $\sr(I)$) is the smallest natural $n$ such that $\SR_n(I)$ holds (we set $\sr(I)=\infty$ if no such $n$ exists).

The following proposition summarizes basic properties of stable ranks.
\begin{prop} \label{prop:sr_properties}
Let $R$ be arbitrary commutative unital ring and let $I\trianglelefteq R$ be an ideal.
\begin{enumerate}
\item For any ideal $J\subseteq I$ one has $\sr(J)\leqslant\sr(I)$, $\sr(I/J)\leqslant\sr(I)$;
\item One has $\sr\Bigl(\prod\limits_{i=1}^n R_i\Bigr)=\max\limits_{i=1}^n\bigl(\sr(R_i)\bigr)$;
\item Let $J$ denote the Jacobson radical of $R$, then one has $\sr(R)=\sr(R/J)$;
\item If $I\subseteq J$, then $\sr(I)=1$;
\item Let $I\rtimes\mathbb{Z}$ denote the ring obtained by formally adjoining $1$ to $I$. Then $\sr(I\rtimes\mathbb{Z})=\max(2,\sr(I))$.
\end{enumerate}
\end{prop}
\begin{proof} See~\cite[Theorem~2.3]{Va69}, \cite{Va71}. 
\end{proof}
\begin{example}
Since the stable rank of a field equals $1$, one can conclude from the previous proposition that $\sr(R)=1$ for any semilocal ring $R$.
Other examples of rings of stable rank $1$ are provided by boolean rings, the ring of algebraic integers, the ring of entire functions, the disk-algebra, etc. For this and many other examples see~\cite{VasSR1}.

%Formally adjoining a unit to a non-unital ring of stable rank $1$ produces a non-trivial example of an ideal having the stable rank strictly lower that the ambient ring.
\end{example}

\subsection{Relative absolute stable rank}\label{sec:rel-asr}
For a row $a=(a_1,\ldots,a_n)\in{}^n\!R$ denote by $\mathfrak{J}(a)$ the intersection of all maximal ideals of $R$ containing $a_1,\ldots,a_n$.
It is easy to see that a row $a\in R^n$ is unimodular if and only if $\mathfrak{J}(a)=R$. 
Clearly, for any $g\in\GL(n,R)$ one has $\mathfrak{J}(a\cdot g)=\mathfrak{J}(a)$.
\begin{dfn}\label{dfn:j-stable}
We say that a row $a=(a_1,\ldots,a_{n+1})\in{}^{n+1}\!I$ is {\it $\mathfrak{J}$-stable}, if there exist $c_1,\ldots,c_n\in I$ such that
$\mathfrak{J}(a_1,\ldots,a_{n+1})=\mathfrak{J}(a_1+c_1a_{n+1},\ldots,a_n+c_na_{n+1}).$
\end{dfn}
\begin{dfn}\label{dfn:asr}
We say that an ideal $I$ satisfies condition $\ASR_n(I)$ if it satisfies $\SR_n(I)$ and, moreover, any row $a\in{}^{n+1}\!I$ is $\mathfrak{J}$-stable.
\end{dfn}

It is easy to see that $\ASR_m(I)$ implies $\ASR_n(I)$ for any $n\geqslant m$. 
By definition, the \emph{absolute relative stable rank} $\asr(I)$ is the smallest natural $n$ such that $\ASR_n(I)$ holds. 
Similarly to the case of usual stable rank, we set $\asr(I)=\infty$ if no such $n$ exists.

A priori our definition of condition $\ASR_n(I)$ depends on $R$.
Below we will see that, in fact, there is no such dependence.
\begin{lemma}\label{lemma:relative-asr-unimod}
For a commutative ring $R$ and an ideal $I \leq R$ the following statements are equivalent:
\begin{enumerate} \item\label{asr-j-stable} Any row $a\in{}^{n+1}\!I$ is $\mathfrak{J}$-stable;
\item\label{asr-bak-like} For any $I$-unimodular row $(b,a_1,\ldots,a_n,d)\in\Umd(n+2,I)$ there exist $c_1,\ldots,c_n\in I$ 
 such that $(b+b',a_1+c_1d,\ldots,a_n+c_nd)$ is $I$-unimodular for any $b'\in J$, where $J=I \cdot a_1 + \ldots + I \cdot a_n + I \cdot d\leq I$.
\end{enumerate} 
\end{lemma}
\begin{proof}
Assume first that any row $a\in{}^{n+1}\!I$ is $\mathfrak{J}$-stable. 
In particular for a given $I$-unimodular row $(b,a_1,\ldots,a_n,d)\in\Umd(n+2,I)$ there exist $c_1,\ldots,c_n$ such that
\[\mathfrak{J}(a_1,\ldots,a_{n+1})=\mathfrak{J}(a_1+c_1a_{n+1},\ldots,a_n+c_na_{n+1}).\]
Therefore $(b,a_1+c_1d,\ldots,a_n+c_nd)$ is also unimodular. Of course, for any $b'\in J$ we could replace $b$ with $b+b'$ from the very start.

To show the converse take arbitrary row $(a_1,\ldots,a_n,d)\in{}^{n+1}\!I$ and consider $I$-unimodular row $(1,a_1,\ldots,a_n,d)\in\Umd(n+2,I).$
By hypothesis, there exist $c_1,\ldots,c_n$ such that \[ v=(1+b',a'_1,\ldots,a'_n)=(1+b',a_1+c_1d,\ldots,a_n+c_nd) \] is unimodular for any $b'\in J$.
Assume that there exists a maximal ideal $\mathfrak{m}\trianglelefteq R$ such that all $a'_1,\ldots,a'_n$ are contained in $\mathfrak{m}$
 but at least one of the elements $d$, $a_i$ is not.
Clearly then $d\notin\mathfrak{m}$ and $I\not\subseteq \mathfrak{m}$ (otherwise $a_i=a'_i-c_id\in\mathfrak{m}$ contrary to the assumption).
Now we can find $t\in I$ such that its image $\bar{t}$ in the residue field $R/\mathfrak{m}$ equals $-\bar{1}/\bar{d}$.
We get that $1 + b' \in \mathfrak{m}$ for $b'=td\in J$ which contradicts unimodularity of $v$.
This shows that no such $\mathfrak{m}$ may exist and, therefore, $\mathfrak{J}(a'_1,\ldots,a'_n)=\mathfrak{J}(a_1,\ldots,a_n,d)$.
\end{proof}

Obviously, the second statement of Lemma~\ref{lemma:relative-asr-unimod} does not depend on $R$, hence, as suggested by the notation, $\asr(I)$ is independent of $R$.
As another corollary of Lemma~\ref{lemma:relative-asr-unimod} we obtain the following reformulation of the absolute stable rank condition $\ASR_n(R)$.
\begin{cor}[A.~Bak, {cf.~\cite[\S~1]{Pl98}}] For a commutative ring $R$ the following statements are equivalent:
\begin{enumerate}
 \item\label{asr_Stein} $R$ satisfies $\ASR_n(R)$;
 \item\label{asr_Bak} For any \emph{unimodular} row $(a_1,\ldots,a_n,b,d)\in\Umd(n+2,R)$ there exist $c_1,\ldots,c_n\in R$ such that $(a_1+c_1d,\ldots,a_n+c_nd,b+td)$ is unimodular for all $t\in R$.
\end{enumerate}
\end{cor}

Let $R$ be a commutative ring. We denote by $\Max(R)$ its \emph{maximum spectrum} i.e. the set of maximal ideals of $R$ equipped with Zariski topology. For a topological space $X$ denote by $\dim(X)$ its usual topological dimension.
The following result gives an upper estimate for the absolute stable rank (see~\cite[Theorem~2.3]{EO}, \cite[Theorem~3.7]{MKV}).
\begin{prop}\label{prop:asr-dimmax}
Let $R$ be a commutative ring whose maximum spectrum $\Max(R)$ can be covered with a finite family of subsets 
$X_1\ldots, X_m$ such that $\dim(X_i)\leqslant d$ for every $i$. Then $\asr(A)\leqslant d+1$.
\end{prop}
%TODO: Add more comments?
From the definition of $\asr(I)$ and the previous proposition it follows that
\[ \sr(I)\leqslant\asr(I)\leqslant\asr(R)\leqslant \dim(\Max(R))+1\leqslant\dim(\Spec(R))+1. \]
\begin{rem}
There exist examples of rings for which the first inequality in the above formula is strict. Namely, there are rings of stable rank $1$ but of absolute stable rank $\geqslant2$, see~\cite{MKV}. However, \cite[Theorem~1.3]{MKV} shows that $\sr(R)=\asr(R)$ for any principal ideal ring $R$.
\end{rem}
\cref{prop:asr-dimmax} shows in particular, that every Dedekind domain $R$ satisfies $\asr(R)\leqslant2$, and that $\sr(A[x])\leqslant3$ for every locally principal ring $A$. It is shown in \cite{GruMenVasSL2ZxSL2kxy} that $\sr(\mathbb{Z}[x])=3$. Note that $\sr(\mathbb{Z}[x_1,\ldots,x_n])=n+1$ if $n>1$ \cite[Example after Theorem~18.2]{VasSusSerreProblem}.

Another corollary of \cref{prop:asr-dimmax} is an estimate for the stable rank of the polynomial rings over a field. Namely, $\sr(F[x_1,\ldots,x_n])\leqslant n+1$ for any field $F$. The estimate is known to be not always exact, for example, if $F$ is algebraic over a finite field, then $\sr(F[x_1,\ldots,x_n])\leqslant n$ \cite[Corollary~17.4]{VasSusSerreProblem}. However, $\sr(F[x,y])=3$ for every $F$ with $\K_2^M(F)\neq0$ \cite{KrusenmeyerThesis} and $\sr(F[x_1,\ldots,x_n])=n+1$ if $f\leqslant\mathbb{R}$ \cite{Va71}.

\subsection{Action of unipotent radicals}\label{sec:ur-action}
Let $v\in V=R^{2\ell}$ be a vector of the natural representation of $\G(\rD_\ell, R)$.
Denote by $v_+$ and $v_-$ the upper and the lower halves of $v$, i.e. $v_+=(v_1,\ldots, v_\ell)^t$, $v_-=(v_{-\ell},\ldots, v_{-1})^t$.
\begin{lemma}\label{lemma:asrUnip}
 Let $\Phi=\rD_\ell$ and assume that $\asr(I)\leqslant \ell -1$. 
 Then for any $I$-unimodular column $v=(v_+, v_-)^t\in\Ums(2\ell, I)$
 there exists $g\in\U(\Sigma^+_\ell, I)$ such that $(g \cdot v)_+ \in \Ums(\ell, I)$.
\end{lemma}
\begin{proof} Denote by $J$ the ideal of $R$ spanned by the components of $v_{-}$. Clearly, $J \subseteq I$.
By Proposition~\ref{prop:sr_properties} $\sr(I/J) \leqslant \ell-1$, therefore
the elementary group $\E(\rA_{\ell-1}, R/J, I/J)$ acts transitively on $\Ums(\ell, I/J)$ (see~\cite[Theorem~2.3c]{Va69}). 
This implies the existence of an element $h\in \E(\Delta_\ell, R, I)$ such that the vector $v' = h \cdot v$ satisfies $v'_i \equiv \delta_{i1} \pmod J$ for $i=1,\ldots, \ell$.

Clearly, $(v'_1,v'_{-\ell},\ldots, v'_{-1})$ is $I$-unimodular.
Applying statement~\ref{asr-bak-like} of Lemma~\ref{lemma:relative-asr-unimod} we find $c_2,\ldots, c_\ell\in I$ such that for $v''= \prod_{i=2}^{\ell}x_{-i, -1}(c_i)\cdot v'$ one has
$(v''_1, v''_{-\ell},\ldots, v''_{-2})\in\Ums(\ell+1, I)$.
%$\mathfrak{J}(v''_{-\ell},\ldots, v''_{-2}) = \mathfrak{J}(v'^-)$. Clearly, $v''_1$ is still congruent to $1$ modulo $J$, therefore 
%$(v''_1, v''_{-\ell},\ldots, v''_{-2})$ is $I$-unimodular. 
Now, applying the condition $\sr(I) \leqslant \ell-1$ once again we find
$d_1,d_3,\ldots, d_{\ell}\in I$ such that the entries $(v'''_1, v'''_{-\ell},\ldots, v'''_{-3})$
of $v'''=x_{-2,1}(d_1) \cdot \prod_{i=3}^{\ell} x_{-2,-i}(d_i) \cdot v''$ form an $I$-unimodular column.

We can find $f_1, f_3,\ldots, f_\ell \in R$ such that $f_1v'''_1+\sum_{i=3}^\ell f_i v'''_{-i} = 1$.
Set $\xi = v'''_1-v'''_2-1 \in I$, $v^{(4)}=x_{1,2}(\xi f_1) \cdot \prod_{i=3}^\ell x_{-i,2}(\xi f_i) \cdot v'''$.
Clearly $v^{(4)}_2 = v^{(4)}_1-1$, therefore $v^{(4)}_+$ is $I$-unimodular.
Summarizing the above, we have found $g\in \EP_\ell(R, I)$ such that $v^{(4)}=g \cdot v$
and the assertion of the lemma immediately follows from Levi decomposition. \end{proof}

\begin{lemma} \label{lemma:uraction} 
Let $\Phi=\rA_\ell, \rD_\ell$. Denote by $\pi$ the natural representation of $\G(\Phi, R)$ on $V=R^n$, $n=\ell+1,2\ell$.
Assume that one of the following conditions holds:
\begin{itemize}
 \item $\Phi=\rA_\ell$, $\Gamma=\{ k+1, \ldots, \ell+1\} \subset \Lambda(\pi)$ and $\sr(I)\leq k\leq \ell$;
 \item $\Phi=\rD_\ell$, $\Gamma=\{-\ell,\ldots, -2, -1\} \subset \Lambda(\pi)$ and $\asr(I)\leq \ell-1$. 
\end{itemize}
Then for $g\in \G(\Phi, R, I)$ there exist $x\in \U(\Phi^+, I)$, $y\in \U(\Phi^-, I)$ such that $(yxg \cdot v^+)_\lambda = 0$ for $\lambda\in \Gamma$.
\end{lemma}
\begin{proof} Denote by $v$ the image of the highest weight vector $v^+$ under $g$.

\textsc{Case $\Phi=\rA_\ell$.} From the definition of the relative stable rank it follows that we can find 
$x= \left(\begin{smallmatrix} e_k & a \\ * & e_{n-k} \end{smallmatrix}\right) \in \U(\Sigma_k^+, I)$ such that 
the upper $k$ components of $v'= x \cdot v$ form an $I$-unimodular column. 
Now, to obtain zeroes at desired positions it remains to subtract from $v'_{k+1},\ldots, v'_{\ell+1}$ suitable multiples of $v'_1,\ldots v'_k$.
Clearly, this operation corresponds to the left multiplication by some element $y\in\U(\Sigma_k^-, I)$.

\textsc{Case $\Phi=\rD_\ell$.} From the proof of Lemma~\ref{lemma:asrUnip} it follows that there exists $h_1 \in \EP_\ell(R, I)$ such that for $v'=h_1\cdot v$ one has $v'_2=v'_1-1\in I$.
Clearly, for $v'' = z_{-\alpha\ssub{2}}(-v'_2, 1)\cdot v'$ one has $v''_1=1$ hence by Chevalley---Matsumoto lemma there exists $h_2 \in \U(\Phi^-, I)$ such that the element
$g'=h_2 \cdot z_{-\alpha\ssub{2}}(-v'_2, 1) \cdot h_1 \cdot g$ fixes $v^+$. 
Using Levi decomposition we can write $g'=h \cdot y \cdot x \cdot g$ for some $y\in\U(\Sigma^-_\ell, I)$, $x \in \U(\Sigma^+_\ell, I)$, $h\in\E(\Delta_\ell, R, I)$.
It is clear that $x$, $y$ are the desired elements.
\end{proof}

\section{Relative parabolic factorizations} \label{sec:factorizations}
%TODO: Add references to works of Dennis, Bass, Vaserstein, Suslin etc.
In this section we formulate and prove relative versions of decompositions from~\cite{St78} which will be our main technical tools throughout the next section.

\subsection{Relative Bass---Kolster decompositions}\label{sec:bass-kolster}
The next theorem is a relative version of the so called Bass---Kolster decomposition (cf.~\cite[Theorem~2.1]{St78}).
\begin{thm}\label{thm:BassKolster}
Let $\Phi$ be a classical root system of rank $\ell\geqslant2$, let $R$ be arbitary commutative ring and $I$ be an ideal, satisfying one of the following assumptions:
\newline \indent \begin{tabular}{l l l}      
\textbullet & $\Phi = \rA_\ell$, $\ell\geq 2$, & $\sr(I) \leqslant \ell$; \\     
\textbullet & $\Phi = \rC_\ell$, $\ell\geq 2$, & $\sr(I) \leqslant 2\ell-1$; \\ 
\textbullet & $\Phi = \rB_\ell$, $\rD_\ell$ $\ell\geq 3$,  & $\asr(I) \leqslant \ell-1$. \end{tabular}

\noindent Then the principal congruence subgroup $\G(\Phi,R,I)$ admits the following relative version of Bass---Kolster decomposition:
\[\G(\Phi,R,I)=  \U(\Phi^+,I) \cdot \U(\Phi^-,I) \cdot Z \cdot \U(\Sigma_1^-\setminus\{-\alpha_\mathrm{max}\},I) \cdot \U(\Sigma_1,I) \cdot \G(\Delta_1,R,I)\]
where $\Z=\left\{ z_{-\alpha_\mathrm{max}}(r,1)\mid r\in I \right\}$.
\end{thm}
\begin{proof}

Let $g$ be an element of $\G(\Phi, R, I)$. Set $v=g \cdot v^+\in\Ums(n, I)$. 
Notice that in each of the cases it suffices to find $g' \in \U(\Phi^-, I) \cdot \U(\Phi^+, I) \cdot g$ such that 
\begin{equation} \label{eq1} (g'\cdot v^+)_{1} = 1 + s \text{ and } (g'\cdot v^+)_{\varpi\ssub{1}-\alpha\ssub{max}} = s\ \text{for some}\ s\in I. \end{equation}
Indeed, set $g'' = z_{-\alpha\ssub{max}}(-s, 1) \cdot g'$.
Obviously, one has $(g''\cdot v^+)_1 = 1$, $(g''\cdot v^+)_{\varpi\ssub{1}-\alpha\ssub{max}}=0$ and the conclusion of the theorem follows from Lemma~\ref{lemma:Chevalley-Matsumoto}.

\textsc{Case $\Phi=\rA_\ell$, $n=\ell + 1$:}
%Set $v=(1+v_1,v_2,\ldots,v_\ell,v_{\ell+1})^t\in\Ums(\ell+1,R,I)$.
Thanks to the relative stable rank condition one can add suitable multiples of the last component $v_{\ell+1}$ to the first $\ell$ components of $v$ so that the upper
$\ell$ coefficients of the resulting vector $v'$ form an $I$-unimodular column of length $\ell$.
Now multiplying $v'$ by a suitable $y\in \U(\Sigma_\ell^-, I)$ we obtain equalities~\ref{eq1}.

\textsc{Case $\Phi=\rC_\ell$, $n=2\ell$:}
Notice that column $(v_1,\ldots, v_{-2}, v_{-1}^2)^t$ is also $I$-unimodular.
Applying condition $\sr(I)\leq 2\ell-1$ we find $c_1, c_2, \ldots, c_{-2} \in I \cdot v_{-1}$ such that upper $2\ell -1$ components of $v'=(v_1 + c_1 v_{-1}, \ldots, v_{-2} + c_{-2}v_{-1}, v_{-1})^t$ form an $I$-unimodular column.
By the choice of $c_i$ we can find suitable $d\in I$ such that $h_1 \cdot v = v'$ for $h_1 = t_{1,-1}(c_1 + d) \cdot \prod_{i=2}^{-2} t_{i,-1}(c_i) \in \U(\Sigma_1^-, I)$.

We can find $f_1, f_2,\ldots, f_{-2} \in R$ such that $f_1v'_1+\sum_{i=2}^{-2} f_i v'_i = 1$.
%TODO: Determine exact sign
Set $\xi = v''_1-v''_{-1}-1 \in I$, $h_2 = t_{-1,1}(\xi f_1 \pm \sum_{i=2}^\ell v_1' \xi^2 f_i f_{-i}) \cdot \prod_{i=2}^{-2} t_{-1,i}(\xi f_i) \in \U(\Sigma_1, I)$.
Direct computation shows that $v'' = h_2 \cdot v'$ satisfies equalities~\ref{eq1}.

%Now we can assume that the first $2\ell-1$ entries of $v$ are unimodular and find $c_1,\ldots,c_{-2}\in I$ such that $c_1v_1+\ldots+c_{-2}v_{-2}=(v_1-1)-v_{-1}$. Add $c_{-i}v_i$ to $v_{-1}$, $i=2,\ldots,\ell$:
%\[ (v_1,v_2\ldots,v_\ell,v_{-\ell},\ldots,v_{-2},v_{-1})^t\longmapsto (v_1,v'_2\ldots,v'_\ell,v_{-\ell},\ldots,v_{-2},v'_{-1})^t. \]
%Then add $c_iv_i'=c_i(v_1+c_{-i}v_{-i})$, $i=2,\ldots,\ell$ to the last entry:
%\[ v_1\longmapsto v_1,\quad v'_{-1}\longmapsto v_{-1}+\sum_{i=2}^\ell c_{-i}v_{-i}+\sum_{i=2}^\ell c_i(v_i+c_{-i}v_1)=v''_{-1}. \]
%Next add $\left(c_1-\sum_{i=2}^\ell c_ic_{-i}\right)v_1$ to $v''_{-1}$ to get $v_1-1$ in position $-1$.
%Again, as in case of $\rA_n$, apply $z_\gamma(1-v_1,1)$ with $\gamma=-\alpha_\mathrm{max}$, to get $1$ as the first entry and $0$ as the last.

\textsc{Case $\Phi=\rD_\ell$, $n= 2\ell$.} 
By Lemma~\ref{lemma:asrUnip} we can find $h_1\in \U(\Sigma^+_\ell, I)$ such that the upper half $v'_+$ of $v'=h_1 \cdot v$ is $I$-unimodular.
Since $\sr(I)\leq \ell-1$ we can find $c_1$, $c_3, \ldots c_\ell \in I$ such that $(v''_1, v''_3, \ldots, v''_\ell) \in \Ums(\ell-1, I)$, where
$v''=h_2 \cdot t_{1,2}(c_1) \cdot v'$, $h_2=\prod_{i=3}^\ell t_{i,2}(c_i)$.

We can find $f_1, f_3,\ldots, f_\ell \in R$ such that $f_1v''_1+\sum_{i=3}^\ell f_i v''_{i} = 1$.
As before, set $\xi = v''_1-v''_{-2}-1 \in I$, $h_3 = t_{1,-2}(\xi f_1) \cdot \prod_{i=3}^\ell t_{i,-2}(\xi f_i)$, $v'''=h_3 \cdot v''$.
%Clearly, $v'''_{-2}=v'''_1-1$, therefore for $v_4 = z_{-\alpha_{max}}(-v'''_{-2}, 1) \cdot v'''$ one has $v^4_1 = 1$, as required.
Clearly, $t_{1,2}(c_1) \cdot h_1 \in \U(\Phi^+, I)$, while $ h_3 \cdot h_2 \in \U(\Phi^-, I)$.

\textsc{Case $\Phi=\rB_\ell$, $n=2\ell+1$:} Subdivide $v\in \Ums(2\ell+1, I)$ as $v=(v_+, v_0, v_-)\in R^\ell\times R\times R^\ell$.
Denote by $J\leq I$ the ideal spanned by components of $v_-$.
Since $\sr(I/J)\leq \ell$ we can find $c_1,\dots,c_\ell\in I$ such that for $v' = h \cdot v$, $h = \prod_{i=1}^\ell t_{i, 0}(c_i) \in \U(\Phi^+, I)$
one has $\bar{v'}_+=(\bar{v'_1},\ldots, \bar{v'_\ell}) \in \Ums(\ell, I/J)$ and, therefore, $(v'_+, v'_-) \in \Ums(2\ell, I)$.
Now the proof can be finished by repeating the argument for the case $\Phi=\rD_l$ (applied to the subset of long roots of $\rB_\ell$).
%(clearly, the maximal root of $\rD_\ell$ maps to the maximal root of $\rB_\ell$ under the natural embedding $\rD_\ell\subseteq\rB_\ell$). 
\end{proof}

We can estimate the number of elementary root unipotents involved in the decomposition of the previous theorem.
\begin{lemma}
In the assumptions and notation of Theorem~\ref{thm:BassKolster} one can write any element $g$ of 
$\G(\Phi,R,I)$ as a product of an element of $\G(\Delta_1,R,I)$, element $z_{-\alpha}(s, 1)$ for some $s\in I$ and $n(\Phi)$ elementary unipotents $t_{\alpha_i}(s_i)$, where
\begin{itemize}
 \item $n(\rA_\ell)= 4 \ell -1$;
 \item $n(\rC_\ell)= 8 \ell - 5$;
 \item $n(\rD_\ell)= $
\end{itemize}


\cdot \U(\Sigma_1,I)\cdot \U(\Sigma_1^-,I)\cdot\Z \cdot \U(\Sigma^-_1,I)\cdot \U(\Sigma_1, I).\]
\end{lemma}


\subsection{Relative Dennis---Vaserstein decompositions}\label{sec:dennis-vaserstein}
Let $\Phi$ be an irreducible root system of rank $\ell$.
In what follows, $r$ and $s$ are two distinct integers such that $1 \leq r, s \leq \ell$.
From Levi decomposition~\eqref{rel:Levi-decomp} it follows that the following four subsets of $\E(\Phi, R, I)$ are equal:
\begin{multline*}
A_{rs} := \U(\Phi^+, I)\cdot \U(\Phi^-, I) \cdot \E(\Delta_r, R, I) \cdot \EP_s(R, I) = \\
= \U(\Sigma_r, I)\cdot \U(\Sigma^-_r, I) \cdot \E(\Delta_r, R, I) \cdot \EP_s(R, I) = \hspace{5em} \\
\hspace{5em} = \EP_r(R, I) \cdot \E(\Delta_s, R, I) \cdot \U(\Sigma_s^-, I)\cdot \U(\Sigma_s, I) = \\
= \EP_r(R, I) \cdot \U(\Sigma^-_r \cap \Sigma^-_s, I) \cdot \EP_s(R, I).
\end{multline*}

%The relative elementary subgroup $\E(\Phi, R, I)$ admits Dennis---Vaserstein decomposition, i.\,e. $\E(\Phi, R, I) = A_{rs}$ under the following assumptions on $I$ and $\Phi$.
%\[\begin{array}{l@{\qquad}c@{\quad}c@{\qquad}l}
%\Phi                                 & s    & r      & \text{ring condition} \\ \hline\vphantom{\Bigl(}
%\rA_\ell,\ \ell\geqslant 2           & 1    & \ell   & \sr(I) \leqslant \ell-1 \\    
%\rB_\ell,\ \ell\geqslant 2           & 1    & \ell   & \sr(I) \leqslant \ell-1 \\
%\rC_\ell,\ \ell \geqslant 2          & 1    & \ell   & \sr(I) \leqslant \ell-1 \\
%\rD_\ell,\ \ell\geqslant 4           & 1    & \ell   & \sr(I) \leqslant \ell-2 \\ 
%\rD_\ell,\ \ell\geqslant 4           & \ell & \ell-1 & \sr(I) \leqslant 2      \\
%\rE_\ell,\ \ell=6,7                  & \ell & 2      & \sr(I) \leqslant \ell-3 \\ 
%\rE_\ell,\ \ell=6,7                  & \ell & 1      & \asr(I)\leqslant \ell-2 \\ 
%\rF_4                                & 4    & 1      & \asr(I)\leqslant 3 \end{array}\]
Denote by $N_{rs}$ the submonoid of $\E(\Phi, R)$ consisting of elements which normalize the subset $A_{rs}$, i.\,e.
\[ N_{rs} = \left\{ g\in \E(\Phi, R)\ \middle|\ g \cdot A_{rs} \cdot g^{-1} \subseteq A_{rs} \right\}. \]

\begin{lemma}\label{lemma:dv-normal} 
For every $\alpha \in \Delta_{\{r,s\}} \cup (\Phi^+ \setminus (\Sigma^+_r \cap \Sigma^+_s))$ one has $X_\alpha(R) \subseteq N_{rs}$. \end{lemma}
\begin{proof}
Notice that for every $i$ the group $\EP_i(R, I)$ is normalized by $\EP_i(R)$, hence a fortiori it is normalized by $X_\alpha(R)$, $\alpha \in S_i^+$.
Since $\U(\Sigma_r^- \cap \Sigma_s^-, I)$ is normalized by $\E(\Delta_{\{r,s\}}, R)$ we obtain the assertion of the lemma for $\alpha \in \Delta_{\{r, s\}}$.

Now if $\alpha$ lies in $\Phi^+ \setminus \Sigma_r \cap \Sigma_s$ it is contained either in $\Delta_r$ or in $\Delta_s$.
Assume, for example, the latter. By Levi decomposition we have $\U(\Sigma_r^- \cap \Sigma_s^-, I)^{X_\alpha(R)} \subseteq \U(\Sigma_s^-, I) \subseteq A_{rs}$
which again implies the assertion of the lemma.
\end{proof}

\begin{proof}[Proof of \cref{thm:DennisVaserstein}]
First of all, notice that in the case $\Phi=\rG_2$ the assertion of the theorem formally follows from \cref{thm:srRI1} and identities~\ref{rel:h-comm}--\ref{rel:h-w}.
Since $A_{rs} = \E(\Phi, R, I)$ implies $A_{sr} = \E(\Phi, R, I)$, it suffices to consider only the following possibilities for $\Phi$, $s$, $r$.
\begin{table}[htb]
\[\begin{array}{l @{\qquad} l @{\qquad} c @{\quad} c @{\quad} c @{\qquad} c @{\qquad} c @{\qquad} c}
\Phi                                 & s    &r      & |\Lambda(\pi)| & \text{type of $\pi$} & \text{type of $\Delta_r$} & |\Lambda(\pi')|& |\Gamma|  \\ \hline\vphantom{\Bigl(}
\rA_\ell,\ \ell\geqslant 2           & 1    &\ell   & \ell+1         & \text{natural}       & \rA_{\ell-1}              & \ell           & 1  \\     
\rB_\ell,\ \ell\geqslant 2           & 1    &\ell   & 2\ell+1        & \text{natural}       & \rA_{\ell-1}              & \ell           & 1  \\     
\rC_\ell,\ \ell\geqslant 2           & 1    &\ell   & 2\ell          & \text{natural}       & \rA_{\ell-1}              & \ell           & 1  \\
\rD_\ell,\ \ell\geqslant 4           & 1    &\ell   & 2\ell          & \text{natural}       & \rA_{\ell-1}              & \ell           & 2  \\ 
\rD_\ell,\ \ell\geqslant 4           & \ell &\ell-1 & 2^{\ell-1}     & \text{half-spinor}   & \rA_{\ell-1}              & \ell           & \ell-2  \\
\rE_\ell,\ \ell=6,7                  & \ell &2      & 27, 56         & \text{minimal}       & \rA_{\ell-1}              & \ell           & 3       \\ 
\rE_\ell,\ \ell=6,7                  & \ell &1      & 27, 56         & \text{minimal}       & \rD_{\ell-1}              & 2(\ell-1)      & \ell-1  \\
\rF_4                                & 4    &1     & 26             & \text{minimal}       & \rC_3                     & 6              & 3\end{array}\]
 \caption{List of the cases considered in the proof of \cref{thm:DennisVaserstein}.} \label{table:dv-reps}
\end{table}

Denote by $L_{rs}$ the set consisting of all the elements $g\in \E(\Phi, R, I)$ such that $g \cdot A_{rs} \subseteq A_{rs}$.
It is easy to see that $L_{rs}$ contains $\EP_r(R, I)$ and is normalized by $N_{rs}$, i.\,e. ${L_{rs}}^{N\ssub{rs}}\subseteq L_{rs}$. Indeed, for $g\in N_{rs}$, $h\in L_{rs}$ one has
\begin{equation}\label{rel:NnormL} h^g \cdot A_{rs} = g^{-1} \cdot h \cdot g \cdot A_{rs} \subseteq g^{-1} \cdot h \cdot A_{rs} \cdot g \subseteq {A_{rs}}^g \subseteq A_{rs}.\end{equation}

The first step of the proof is to show that $N_{rs}$ contains sufficiently many elements. 
This is accomplished in~\cref{lemma:Stein_reduction} below, where we invoke stability conditions for the first time.
We show that $N_{rs}$ contains the root subgroup $X_{-\alpha\ssub{r}}(R)$, which together with \cref{lemma:dv-normal} and \cref{item-egen} immediately implies that $N_{rs}$ contains the Levi subgroup $\E(\Delta_s, R)$ and hence the extended Weyl group $\widetilde{W}(\Delta_s)$.

The next goal is to demonstrate the inclusion $\U(\Sigma_s, R) \subseteq N_{rs}$ (which implies $\EP_s(R) \subseteq N_{rs}$).
All the possibilities for $\Phi$, $s$ and $r$ fall into one of the following three cases:
\begin{enumerate}
 \item \textit{Case $m_s(\alpha_{\mathrm{max}})=1$ and $\Phi$ is simply laced.}
  In this case all the roots of $\Sigma^+_s$ have the same $s$-shape and length, hence the assertion follows from \cref{item-trans2}, \eqref{rel:NnormL} and the fact that we already have the inclusion $X_{\alpha_s}(R) \subseteq N_{rs}$.
 \item \textit{Case $\Phi = \rB_\ell$.} Using the argument from the previous case we get that $N_{rs}$ contains the subgroup generated by the root subgroups $X_\alpha(R)$, where $\alpha$ varies over long roots of $\Sigma_s^+$. 
  It remains to prove the inclusion $X_{1,0}(R) \subset N_{rs}$.
  Specializing identity~\eqref{rel:CCF} we get \begin{equation*} \label{rel:CCF-specBC} x_{1,0}(ab) = [x_{1, 2}(a), x_{2, 0}(b)] \cdot x_{1,-2}(-a b^2) \end{equation*}
  Since $X_{2, 0}(R)$ and $X_{1,-2}(R)$ are contained in $N_{rs}$, the inclusion $X_{1,0}(R) \subseteq N_{rs}$ follows.
 \item \textit{Case $\Phi = \rC_\ell, \rF_4$.}
 We settle these remaining cases in \cref{lemma:DVST} where we invoke the stability conditions one more time. 
 In fact, we prove a stronger result, namely that $N_{rs}$ contains $X_{-\alpha\ssub{s}}(R)$ and, consequently, $N_{rs} = \E(\Phi, R)$.
\end{enumerate}

In view of \cref{prop:Stepanov} to prove the theorem it suffices to show that $L_{rs}$ contains the generating set $\mathcal{Z}(\Sigma_s^-)$ of $\E(\Phi, R, I)$.
From $\mathcal{Z}(\Sigma_s^-) \subseteq \mathcal{X}^{\EP_s(R)}$ and \eqref{rel:NnormL} it follows that we only have to show $\mathcal{X} \subseteq L_{rs}$.

The last inclusion follows from the same transitivity argument as above and the fact that $X_\alpha(I) \subseteq \EP_r(R, I) \subseteq L_{rs}$ for $\alpha \in \Delta_r \cup \Sigma_r$.
For $\Phi=\rB_\ell$ one has to once again apply Chevalley commutator formula~\eqref{rel:CCF}:
\[ X_{0,1}(I) \subseteq X_{2,1}(I) \cdot X_{2,1}(I)^{x_{0,2}(1)} \cdot X_{-2,1}(I) \subseteq L_{rs} \cdot L_{rs}^{N_{rs}} \cdot L_{rs} \subseteq L_{rs}. \qedhere \] 
\end{proof}

\begin{lemma}\label{lemma:dv_unipotent} For any $1\leq i\leq n$ the following statements hold. 
\begin{thmlist} \item \label{item-dvu1} $\U(\Phi^+, I) = X_{\alpha\ssub{i}}(I)\cdot \U(\Phi^+\setminus\{\alpha\ssub{i}\}, I) = \U(\Phi^+\setminus\{\alpha\ssub{i}\}, I)\cdot X_{\alpha\ssub{i}}(I)$.
\item \label{item-dvu2} For any $\xi\in R$ one has $\U(\Phi^+\setminus\{\alpha_i\}, I)^{x_{-\alpha\ssub{i}}(\xi)} \subseteq \U(\Phi^+, I)$.
\item \label{item-dvu3} $\U(\Phi^+, I)\cdot \U(\Phi^-, I) \subseteq \U(\Phi^+\setminus \{\alpha_i\}, I) \cdot \U(\Phi^-, I) \cdot X_{\alpha\ssub{i}}(I) \cdot X_{-\alpha\ssub{i}}(I)$.
\end{thmlist} \end{lemma}
\begin{proof} The first two statements easily follow from Chevalley commutator formula~\eqref{rel:CCF} while the third one is a formal consequence of the first two. \end{proof}

\begin{lemma}\label{lemma:Stein_reduction}
Under the assumptions of \cref{thm:DennisVaserstein} one has $X_{-\alpha_r}(R) \subseteq N_{rs}.$
\end{lemma}
\begin{proof}
Let $\pi$ be the irreducible representation of $\G(\Phi, R)$ with the highest weight $\varpi_s$ (see Table~\ref{table:dv-reps}).
Notice that $\Delta_r$ is an irreducible classical root system of type $\rA_\ell$, $\rC_\ell$ or $\rD_\ell$.
Denote by $(\pi', V')$ the irreducible component of the restriction of $\pi$ to $\G(\Delta_r, R)$ containing the highest weight vector $v^+$ of $\pi$.
In all the cases under consideration, $\pi'$ is isomorphic to the natural representation of $\G(\Delta, R)$.
Set $\Gamma = \{\lambda \in \Lambda(\pi') \mid \lambda - \alpha_r \in \Lambda(\pi) \}.$

The set $\Gamma$ can be visualized using weight diagrams in the following manner.
After removing all bonds marked $r$ the weight diagram of $\pi$ splits into several connected components.
The subset $\Lambda(\pi') \subseteq \Lambda(\pi)$ corresponds to the component of the diagram containing the vertex marked $\varpi_s$.
Clearly, $\Gamma$ consists of the weights of $\Lambda(\pi')$ incident to the removed bonds.

From a consideration of the weight diagrams (see~\cite{PSV98}) we determine the number of elements in $\Lambda(\pi')$ and $\Gamma$ (see~Table~\ref{table:dv-reps}).

Consider the subset $B \subset \E(\Delta_r, R, I)$ consisting of elements $g$ such that $(g \cdot v^+)_\lambda = 0$ for all $\lambda\in\Gamma$.
Set $A:=\U(\Phi^+, I)\cdot \U(\Phi^-, I) \cdot B \cdot \EP_s(R, I).$

Applying \cref{lemma:uraction} to the subsystem $\Delta_r$ we find
$x\in\U(\Delta_r^+, I)$, $y\in \U(\Delta_r^-, I)$ such that $yx\cdot g \in B$.
Consequenty, this shows that $A = A_{rs}$, indeed:
\begin{equation*} \U(\Sigma^+_r, I) \cdot \U(\Sigma^-_r, I) \cdot g = \U(\Sigma^+_r, I) x^{-1} \cdot \U(\Sigma^-_r, I)^{x^{-1}} y^{-1} (yxg) \subseteq \U(\Phi^+, I) \cdot \U(\Phi^-, I) \cdot B. \end{equation*}

Notice that by the definition of $\Gamma$ and \cref{lemma:Matsumoto} for any $s\in I$, $ g\in B$ one has $x_{-\alpha_r}(s) \cdot g \cdot v^+ = g \cdot v^+$.
Consequently, one has
\[ X_{-\alpha\ssub{r}}(I)^{B} \subseteq \U(\Phi^-, I) \cap \Stab(v^+) \subseteq \U(\Delta_s^-, I) \subseteq \EP_s(R, I), \]
which implies the following inclusion
\begin{multline*}
X_{\alpha_r}(I) \cdot X_{-\alpha_r}(I) \cdot B \cdot \EP_s(R, I) \subseteq 
X_{\alpha_r}(I) \cdot B \cdot \EP_s(R, I) \cdot \EP_s(R, I) \subseteq \\
\subseteq B \cdot \U(\Sigma_r, I) \cdot \EP_s(R, I) =
B \cdot \EP_s(R, I).
\end{multline*}
Together with the third statement of \cref{lemma:dv_unipotent} this shows that
\begin{equation*} \label{rel:sred}
A = \U(\Phi^+\setminus\{\alpha_r\}, I) \cdot \U(\Phi^-, I) \cdot B \cdot \EP_s(R, I).
\end{equation*}
Finally, since $[B, X_{-\alpha_r}(R)] \subseteq \U(\Sigma_r^-, R) \cap \E(\Phi, R, I) = \U(\Sigma_r^-, I)$ we obtain the assertion of the lemma, indeed
\begin{equation*} \label{rel:ninv} A^{X_{-\alpha\ssub{r}}(R)} = \U(\Phi^+, I) \cdot \U(\Phi^-, I) \cdot B ^{X_{-\alpha\ssub{r}}(R)} \cdot \EP_s(R, I) = A. \qedhere \end{equation*}
\end{proof}

\begin{lemma}\label{lemma:DVST}
Assume that one of the following holds:
\begin{lemlist}
 \item \label{lemma:DVcaseCl} $\Phi=\rC_\ell$ and $\sr(I) \leq \ell-1$;
 \item \label{lemma:DVcaseF4} $\Phi=\rF_4$ and $\asr(I) \leq 3$.
\end{lemlist}
Then one has $X_{-\alpha_s}(R) \subseteq N_{rs}$.
\end{lemma}
\begin{proof}
Consider the subset $C \subseteq \EP_s(R, I)$ consisting of elements satisfying the following condition:
\begin{itemize}
 \item \textsc{Case $\Phi=\rC_\ell$.} Matrix entries $(g_{i,2})$, $i=2,\ldots, \ell$ form an $I$-unimodular column of height $\ell-1$.
 \item \textsc{Case $\Phi=\rF_4$.} Matrix entries $(g_{\lambda, \lambda}, \ldots, g_{\lambda, \lambda - \alpha_3 - \alpha_2 - \alpha_1})$, where $\lambda = \varpi_4 - \alpha_4$ (see~\cite[Fig.~26]{PSV98}),
                                     form an $I$-unimodular column of height $4$. \end{itemize}

Set $A' = \EP_\ell(R, I) \cdot \U(\Sigma_s^- \cap \Sigma_r^-, I) \cdot C$.
By \cref{item:asrUnipC} (in the case $\Phi=\rC_\ell$) or \cref{cor:embeddingBD} (in the case $\Phi=\rF_4$) applied to the subgroup $\G(\Delta_s, R, I)$ 
we find for every $g \in \EP_s(R, I)$ an element $x \in \U(\Sigma_r \cap \Delta_s, I)$ such that $xg \in C$.  
Notice that one immediately gets the equality $A_{rs} = A'$ from this.
Indeed for $g\in \EP_s(R, I)$ one has
\begin{equation*} \EP_r(R, I) \cdot \U(\Sigma_s^- \cap \Sigma_r^-, I) \cdot g \subseteq 
 \EP_r(R, I)x^{-1}  \cdot \U(\Sigma_s^-, I) \cdot xg \subseteq A'. \end{equation*}

By the very definition of $C$, for every $g \in C$ one can choose $y \in \U(\Sigma_s \cap \Delta_r, I)$ such that $(y \cdot g)_{\varpi_s,\varpi_s - \alpha_s} = 0$.
Consequently, for every $g\in C$ one has
\begin{multline*}
 \EP_r(R, I) \cdot \U(\Sigma_s^- \cap \Sigma_r^-, I) \cdot g \subseteq \EP_r(R, I) y^{-1} \cdot \U(\Sigma_r^-\cap \Sigma_s^-, I)^{y^{-1}} \cdot y g \subseteq \\
  \subseteq \EP_r(R, I) \cdot \U(\Sigma_r^-, I) \cdot y g
\end{multline*}
Notice that the matrix entry $(yg)_{\varpi_s,\varpi_s}$ is invertible.
From the choice of $y$ it follows that for every $\xi\in R$ the element $g_1:=(yg)^{x_{-\alpha_s}(\xi)}$
satisfies the assumptions of \cref{lemma:Chevalley-Matsumoto} and therefore can be rewritten as $g_1 = uh$ for some $u \in \U(\Sigma_s^-, I)$, $h \in \EP_1(R, I)$.
Consequently, one has
\begin{multline*} {A_{rs}}^{X_{-\alpha_{s}}(R)} \subseteq \EP_r(R, I)^{X_{-\alpha_{s}}(R)} \cdot \U(\Sigma_r^-, I)^{X_{-\alpha_{s}}(R)} \cdot \U(\Sigma_s^-, I) \cdot \EP_s(R, I) \subseteq \\
 \subseteq \EP_\ell(R, I) \cdot \U(\Phi^-, I) \cdot \EP_s(R, I) \subseteq A_{rs} \end{multline*}
 as claimed.
\end{proof}
\subsection{Relative Gauss decomposition}\label{sec:gauss}
The proof of Gauss decomposition presented below is similar to the absolute case (cf.~\cite[Theorem~5.1]{Sm12}).
It is also similar to a result of E.~Abe (cf.~\cite[Proposition~2.8]{Abe69}).

\begin{prop}\label{thm:Gauss}
Let $\Phi$ be a reduced irreducible root system of rank $\ell$ and let $\Delta_1, \Delta_\ell$ be
two reductive subsystems of $\Phi$ corresponding to the endnodes of the Dynkin diagram of $\Phi$.
Suppose that both relative elementary subgroups $\E(\Delta_i, R, I)$, $i=1,\ell$ admit triangular decomposition with $N$ triangular factors:
\[ \E(\Delta_i, R, I) = \Hh(\Delta_i, R, I) \cdot \U(\Delta^+_i, I) \cdot \U(\Delta^-_i, I) \cdot \ldots \cdot \U(\Delta^\pm_i, I),\quad i=1,\ell. \]
Then the group $\E(\Phi, R, I)$ admits the decomposition with the same number of factors:
\[ \E(\Phi, R, I) = \Hh(\Phi,R,I) \cdot \U(\Phi^+,I) \cdot \U(\Phi^-, I) \cdot \ldots \cdot \U(\Phi^\pm,I). \]
\end{prop}
\begin{proof}
Denote by $Y$ the product of subgroups in the right-hand side of the above equality.
To show that $Y=\E(\Phi,R,I)$ it suffices to check that
\begin{enumerate}
\item $Y$ is normalized by $\E(\Phi,R)$, i.\,e. $Y^{\E(\Phi,R)}\subseteq Y$;
\item there exists a set $X$ generating $\E(\Phi,R,I)$ as a \emph{normal} subgroup of $\E(\Phi,R)$ such that $XY\subseteq Y$.
\end{enumerate}
To prove the first assertion it suffices to show that $Y^{x_\alpha(\xi)} \subseteq Y$ for any $\alpha\in \pm \Pi$, $\xi\in R$.
Fix a root $\alpha\in\pm\Pi$. Clearly, $\alpha \in \Delta_i$ for $i=1$ or $i=\ell$ and we can expand $Y$ as follows:
\[ Y=\Hh(\Phi,R,I) \cdot \U(\Delta_i^+,I) \cdot \ldots \cdot \U^\pm(\Delta_i,I) \cdot \U(\Sigma_i^+,I) \cdot \ldots \cdot \U(\Sigma_i^\pm,I). \]
For every  $h \in \Hh(\Phi, R, I)$ one has for some $s\in I$
\[ x_\alpha(\xi)\cdot h = h\cdot x_\alpha((1+s)\xi)=h\cdot x_\alpha(\xi)\,x_\alpha(s\xi). \]

Therefore by the assumption of the theorem
\begin{multline*}
Y^{x_\alpha(\xi)} \subseteq \Hh(\Phi,R,I) \cdot x_\alpha(-\xi) X_{\alpha}(I) \cdot \E(\Delta_i,R,I)\cdot x_\alpha(\xi) \cdot \U(\Sigma^+_i,I) \cdot \ldots \cdot \U(\Sigma^\pm_i,I) =\\
= \Hh(\Phi, R, I) \cdot \E(\Delta_i,R,I) \cdot \U(\Sigma^+_i,I) \cdot \ldots \cdot \U(\Sigma^\pm_i,I) = Y.
\end{multline*}

Now set $X=\left\{x_\alpha(\xi)\ \middle|\ \alpha\in\Pi,\ \xi\in I \right\}$. 
Every root is a conjugate of some fundamental root under the action of $W(\Phi)$.
Since $\E(\Phi,R)$ contains the normalizer of the torus, the set $X^{\E(\Phi,R)}$ contains all the generators of the group $\E(\Phi, I) = \langle x_\alpha(\xi),\  \alpha\in\Phi,\ \xi\in I \rangle$.
Finally, the inclusion $XY \subseteq Y$ follows from the fact that $\Hh(\Phi,R,I)$ normalizes every root subgroup $X_\alpha(I)$.
\end{proof}

\begin{thm}\label{thm:srRI1}
Let $\Phi$ be a root system, let $I$ be an ideal of arbitrary commutative ring $R$ such that $\sr(I)=1$.
Then the relative elementary Chevalley group admits Gauss decomposition
\[ \E(\Phi,R,I) = \Hh(\Phi,R,I) \cdot \U(\Phi,I) \cdot \U(\Phi^-,I) \cdot \U(\Phi,I). \]
\end{thm}
\begin{proof}
In view of the above proposition it suffices to prove that Gauss decomposition of length $3$ holds for $\Phi=\rA_1$.

Let $A=\begin{psmallmatrix}a & b \\ c & d\end{psmallmatrix}$ be an element of $\SL(2, R, I)$.
The first column of $A$ is $I$-unimodular, therefore there exists $z\in I$ such that $a+cz\in R^*$.
Multiplying $A$ on the left by $x_{12}(z)$ we get matrix $A'=x_{12}(z)\cdot A=\begin{psmallmatrix}a' & b' \\ c & d\end{psmallmatrix}$ with invertible top-left corner element $a'$.
After multiplying $A'$ on the left by $x_{21}(-c/a')$ and on the right by $x_{12}(-b'/a')$ we get a diagonal matrix. 
Thus we obtain relative Gauss decomposition of $A$
\begin{equation}\nonumber
A=x_{12}(-z)\cdot x_{21}(c/a')\cdot
\begin{pmatrix} \varepsilon & 0 \\ 0 & 1/\varepsilon \end{pmatrix}
\cdot x_{12}(b'/a')=x_{12}(-z)\cdot
\begin{pmatrix} \varepsilon & 0 \\ 0 & 1/\varepsilon \end{pmatrix}
\cdot x_{21}(y) \cdot x_{12}(b'/a'),
\end{equation}
where $\varepsilon\in 1+I$ and $y\in I$. \end{proof}

%\section{Applications} %\label{sec:applications}
\section{Bounded generation}\label{sec:boundgen}
We now give an application of the parabolic factorizations to the problem of bounded generation.
 
\begin{lemma}\label{lemma:srRI1_width}
In the assumptions of \cref{thm:srRI1} the width of $\E(\Phi,R,I)$ with respect to $\mathcal{Z}(\Pi)$ does not exceed $3\left|\Phi^+\right|+2\rk\Phi-1$.
\end{lemma}
\begin{proof}
Take an element $g\in\E(\Phi,R,I)$ and decompose it into $g=u_1 h v_2 u_3$, where $h\in\Hh(\Phi,R,I)$, $u_1,u_3\in\U(\Phi,I)$, $v_2\in\U(\Phi^-,I)$. 
Write $h=\prod_{i=1}^\ell h_{\alpha_i}(\varepsilon_i)$, $\varepsilon\in1+I$. 
Each $h_{\alpha_i}(\varepsilon_i)$ decomposes into $h_{\alpha_i}(\varepsilon_i) = x_{\alpha_i}(*) z_{-\alpha_i}(*,*) x_{-\alpha_i}(*)$ 
(see \eqref{eq:rel-tor-elementary}), and since the torus normalizes each of $X_\alpha(I)$ (see formula~\eqref{rel:h-w}), we have a decomposition
\[ g\in\U(\Phi,I)\cdot\prod_{i=1}^\ell\bigl(x_{\alpha_i}(*)z_{-\alpha_i}(*,*)\bigr)\cdot \U^-(\Phi,I) \U(\Phi,I), \]
and the estimate follows.
\end{proof}

The following lemma is a corollary of Theorems~5.7 and 5.8 of~\cite{LSM}.
\begin{lemma}
Let $p$ be a rational prime, let $c$, $d$ be a pair of coprime integers such that $p \perp d$.
Then under the assumption of the Generalized Riemann Hypothesis there exist infinitely many primes $q\equiv c\pmod{d}$ such that $p$ is a primitive root modulo $q$.
\end{lemma}

The following lemma is a relative version of \cite[Lemma~6]{VavSmSuUnitrEng} (see also~\cite{VseUnitrZ1p}):

\begin{lemma}\label{lemma:Z1p}
Set $R=\mathbb{Z}[\sfrac{1}{p}]$ and let $I$ be an ideal of $R$.
Under the assumption of the GRH the width of $\SL(2, R, I)$ with respect to the generating set
\[ \mathcal{Z}(\{-\alpha_1\})=X_{12}(I)\cup X_{21}(I) \cup \{z_{21}(s, \xi) \mid s\in I,\ \xi\in R\} \]
does not exceed $6$.
\end{lemma}

\begin{proof}
Clearly, $I$ is a principal ideal generated by some integer $m\in\mathbb{Z}$ not divisible by $p$.
Let $g$ be an element of $\SL(2,R,I)$. Write
\[ g=\begin{pmatrix}x & y \\ z & w\end{pmatrix},\ \text{for}\ x=p^\alpha a,\ z=p^\beta bm,\ \text{where}\ a,b,\alpha,\beta\in\mathbb{Z},\ p\nmid a,b. \]

\textsc{Case 1:} $\alpha\geqslant\beta$. 
Since $p^{\alpha-\beta}a\perp bm^2$ and $p\perp bm^2$, there exist infinitely many rational primes $q$ of the form $p^{\alpha-\beta}a+bm^2k$,
such that $p$ is a primitive root modulo $q$. 
We may assume that $q$ is prime to $b$. 
Write
\begin{align*}
g_1 & = x_{12}(mk)\cdot g =
\begin{pmatrix} p^\beta q & * \\ p^\beta bm & * \end{pmatrix}. \\
\intertext{
There exists $u\geqslant 1$ such that $p^u\equiv b\pmod q$, say $p^u=b+lq$. Then
}
g_2 & = x_{21}(ml)\cdot g_1 =
\begin{pmatrix} p^\beta q & * \\ mp^{\beta+u} & * \end{pmatrix}. \\
\intertext{
Since $g_2\equiv 1\pmod m$, we can write $p^\beta q=1+cm$ for some $c$. Now set
}
g_3 & = x_{12}\left(\dfrac{-c}{p^{\beta+u}}\right)\cdot g_2 =
\begin{pmatrix} 1 & * \\ mp^{\beta+u} & * \end{pmatrix}, \\
g_4 & = x_{21}\left(-mp^{\beta+u}\right)\cdot g_3 =
\begin{pmatrix} 1 & * \\ 0 & * \end{pmatrix}, \\
g_5 & = x_{12}\left(\dfrac{c}{p^{\beta+u}}\right)\cdot g_4 =
\begin{pmatrix} 1 & * \\ 0 & * \end{pmatrix}.
\end{align*}
Notice that $g_5=z_{21}\left(-mp^{\beta+u}, c/p^{\beta+u}\right)\cdot g_2$ hence $g=x_{12} \cdot x_{21} \cdot z_{21} \cdot x_{12}$
and the length of $g$ does not exceed $4$.

\textsc{Case 2:} $\alpha<\beta$. 
Since $\mathbb{Z}[\sfrac{1}{p}]/I$ is finite, there exists $k>0$ such that $p^k\equiv 1\pmod I$.
One can choose $k>\beta-\alpha$.
Then $k+\alpha>-k+\beta$ and
\[ h_{12}\left(p^k\right)\cdot g =
\begin{pmatrix} p^k & 0 \\ 0 & p^{-k} \end{pmatrix}
\begin{pmatrix} p^\alpha a & * \\ p^\beta bm & * \end{pmatrix}=
\begin{pmatrix} p^{k+\alpha} a & * \\ p^{-k+\beta} bm & * \end{pmatrix}. \]
We find ourselves in the situation of the previous case, therefore, we can write $g=h_{12}\cdot x_{12} \cdot x_{21} \cdot z_{21} \cdot x_{12}$.
Finally, expressing $h=x_{21}\cdot z_{21}\cdot x_{12}$ as in \eqref{eq:rel-tor-elementary}, we get that $g=x_{21} \cdot z_{21} \cdot x_{12} \cdot x_{21} \cdot z_{21} \cdot x_{12}$.
\end{proof}

For the rest of this subsection $k$ denotes a global field. We assume that there is some finite set of places $S$ chosen on $k$. 
Let $\mathcal{O}_S$ be a Dedekind ring of arithmetic type defined by $S$ and let $I$ be an ideal of $\mathcal{O}_S$.

\begin{lemma}\label{lemma:width-dedekind}
Let $\Phi$ be an irreducible classical root system of rank $\ell \geqslant 2$.
If $k$ has a real embedding, then $\G(\Phi, \mathcal{O}_S, I)$ has finite width with respect to the generating set $\mathcal{Z}(\Sigma_\ell)$.
\end{lemma}
\begin{proof}
First of all, notice that $\asr(I) \leqslant \asr(\mathcal{O}_S) \leqslant 2$. 
By \cref{cor:bass-kolster-iterated} we can present any element 
of $G=\G(\Phi, \mathcal{O}_S, I)$ as a product of a finite number of generators from $\mathcal{Z}(\Sigma_\ell)$ and one element of 
$G_0 = \G(\{\alpha_\ell, -\alpha_\ell\}, \mathcal{O}_S, I)\cong\SL(2, \mathcal{O}_S, I)$.
Consequently, to prove the statement of the lemma it suffices to express every element 
$g = \begin{psmallmatrix}1+a & b \\ c & 1+d \end{psmallmatrix} \in G_0$
as a product of a finite number of generators contained in some rank $2$ subgroup of $G$ containing $G_0$.

From $\det(g)=1$ we conclude that $a+d=bc-ad\in I^2$. 
Recall that the Vaserstein's congruence subgroup is defined as
\[ G(I, I)=\left\{ \begin{pmatrix}1+a & b \\ c & 1+d\end{pmatrix}\in\SL(2, \mathcal{O}_S)\;\middle|\; a, d\in I^2, \ b, c\in I \right\}. \]
Notice that $g_1=g\cdot z_{21}(a, 1)$ is contained in $G(I, I)$, indeed,
\[ \begin{pmatrix} 1+a & b \\ c & 1+d \end{pmatrix} \cdot \begin{pmatrix} 1-a & -a \\ a & 1+a \end{pmatrix} = \begin{pmatrix} 1+ba-a^2 & b-a-ba-a^2 \\ c+a+ad-ac & 1+bc-ac \end{pmatrix} \in G(I, I). \]
For any matrix $g'=\begin{psmallmatrix}1+a & b \\ c & 1+d\end{psmallmatrix}\in G(I, I)$ the matrix $x_{21}(-c)\cdot g'\cdot x_{12}(-b)$ lies in $\SL\left(2, \mathcal{O}_S, I^2\right)$.

By \cref{lemma:Stepanov-ideal} the group $\E\left(\Phi, \mathcal{O}_S, I^2\right)$ is contained in $\E(\Phi, I)$ for any root system $\Phi\neq\rC_\ell$ of rank $\geqslant2$.
Notice that under the assumptions of the lemma it is known that $\E(\Phi, I)$ has finite width with respect to $\mathcal{X}$, see~\cite[Theorem~3.3]{TavgenThesis}.

In remains to consider the case $\Phi=\rC_\ell$. First of all, notice that $2abc-abd\in II^{\indexbox{2}}$, indeed,
\[ \det(g_1)=a^3d-3a^2bc+a^2bd+ab^2c+a^3+a^2b+a^2d+2abc-abd+1. \]
This means that
\[ g_2=x_{21}(-a-c)\cdot g_1\cdot x_{12}(a-b)\equiv
\begin{pmatrix}
1+ab-a^2 & -ab-a^2 \\ ad-ac-abc & 1-ab+a^2
\end{pmatrix}\bmod II^{\indexbox{2}}. \]
Now for $g_3=g_2\cdot z_{12}\left(a^2-ab, 1\right)$ we have that
\begin{align*}
& g_3\equiv\begin{pmatrix} 1 & -2ab \\ -abc-a^2+ab-ac+ad & 1 \end{pmatrix}\bmod II^{\indexbox{2}}, \\
& g_4=x_{12}(2ab)\cdot g_3\equiv x_{21}\left(-abc-a^2+ab-ac+ad\right)\bmod II^{\indexbox{2}}.
\end{align*}
Thus $g_4\cdot x_{21}(*)\in\SL\left(2, \mathcal{O}_S, II^{\indexbox{2}}\right)$ is contained in $\E(\rC_\ell, I)$ by \cref{lemma:Stepanov-ideal} and therefore can be expressed as a bounded product of $x_\alpha$.
\end{proof}

\begin{proof}[Proof of \cref{thm:width}] The second assertion of the theorem is contained in \cref{lemma:srRI1_width}.
The the first and the third assertions follow from \cref{lemma:Z1p,lemma:width-dedekind}. \end{proof}
%\subsection{Subsystem factorizations}\label{sec:subsysfact}
%%The main result of \cite{NikProdDecomp} is the following
%\begin{thm*}
%Let $G$ be a classical (possibly twisted) Chevalley group of rank $n$ over a finite field. Then $G$ equals the product of at most $200$ conjugates of an $\SL_n$ subgroup.
%\end{thm*}
%As indicated in the introduction, the Dennis---Vaserstein decomposition gives the following result.
%\begin{lemma}
%Assume $\sr(I)\leqslant n-1$. Then $\SL(n+1, R, I)$ is a product of at most $5$ subgroups isomorphic to $\SL(n, R, I)$.
%\end{lemma}
%\begin{proof}
%By the Dennis---Vaserstein decomposition one can present $\SL(n+1, R, I)$ as a product
%\[ \SL(n+1, R, I) =  \Par_1\cdot X_{n1}\cdot\Par_n=S_1\U_1\cdot X_{n1}\cdot\U_n S_n, \]
%where $S_1$ and $S_n$ are two obvious embeddings of $\SL(n, R, I)$ in $\SL(n+1, R, I)$, avoiding respectively the first and the last row and column. Now $\U_1=(\U_1\cap S_n)\cdot X_{1n}$ and $U_n= X_{1n}\cdot(U_n\cap S_1)$, while $X_{1n}X_{n1}X_{1n}\in S_1^{w_{12}(1)}$.
%\end{proof}
%We will now elaborate on the case $\Phi=\rD_\ell$ to show that the Dennis---Vaserstein decomposition is suitable for handling other Chevalley groups, albeit with a stronger assumption on the base ring. 

Consider the decreasing chain $\Phi_k$, $k=1, \ldots, \lfloor \ell/2 \rfloor$ of root subsystems of $\Phi=\rD_\ell$ defined as follows.
If $2k \neq \ell$ let $\Phi_k$ be the subsystem of $\Phi$ spanned by simple roots $\alpha_{2k-1}, \ldots, \alpha_\ell$.
Clearly, such $\Phi_k$ has type $\rD_{(\ell-2k+2)}$. 
In the remaining case $2k = \ell$ set $\Phi_k = \langle \alpha_\ell \rangle \cong \rA_1$.
Now let $\beta_k$ be the maximal root of $\Phi_k$, i.\,e. $\beta_k = \alpha_\mathrm{max}(\Phi_k)$, $k=1, \ldots, \lfloor \ell/2 \rfloor$.
Denote by $B$ the set of all $\beta_k$. From the definition it is clear that elements of $B$ are mutually orthogonal to each other.
The roots $\beta_k$ can also be defined by explicit formulae:
\begin{align*}
 \beta_k =  \alpha_{2k-1} + 2\alpha_{2k}+ \ldots + 2\alpha_{\ell-2} + \alpha_{\ell-1} + \alpha_\ell, & \text{ for } k=1, \ldots, \lfloor\ell/2\rfloor-1, \\
 \beta_{\lfloor\ell/2\rfloor} = \alpha_{\ell-2}+\alpha_{\ell-1}+\alpha_\ell, & \text{ if $\ell$ is odd, } \\
 \beta_{\lfloor\ell/2\rfloor} = \alpha_\ell, & \text{ if $\ell$ is even.}
\end{align*}

\begin{lemma}\label{lemma:nikolov-weyl} There exists an element $w\in W(\rD_\ell)$ such that $w(B) \subseteq \Delta_\ell^+$. \end{lemma}
\begin{proof}
\textsc{Case $\ell=4$.} Set $w = \sigma_{\alpha\ssub{1} + \alpha\ssub{2}} \circ \sigma_{\alpha\ssub{2} + \alpha\ssub{4}}$.
Straightforward computation shows that 
$$w(\beta_1) = w(\alpha_\mathrm{max}) = \sigma_{\alpha\ssub{1} + \alpha\ssub{2}}(\alpha_1 + \alpha_2 + \alpha_3) = \alpha_3, \quad w(\beta_2) = w(\alpha_4) = \sigma_{\alpha\ssub{1} + \alpha\ssub{2}}(- \alpha_2) = \alpha_1 $$
which implies the assertion of the lemma.

\textsc{Case $\ell \geq 5$.}
Recall from \cite[Table~9]{Dy72} that for odd (resp. even) $\ell$ all maximal subsystems of type $\rA_1+\ldots+\rA_1+\rD_3$
(resp. $\rA_1+\ldots+\rA_1+\rD_4$) are conjugate under the action of $W(\Phi)$. Consequently, we can find $w\in W(\Phi)$
such that $w(\beta_k) = \alpha_{2k-1}$ for $k < \lfloor\ell/2\rfloor$ (resp. $k < \lfloor\ell/2\rfloor-1$). 
Now using transitivity of the action of $W(\rD_3)$ on roots (resp. by the same argument as in the case $\ell=4$) we can move the remaining root $\beta_{\lfloor\ell/2\rfloor}$ 
(resp. the remaining $2$ roots $\beta_{\lfloor\ell/2\rfloor-1}$, $\beta_{\lfloor\ell/2\rfloor}$) to
$\alpha_{\ell-1}$ (resp. to $\alpha_{\ell-3}$, $\alpha_{\ell-1}$) while fixing all the other $\beta_k$. \end{proof}

The following lemma is an analogue of Proposition~1 of~\cite{Nik07}.
\begin{lemma}\label{lemma:nikolov-type-dl}
Let $\Phi=\rD_\ell$, $\ell\geq 2$ and let $R$ be a commutative ring.
There exist an element $y\in\E(\Phi, R)$ and an element $w\in\widetilde{W}(\Phi)$ such that $\U(\Sigma_\ell^+, I)\subset[\U(\Delta_\ell^-, I), y]\cdot{}^w\!\U(\Delta_\ell^+, I)$.
\end{lemma}
\begin{proof}
Since $\U(\Sigma_\ell^+, I)$ is abelian we can decompose it as $\U(\Sigma_\ell^+, I)=\U(\Sigma_\ell^+\setminus B, I) \cdot \U(B, I)$. 
Set $y=\prod_{\beta\in B}x_\beta(1)$. 
We will now show by induction on $\ell$ that 
\begin{equation}\label{eq:ind-stat} \U(\Sigma_\ell^+\setminus B, I)\subset[\U(\Delta_\ell^-, I), y]\cdot\U(B, I). \end{equation}
The induction base in the cases $\ell=2, 3$ is trivial.

Notice that $\beta_1$ is the only root of $\Phi$ satisfying $m_2(\beta_1)=2$, therefore the commutator formula~\eqref{rel:CCF} implies
\[ \bigl[\U(\Delta_2^-, I), x_{\beta_1}(1)\bigr]=1. \]
There is no root of the form $\gamma=\alpha+\beta$ with $\alpha\in\Sigma_2^-\cap\Delta_\ell$ and $\beta\in B\setminus\{\beta_1\}$, because such a root $\gamma$ must satisfy $m_2(\gamma)=-1$ and $m_\ell(\gamma)=1$. Thus~\eqref{rel:CCF} gives
\[\Bigl[\U(\Sigma_2^-\cap\Delta_\ell, I), \prod_{i\neq1}x_{\beta_i}(1)\Bigr]=1. \]
Since $B\setminus\{\beta_1\}\subset\Sigma_\ell^+\cap\Delta_2$, the above two identities imply
\[ \Bigl[ \U(\Sigma_2^-\cap\Delta_\ell, I)\cdot\U(\Delta_{2, \ell}^-, I), x_{\beta_1}(1)\cdot\prod_{i\neq1}x_{\beta_i}(1) \Bigr] \equiv \bigl[ \U(\Sigma_2^-\cap\Delta_\ell, I), x_{\beta_1}(1) \bigr] \bmod \U(\Sigma_\ell^+\cap\Delta_2, I). \]
Take an element $u\in\U(\Sigma_2^-\cap\Delta_\ell, I)$ and decompose it as $u=vw$, $v\in\U(\Sigma_1^-\cap\Sigma_2^-\cap\Delta_\ell, I)$, $w\in\U(\Sigma_2^-\cap\Delta_{1, \ell}, I)$.
Using the identity
\begin{equation}\label{eq:comm-ab-c}
[ab, c]={}^a[b, c]\cdot[a, c], 
\end{equation}
we can rewrite
\[ [vw, x_{\beta_1}(1)] = {}^v[w, x_{\beta_1}(1)]\cdot[v, x_{\beta_1}(1)].  \]
Since $\U(\Sigma_1^-\cap\Sigma_2^-\cap\Delta_\ell, I)$ and $\U(\Sigma_2^-\cap\Delta_{1, \ell}, I)$ are abelian, it is easy to see that
\[ [v, x_{\beta_1}(1)]\in\U(\Sigma_2^+\cap\Sigma_\ell^+\cap\Delta_1, I), \quad [w, x_{\beta_1}(1)]\in\U((\Sigma_1^+\cap\Sigma_\ell^+)\setminus\{\beta_1\}, I). \]
Every element of $\U(\Sigma_2^+\cap\Sigma_\ell^+\cap\Delta_1, I)$ (resp. $\U(\Sigma_1^+\cap\Sigma_\ell^+\setminus\{\beta_1\}, I)$) can be expressed as such a commutator for a suitable choice of $v$ (resp. $w$).
Indeed, set $v=x_\gamma(\xi_\gamma)\cdot v'$, $\gamma=-\alpha_1-\alpha_2$, $v' \in \U(\Sigma_1^- \cap \Sigma_2^- \cap \Delta_\ell \setminus \{\gamma\})$.
Using relation \eqref{eq:comm-ab-c} and the fact that $\U(\Sigma_2^-\cap\Delta_\ell)$ commutes with $\U(\Sigma_\ell^+)$ we get that:
\begin{multline*}
[v, x_{\beta_1}(1)] = [x_\gamma(\xi_\gamma)\cdot v', x_{\beta_1}(1)] = {}^{x_\gamma(\xi_\alpha)}[v', x_{\beta_1}(1)] \cdot [x_\gamma(\xi_\gamma), x_{\beta_1}(1)] = \\
= [v', x_{\beta_1}(1)]\cdot x_{\beta_1-\alpha_1-\alpha_2}(\xi_\gamma) = \ldots = \prod_{\mathclap{\alpha\in\Sigma_1^- \cap \Sigma_2^- \cap \Delta_\ell}} x_{\beta_1+\alpha}(\xi_\alpha).
 \end{multline*}
It remains to note that $\Sigma_2^+ \cap \Sigma_\ell^+ \cap \Delta_1 = \beta_1 + \Sigma_1^- \cap \Sigma_2^- \cap \Delta_\ell$. The same argument works for $[w, x_{\beta_1}(1)]$.
Direct calculation using the commutator formula shows that
\[ {}^v\!\U(\Sigma_1^+\cap\Sigma_\ell^+\setminus\{\beta_1\}, I) \equiv \U(\Sigma_1^+\cap\Sigma_\ell^+\setminus\{\beta_1\}, I) \bmod \U(\Sigma_\ell^+\cap\Delta_2, I). \]
Summing up the above arguments, we get that
\[ [\U(\Sigma_2^-, I)\cdot\U(\Delta_{2, \ell}^-, I), y] \equiv \U((\Sigma_{1, 2}^+\cap\Sigma_\ell^+)\setminus\{\beta_1\}) \bmod \U(\Sigma_\ell^+\cap\Delta_2, I), \]
hence the inclusion~\eqref{eq:ind-stat} follows from the induction hypothesis (applied to $\Delta_{1, 2} \cong \rD_{\ell-2}$). 

Finally, we have found $a\in\U(\Sigma_\ell^+\setminus B, I)$ and $b\in\U(\Delta_\ell^-, I)$
such that $$a\in[b, y]\cdot\prod_{\beta\in B}X_\beta\subset[\U(\Delta_\ell^-, I), y]\cdot\U(B, I).$$
Now the assertion of the lemma follows from~\cref{lemma:nikolov-weyl}.
\end{proof}

\begin{proof}[Proof of \cref{thm:spin-sln-prod}]
Set $L = \E(\Delta_\ell, R, I) \leq \E(\rD_\ell, R, I)$ and denote by $\sigma$ the automorphism of $\G(\rD_\ell, R)$ induced by the sole diagram automorphism of the Dynkin diagram of $\rD_\ell$.
By \cref{thm:DennisVaserstein} one has
\begin{multline*}
\E(\rD_\ell, R, I) = \EP_{\ell}(R, I)\cdot\U(\Sigma_{\ell-1}^- \cap \Sigma_{\ell}^-, I)\cdot\EP_{\ell-1}(R, I) = \\
= L \cdot\U(\Sigma_\ell^+, I)\cdot\U(\Sigma_{\ell-1}^- \cap \Sigma_{\ell}^-, I)\cdot \big(L \cdot \U(\Sigma_\ell^+, I) \big)^\sigma.
\end{multline*}  
Now using \cref{lemma:nikolov-type-dl} one can find $y_1, y_2\in\G(\rD_\ell, R)$ and $w_1, w_2\in\widetilde{W}(\rD_\ell)$ such that
\begin{align*} & L \cdot \U(\Sigma_\ell^+, I) \subset L \cdot \U(\Delta_{\ell-1}^-, I) \cdot {}^{y_1}\!\U(\Delta_{\ell-1}^-, I) \cdot {}^{w_1}\!\U(\Delta_{\ell-1}^+, I), \\
& \U(\Sigma_{\ell-1}^-\cap\Sigma_\ell^-, I) \subset \U(\Delta_\ell^+, I) \cdot {}^{y_2}\!\U(\Delta_\ell^+, I) \cdot {}^{w_2}\!\U(\Delta_\ell^-, I). \end{align*} 
Thus $\E(\rD_\ell, R, I)$ is a product of at most $9$ conjugates of $L \cong \E(\rA_{\ell-1}, R, I)$. \end{proof}

\printbibliography

\end{document}
