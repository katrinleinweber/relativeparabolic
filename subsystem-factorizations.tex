The main result of \cite{NikProdDecomp} is the following
\begin{thm*}
Let $G$ be a classical (possibly twisted) Chevalley group of rank $n$ over a finite field. Then $G$ equals the product of at most $200$ conjugates of an $\SL_n$ subgroup.
\end{thm*}
As indicated in the introduction, the Dennis---Vaserstein decomposition gives the following result.
\begin{lemma}
Assume $\sr(I)\leqslant n-1$. Then $\SL(n+1,R,I)$ is a product of at most $5$ subgroups isomorphic to $\SL(n,R,I)$.
\end{lemma}
\begin{proof}
By the Dennis---Vaserstein decomposition one can present $\SL(n+1,R,I)$ as a product
\[ \SL(n+1,R,I) =  \Par_1\cdot X_{n1}\cdot\Par_n=S_1\U_1\cdot X_{n1}\cdot\U_n S_n, \]
where $S_1$ and $S_n$ are two obvious embeddings of $\SL(n,R,I)$ in $\SL(n+1,R,I)$, avoiding respectively the first and the last row and column. Now $\U_1=(\U_1\cap S_n)\cdot X_{1n}$ and $U_n= X_{1n}\cdot(U_n\cap S_1)$, while $X_{1n}X_{n1}X_{1n}\in S_1^{w_{12}(1)}$.
\end{proof}
We will now elaborate on the case $\Phi=\rD_\ell$ to show that the Dennis---Vaserstein decomposition is suitable for handling other Chevalley groups, albeit with a stronger assumption on the base ring. The following lemma is an analogue of Proposition~1 of~\cite{NikProdDecomp}.
\begin{lemma}\label{lemma:nikolov-type-dl}
Let $\Phi=\rD_\ell$. There exist an element $y\in\G(\Phi,R)$ and an element $w\in\widetilde{W}(\Phi)$ such that $\U(\Sigma_\ell)\subset[\U(\Delta_\ell^-),y]\cdot{}^w\!\U(\Delta_\ell^+)$.
\end{lemma}
\begin{proof}
Set $\beta_i = \alpha_{2i-1} + 2\alpha_{2i}+ \ldots + 2\alpha_{\ell-2} + \alpha_{\ell-1} + \alpha_\ell$, for $i=1,\ldots,\lfloor\ell/2\rfloor-1$. Set $\beta_{\lfloor\ell/2\rfloor}=\alpha_{\ell-2}+\alpha_{\ell-1}+\alpha_\ell$ if $\ell$ is odd, and $\beta_{\lfloor\ell/2\rfloor}=\alpha_\ell$ if $\ell$ is even.  Denote $B=\{\beta_i\}$, then decompose $\U(\Sigma_\ell)=\U(\Sigma_\ell\setminus B)\cdot\U(B)$. In fact, $\U(\Sigma_\ell)$ is abelian.

Set $y=\prod_{\beta\in B}x_\beta(1)$. We will now show that $\U(\Sigma_\ell\setminus B)\subset[\U(\Delta_\ell^-),y]\cdot\U(B)$.

Note that \eqref{rel:CCF} implies
\[ \bigl[\U(\Delta_2^-),x_{\beta_1}(1)\bigr]=1,\quad \Bigl[\U(\Sigma_2^-\cap\Delta_\ell),\prod_{i\neq1}x_{\beta_i}(1)\Bigr]=1. \]
The first follows from the fact that the only root $\alpha$ with $m_2(\alpha)=2$ is $\beta_1$. The second holds since a root $\alpha\in(\Sigma_2^-\cap\Delta_\ell)+(B\setminus\{\beta_1\})$ must simultaneously satisfy $m_2(\alpha)=-1$ and $m_\ell(\alpha)=1$, which is impossible.

Since $B\setminus\{\beta_1\}\subset\Sigma_\ell\cap\Delta_2$, this implies
\[ \Bigl[ \U(\Sigma_2^-\cap\Delta_\ell)\cdot\U(\Delta_{2,\ell}^-), x_{\beta_1}(1)\cdot\prod_{i\neq1}x_{\beta_i}(1) \Bigr] \equiv \bigl[ \U(\Sigma_2^-\cap\Delta_\ell), x_{\beta_1}(1) \bigr] \bmod \U(\Sigma_\ell\cap\Delta_2). \]
Take an element $u\in\U(\Sigma_2^-\cap\Delta_\ell)$ and decompose it as $u=vw$, $v\in\U(\Sigma_1^-\cap\Sigma_2^-\cap\Delta_\ell)$, $w\in\U(\Sigma_2^-\cap\Delta_{1,\ell})$, then, using the identity
\begin{equation}\label{eq:comm-ab-c}
[ab,c]={}^a[b,c]\cdot[a,c],
\end{equation}
rewrite
\[ [vw,x_{\beta_1}(1)] = {}^v[w,x_{\beta_1}(1)]\cdot[v,x_{\beta_1}(1)].  \]
Since $\U(\Sigma_1^-\cap\Sigma_2^-\cap\Delta_\ell)$ and $\U(\Sigma_2^-\cap\Delta_{1,\ell})$ are abelian, it is easy to see that
\[ [v,x_{\beta_1}(1)]\in\U(\Sigma_2\cap\Sigma_\ell\cap\Delta_1), \quad [w,x_{\beta_1}(1)]\in\U((\Sigma_1\cap\Sigma_\ell)\setminus\{\beta_1\}). \]
Varying $v$ and $w$ one can obtain any element of $\U(\Sigma_2\cap\Sigma_\ell\cap\Delta_1)$ and $\U((\Sigma_1\cap\Sigma_\ell)\setminus\{\beta_1\})$ respectively. This is easy to see by using \eqref{eq:comm-ab-c} repeatedly and the fact that $\U(\Sigma_2^-\cap\Delta_\ell)$ commutes with $\U(\Sigma_\ell)$.
Now by the Levi decomposition
\[ {}^v\!\U((\Sigma_1\cap\Sigma_\ell)\setminus\{\beta_1\}) \equiv \U((\Sigma_1\cap\Sigma_\ell)\setminus\{\beta_1\}) \bmod \U(\Sigma_\ell\cap\Delta_2). \]
Thus we have shown that
\[ [\U(\Sigma_2^-)\cdot\U(\Delta_{2,\ell}),y] \equiv \U((\Sigma_{1,2}\cap\Sigma_\ell)\setminus\{\beta_1\}) \bmod \U(\Sigma_\ell\cap\Delta_2), \]
and this reduces the problem to the $\rD_{\ell-2}$-subsystem $\Delta_{1,2}$. So we can carry the induction on the rank of the subsystem, and construct for any given $a\in\U(\Sigma_\ell\setminus B)$ an element $b\in\U(\Delta_\ell^-)$ such that $a\in[b,y]\cdot\prod_{\beta\in B}X_\beta\subset[\U(\Delta_\ell^-),y]\cdot\U(B)$.

It remains to note that the roots $\pm\beta_i$ lie inside a maximal root subsystem of type $\rA_1+\ldots+\rA_1+\rD_3$. All such subsystems are conjugate under the action of the Weyl group, so there exist an element $w\in W(\Phi)$ such that $w(\beta_1)=\alpha_1$, $w(\beta_2)=\alpha_3$, $w(\beta_3)=\alpha_5$ and so on, so that all of $w(\beta_i)$ lie in $\Delta_\ell^+$.
\end{proof}
\begin{proof}[Proof of Theorem~\ref{thm:spin-sln-prod}]
By Theorem~\ref{thm:DennisVaserstein} one has a decomposition
\[ \Spin(2\ell,R,I) = \Par_{\ell-1}\cdot\U_{\ell-1,\ell}^-\cdot\Par_{\ell} = \G(\Delta_{\ell-1})\cdot\U(\Sigma_{\ell-1})\cdot\U(\Sigma_{\ell-1,\ell}^-)\cdot\U(\Sigma_{\ell})\cdot\G(\Delta_\ell). \]
By Lemma~\ref{lemma:nikolov-type-dl} there exist elements $y_1,y_2,y_3\in\G(\rD_\ell,R)$ and $w_1,w_2,w_3\in\widetilde{W}(\rD_\ell)$ such that
\begin{align*}
& \U(\Sigma_{\ell-1}) \subset \U(\Delta_{\ell-1}^-) \cdot {}^{y_1}\!\U(\Delta_{\ell-1}^-) \cdot {}^{w_1}\!\U(\Delta_{\ell-1}^+), \\
& \U(\Sigma_\ell) \subset {}^{w_2}\!\U(\Delta_\ell^+) \cdot {}^{y_2}\!\U(\Delta_\ell^-) \cdot \U(\Delta_\ell^-), \\
& \U(\Sigma_{\ell-1,\ell}^-) \subset \U(\Delta_\ell^+) \cdot {}^{y_3}\!\U(\Delta_\ell^+) \cdot {}^{w_3}\!\U(\Delta_\ell^-).
\end{align*}
Thus $\G(\rD_\ell,R,I)$ is a product of $9$ subgroups isomorphic to $\G(\rA_{\ell-1},R,I)\cong\SL(\ell,R,I)$.
\end{proof}