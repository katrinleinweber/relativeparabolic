%The main result of \cite{NikProdDecomp} is the following
%\begin{thm*}
%Let $G$ be a classical (possibly twisted) Chevalley group of rank $n$ over a finite field. Then $G$ equals the product of at most $200$ conjugates of an $\SL_n$ subgroup.
%\end{thm*}
As indicated in the introduction, the Dennis---Vaserstein decomposition gives the following result.
\begin{lemma}
Assume $\sr(I)\leqslant n-1$. Then $\SL(n+1,R,I)$ is a product of at most $5$ subgroups isomorphic to $\SL(n,R,I)$.
\end{lemma}
\begin{proof}
By the Dennis---Vaserstein decomposition one can present $\SL(n+1,R,I)$ as a product
\[ \SL(n+1,R,I) =  \Par_1\cdot X_{n1}\cdot\Par_n=S_1\U_1\cdot X_{n1}\cdot\U_n S_n, \]
where $S_1$ and $S_n$ are two obvious embeddings of $\SL(n,R,I)$ in $\SL(n+1,R,I)$, avoiding respectively the first and the last row and column. Now $\U_1=(\U_1\cap S_n)\cdot X_{1n}$ and $U_n= X_{1n}\cdot(U_n\cap S_1)$, while $X_{1n}X_{n1}X_{1n}\in S_1^{w_{12}(1)}$.
\end{proof}
We will now elaborate on the case $\Phi=\rD_\ell$ to show that the Dennis---Vaserstein decomposition is suitable for handling other Chevalley groups, albeit with a stronger assumption on the base ring. 

Consider the decreasing chain $\Phi_k$, $k=1,\ldots, \lfloor \ell/2 \rfloor$ of root subsystems of $\Phi=\rD_\ell$ defined as follows.
If $2k \neq \ell$ let $\Phi_k$ be the subsystem of $\Phi$ spanned by simple roots $\alpha_{2k-1}, \ldots, \alpha_\ell$.
Clearly, such $\Phi_k$ has type $\rD_{(\ell-2k+2)}$. 
In the remaining case $2k = \ell$ set $\Phi_k = \langle \alpha_\ell \rangle \cong \rA_1$.
Now let $\beta_k$ be the maximal root of $\Phi_k$, i.\,e. $\beta_k = \alpha_\mathrm{max}(\Phi_k)$, $k=1,\ldots, \lfloor \ell/2 \rfloor$.
Denote by $B$ the set of all $\beta_k$. From the definition it is clear that elements of $B$ are mutually perpendicular to each other.
The roots $\beta_k$ can also be defined by explicit formulae:
\begin{align*}
 \beta_k =  \alpha_{2k-1} + 2\alpha_{2k}+ \ldots + 2\alpha_{\ell-2} + \alpha_{\ell-1} + \alpha_\ell, & \text{ for } k=1,\ldots,\lfloor\ell/2\rfloor-1, \\
 \beta_{\lfloor\ell/2\rfloor} = \alpha_{\ell-2}+\alpha_{\ell-1}+\alpha_\ell, & \text{ if $\ell$ is odd,} \\
 \beta_{\lfloor\ell/2\rfloor} = \alpha_\ell, & \text{ if $\ell$ is even.}
\end{align*}

\begin{lemma}\label{lemma:nikolov-weyl} There exists an element $w\in W(\rD_\ell)$ such that $w(B) \subseteq \Delta_\ell^+$. \end{lemma}
\begin{proof}
\textsc{Case $\ell=4$.} Set $w = \sigma_{\alpha\ssub{1} + \alpha\ssub{2}} \circ \sigma_{\alpha\ssub{2} + \alpha\ssub{4}}$.
Straightforward computation shows that 
$$w(\beta_1) = w(\alpha_\mathrm{max}) = \sigma_{\alpha\ssub{1} + \alpha\ssub{2}}(\alpha_1 + \alpha_2 + \alpha_3) = \alpha_3,\quad w(\beta_2) = w(\alpha_4) = \sigma_{\alpha\ssub{1} + \alpha\ssub{2}}(- \alpha_2) = \alpha_1 $$
which implies the assertion of the lemma.

\textsc{Case $\ell \geq 5$.}
Recall from \cite[Table~9]{Dy72} that for odd (resp. even) $\ell$ all maximal subsystems of type $\rA_1+\ldots+\rA_1+\rD_3$
(resp. $\rA_1+\ldots+\rA_1+\rD_4$) are conjugate under the action of $W(\Phi)$. Consequently, we can find $w\in W(\Phi)$
such that $w(\beta_k) = \alpha_{2k-1}$ for $k < \lfloor\ell/2\rfloor$ (resp. $k < \lfloor\ell/2\rfloor-1$). 
Now using transitivity of the action of $W(\rD_3)$ on roots (resp. by the same argument as in the case $\ell=4$) we can move the remaining root $\beta_{\lfloor\ell/2\rfloor}$ 
(resp. the remaining $2$ roots $\beta_{\lfloor\ell/2\rfloor-1}$, $\beta_{\lfloor\ell/2\rfloor}$) to
$\alpha_{\ell-1}$ (resp. to $\alpha_{\ell-3}$, $\alpha_{\ell-1}$) while fixing all the other $\beta_k$. \end{proof}

The following lemma is an analogue of Proposition~1 of~\cite{NikProdDecomp}.
\begin{lemma}\label{lemma:nikolov-type-dl}
Let $\Phi=\rD_\ell$, $\ell\geq 4$.
There exist an element $y\in\E(\Phi,R)$ and an element $w\in\widetilde{W}(\Phi)$ such that $\U(\Sigma_\ell, I)\subset[\U(\Delta_\ell^-, I),y]\cdot{}^w\!\U(\Delta_\ell^+, I)$.
\end{lemma}
\begin{proof}
Since $\U(\Sigma_\ell, I)$ is abelian we can decompose it as $\U(\Sigma_\ell, I)=\U(\Sigma_\ell\setminus B, I) \cdot \U(B, I)$. 
Set $y=\prod_{\beta\in B}x_\beta(1)$. We will now show that $\U(\Sigma_\ell\setminus B, I)\subset[\U(\Delta_\ell^-, I),y]\cdot\U(B, I)$.

Notice that $\beta_1$ is the only root of $\Phi$ satisfying $m_2(\beta_1)=2$ therefore the commutator formula~\eqref{rel:CCF} implies
\[ \bigl[\U(\Delta_2^-, I),x_{\beta_1}(1)\bigr]=1. \]
Since the set $A=(\Sigma_2^-\cap\Delta_\ell)+(B\setminus\{\beta_1\})$ is disjoint from $\Phi$ 
(for $\alpha\in A$ one has $m_2(\alpha)=-1$ and $m_\ell(\alpha)=1$)
we also get from~~\eqref{rel:CCF} that
\[\Bigl[\U(\Sigma_2^-\cap\Delta_\ell, I),\prod_{i\neq1}x_{\beta_i}(1)\Bigr]=1. \]
Since $B\setminus\{\beta_1\}\subset\Sigma_\ell\cap\Delta_2$, the above identities imply
\[ \Bigl[ \U(\Sigma_2^-\cap\Delta_\ell, I)\cdot\U(\Delta_{2,\ell}^-, I), x_{\beta_1}(1)\cdot\prod_{i\neq1}x_{\beta_i}(1) \Bigr] \equiv \bigl[ \U(\Sigma_2^-\cap\Delta_\ell, I), x_{\beta_1}(1) \bigr] \bmod \U(\Sigma_\ell\cap\Delta_2, I). \]
Take an element $u\in\U(\Sigma_2^-\cap\Delta_\ell, I)$ and decompose it as $u=vw$, $v\in\U(\Sigma_1^-\cap\Sigma_2^-\cap\Delta_\ell, I)$, $w\in\U(\Sigma_2^-\cap\Delta_{1,\ell}, I)$.
Using the identity
\begin{equation}\label{eq:comm-ab-c}
[ab,c]={}^a[b,c]\cdot[a,c],
\end{equation}
we can rewrite
\[ [vw,x_{\beta_1}(1)] = {}^v[w,x_{\beta_1}(1)]\cdot[v,x_{\beta_1}(1)].  \]
Since $\U(\Sigma_1^-\cap\Sigma_2^-\cap\Delta_\ell, I)$ and $\U(\Sigma_2^-\cap\Delta_{1,\ell}, I)$ are abelian, it is easy to see that
\[ [v,x_{\beta_1}(1)]\in\U(\Sigma_2\cap\Sigma_\ell\cap\Delta_1, I), \quad [w,x_{\beta_1}(1)]\in\U((\Sigma_1\cap\Sigma_\ell)\setminus\{\beta_1\}, I). \]
%TODO: <add text from thesis>
Varying $v$ and $w$ one can obtain any element of $\U(\Sigma_2\cap\Sigma_\ell\cap\Delta_1)$ and $\U((\Sigma_1\cap\Sigma_\ell)\setminus\{\beta_1\})$ respectively. This is easy to see by using \eqref{eq:comm-ab-c} repeatedly and the fact that $\U(\Sigma_2^-\cap\Delta_\ell)$ commutes with $\U(\Sigma_\ell)$. Indeed, write $v=x_\gamma(\xi_\gamma)\cdot v'$, $\gamma=-\alpha_1-\alpha_2$, $v' \in \U(\Sigma_1^- \cap \Sigma_2^- \cap \Delta_\ell \setminus \{\gamma\})$. Then
\begin{align*}
[v, x_{\beta_1}(1)] & = [x_\gamma(\xi_\gamma)\cdot v', x_{\beta_1}(1)] = \\
& = {}^{x_\gamma(\xi_\alpha)}[v', x_{\beta_1(1)}] \cdot [x_\gamma(\xi_\gamma), x_{\beta_1}(1)] = \\
& = [v', x_{\beta_1(1)}]\cdot x_{\beta_1-\alpha_1-\alpha_2}(\xi_\gamma) = \\
& = \ldots = \prod_{\mathclap{\alpha\in\Sigma_1^- \cap \Sigma_2^- \cap \Delta_\ell}} x_{\beta_1-\alpha}(\xi_\alpha).
\end{align*}
It remains to note that $\Sigma_2 \cap \Sigma_\ell \cap \Delta_1 = \beta_1 - \Sigma_1^- \cap \Sigma_2^- \cap \Delta_\ell$. The same argument works for $[w, x_{\beta_1}(1)]$.

Now by the Levi decomposition
\[ {}^v\!\U((\Sigma_1\cap\Sigma_\ell)\setminus\{\beta_1\}) \equiv \U((\Sigma_1\cap\Sigma_\ell)\setminus\{\beta_1\}) \bmod \U(\Sigma_\ell\cap\Delta_2). \]
Thus we have shown that
\[ [\U(\Sigma_2^-)\cdot\U(\Delta_{2,\ell}),y] \equiv \U((\Sigma_{1,2}\cap\Sigma_\ell)\setminus\{\beta_1\}) \bmod \U(\Sigma_\ell\cap\Delta_2), \]
and this reduces the problem to the $\rD_{\ell-2}$-subsystem $\Delta_{1,2}$. So we can carry the induction on the rank of the subsystem, and construct for any given $a\in\U(\Sigma_\ell\setminus B)$ an element $b\in\U(\Delta_\ell^-)$ such that $a\in[b,y]\cdot\prod_{\beta\in B}X_\beta\subset[\U(\Delta_\ell^-),y]\cdot\U(B)$.
%TODO: </add text from thesis>

It remains to notice that the assertion of the lemma follows from~\cref{lemma:nikolov-weyl}.
%It remains to note that the roots $\pm\beta_i$ lie inside a maximal root subsystem of type $\rA_1+\ldots+\rA_1+\rD_3$.
%All such subsystems are conjugate under the action of the Weyl group, so there exist an element $w\in W(\Phi)$ such that
%$w(\beta_1)=\alpha_1$, $w(\beta_2)=\alpha_3$, $w(\beta_3)=\alpha_5$ and so on, so that all of 
%$w(\beta_i)$ lie in $\Delta_\ell^+$.
\end{proof}
\begin{proof}[Proof of \cref{thm:spin-sln-prod}]
By \cref{thm:DennisVaserstein} one has a decomposition
\[ \Spin(2\ell,R,I) = \Par_{\ell-1}\cdot\U_{\ell-1,\ell}^-\cdot\Par_{\ell} = \G(\Delta_{\ell-1})\cdot\U(\Sigma_{\ell-1})\cdot\U(\Sigma_{\ell-1,\ell}^-)\cdot\U(\Sigma_{\ell})\cdot\G(\Delta_\ell). \]
By \cref{lemma:nikolov-type-dl} there exist elements $y_1,y_2,y_3\in\G(\rD_\ell,R)$ and $w_1,w_2,w_3\in\widetilde{W}(\rD_\ell)$ such that
\begin{align*}
& \U(\Sigma_{\ell-1}) \subset \U(\Delta_{\ell-1}^-) \cdot {}^{y_1}\!\U(\Delta_{\ell-1}^-) \cdot {}^{w_1}\!\U(\Delta_{\ell-1}^+), \\
& \U(\Sigma_\ell) \subset {}^{w_2}\!\U(\Delta_\ell^+) \cdot {}^{y_2}\!\U(\Delta_\ell^-) \cdot \U(\Delta_\ell^-), \\
& \U(\Sigma_{\ell-1,\ell}^-) \subset \U(\Delta_\ell^+) \cdot {}^{y_3}\!\U(\Delta_\ell^+) \cdot {}^{w_3}\!\U(\Delta_\ell^-).
\end{align*}
Thus $\G(\rD_\ell,R,I)$ is a product of $9$ subgroups isomorphic to $\G(\rA_{\ell-1},R,I)\cong\SL(\ell,R,I)$.
\end{proof}