%The main result of \cite{NikProdDecomp} is the following
%\begin{thm*}
%Let $G$ be a classical (possibly twisted) Chevalley group of rank $n$ over a finite field. Then $G$ equals the product of at most $200$ conjugates of an $\SL_n$ subgroup.
%\end{thm*}
%As indicated in the introduction, the Dennis---Vaserstein decomposition gives the following result.
%\begin{lemma}
%Assume $\sr(I)\leqslant n-1$. Then $\SL(n+1, R, I)$ is a product of at most $5$ subgroups isomorphic to $\SL(n, R, I)$.
%\end{lemma}
%\begin{proof}
%By the Dennis---Vaserstein decomposition one can present $\SL(n+1, R, I)$ as a product
%\[ \SL(n+1, R, I) =  \Par_1\cdot X_{n1}\cdot\Par_n=S_1\U_1\cdot X_{n1}\cdot\U_n S_n, \]
%where $S_1$ and $S_n$ are two obvious embeddings of $\SL(n, R, I)$ in $\SL(n+1, R, I)$, avoiding respectively the first and the last row and column. Now $\U_1=(\U_1\cap S_n)\cdot X_{1n}$ and $U_n= X_{1n}\cdot(U_n\cap S_1)$, while $X_{1n}X_{n1}X_{1n}\in S_1^{w_{12}(1)}$.
%\end{proof}
%We will now elaborate on the case $\Phi=\rD_\ell$ to show that the Dennis---Vaserstein decomposition is suitable for handling other Chevalley groups, albeit with a stronger assumption on the base ring. 

Consider the decreasing chain $\Phi_k$, $k=1, \ldots, \lfloor \ell/2 \rfloor$ of root subsystems of $\Phi=\rD_\ell$ defined as follows.
If $2k \neq \ell$, let $\Phi_k$ be the subsystem of $\Phi$, spanned by the fundamental roots $\alpha_{2k-1}, \ldots, \alpha_\ell$.
Clearly, such $\Phi_k$ has type $\rD_{\ell-2k+2}$. 
In the remaining case $2k = \ell$ set $\Phi_k = \langle \alpha_\ell \rangle \cong \rA_1$.
Now let $\beta_k$ be the maximal root of $\Phi_k$, i.\,e. $\beta_k = \alpha_\mathrm{max}(\Phi_k)$, $k=1, \ldots, \lfloor \ell/2 \rfloor$.
Denote by $B$ the set of all $\beta_k$. From the definition it is clear that the elements of $B$ are mutually orthogonal to each other.
The roots $\beta_k$ can also be defined by explicit formulae:
\begin{align*}
 \beta_k =  \alpha_{2k-1} + 2\alpha_{2k}+ \ldots + 2\alpha_{\ell-2} + \alpha_{\ell-1} + \alpha_\ell, & \text{ for } k=1, \ldots, \lfloor\ell/2\rfloor-1, \\
 \beta_{\lfloor\ell/2\rfloor} = \alpha_{\ell-2}+\alpha_{\ell-1}+\alpha_\ell, & \text{ if $\ell$ is odd, } \\
 \beta_{\lfloor\ell/2\rfloor} = \alpha_\ell, & \text{ if $\ell$ is even.}
\end{align*}

\begin{lemma}\label{lemma:nikolov-weyl} There exists an element $w\in W(\rD_\ell)$ such that $w(B) \subseteq \Delta_\ell^+$. \end{lemma}
\begin{proof}
\textsc{Case $\ell=4$.} Set $w = \sigma_{\alpha\ssub{1} + \alpha\ssub{2}} \circ \sigma_{\alpha\ssub{2} + \alpha\ssub{4}}$.
Straightforward computation shows that 
\[ w(\beta_1) = w(\alpha_\mathrm{max}) = \sigma_{\alpha\ssub{1} + \alpha\ssub{2}}(\alpha_1 + \alpha_2 + \alpha_3) = \alpha_3, \quad w(\beta_2) = w(\alpha_4) = \sigma_{\alpha\ssub{1} + \alpha\ssub{2}}(- \alpha_2) = \alpha_1, \]
which implies the assertion of the lemma.

\textsc{Case $\ell \geq 5$.}
Recall from \cite[Table~9]{Dy72} that for odd (resp. even) $\ell$ all maximal subsystems of type $\rA_1+\ldots+\rA_1+\rD_3$
(resp. $\rA_1+\ldots+\rA_1+\rD_4$) are conjugate under the action of $W(\Phi)$. Consequently, we can find $w\in W(\Phi)$
such that $w(\beta_k) = \alpha_{2k-1}$ for $k < \lfloor\ell/2\rfloor$ (resp. $k < \lfloor\ell/2\rfloor-1$). 
Now using transitivity of the action of $W(\rD_3)$ on the roots (resp. by the same argument as in the case $\ell=4$) we can move the remaining root $\beta_{\lfloor\ell/2\rfloor}$ 
(resp. the remaining $2$ roots $\beta_{\lfloor\ell/2\rfloor-1}$, $\beta_{\lfloor\ell/2\rfloor}$) to
$\alpha_{\ell-1}$ (resp. to $\alpha_{\ell-3}$, $\alpha_{\ell-1}$) while fixing all the other $\beta_k$. \end{proof}

The following lemma is an analogue of Proposition~1 of~\cite{Nik07}.
\begin{lemma}\label{lemma:nikolov-type-dl}
Let $\Phi=\rD_\ell$, $\ell\geq 2$ and let $I$ be an ideal of a  commutative ring $R$.
There exist an element $y\in\E(\Phi, R)$ and an element $w\in\widetilde{W}(\Phi)$ such that $\U(\Sigma_\ell^+, I)\subset[\U(\Delta_\ell^-, I), y]\cdot{}^w\!\U(\Delta_\ell^+, I)$.
\end{lemma}
\begin{proof}
Since $\U(\Sigma_\ell^+, I)$ is abelian, we can decompose it as $\U(\Sigma_\ell^+, I)=\U(\Sigma_\ell^+\setminus B, I) \cdot \U(B, I)$. 
Set $y=\prod_{\beta\in B}x_\beta(1)$. 
We will now show by induction on $\ell$ that 
\begin{equation}\label{eq:ind-stat} \U(\Sigma_\ell^+\setminus B, I)\subset[\U(\Delta_\ell^-, I), y]\cdot\U(B, I). \end{equation}
The induction base in the cases $\ell=2, 3$ is trivial.

Notice that $\beta_1$ is the only root of $\Phi$ satisfying $m_2(\beta_1)=2$, therefore the commutator formula~\eqref{rel:CCF} implies
\[ \bigl[\U(\Delta_2^-, I), x_{\beta_1}(1)\bigr]=1. \]
There is no root of the form $\gamma=\alpha+\beta$ with $\alpha\in\Sigma_2^-\cap\Delta_\ell$ and $\beta\in B\setminus\{\beta_1\}$, because such a root $\gamma$ must satisfy $m_2(\gamma)=-1$ and $m_\ell(\gamma)=1$. Thus~\eqref{rel:CCF} gives
\[\Bigl[\U(\Sigma_2^-\cap\Delta_\ell, I), \prod_{i\neq1}x_{\beta_i}(1)\Bigr]=1. \]
Since $B\setminus\{\beta_1\}\subset\Sigma_\ell^+\cap\Delta_2$, the above two identities imply
\[ \Bigl[ \U(\Sigma_2^-\cap\Delta_\ell, I)\cdot\U(\Delta_{2, \ell}^-, I), x_{\beta_1}(1)\cdot\prod_{i\neq1}x_{\beta_i}(1) \Bigr] \equiv \bigl[ \U(\Sigma_2^-\cap\Delta_\ell, I), x_{\beta_1}(1) \bigr] \bmod \U(\Sigma_\ell^+\cap\Delta_2, I). \]
Take an element $u\in\U(\Sigma_2^-\cap\Delta_\ell, I)$ and decompose it as $u=vw$, where $v\in\U(\Sigma_1^-\cap\Sigma_2^-\cap\Delta_\ell, I)$ and $w\in\U(\Sigma_2^-\cap\Delta_{1, \ell}, I)$.
Using the identity
\begin{equation}\label{eq:comm-ab-c}
[ab, c]={}^a[b, c]\cdot[a, c], 
\end{equation}
we can rewrite
\[ [vw, x_{\beta_1}(1)] = {}^v[w, x_{\beta_1}(1)]\cdot[v, x_{\beta_1}(1)].  \]
Since $\U(\Sigma_1^-\cap\Sigma_2^-\cap\Delta_\ell, I)$ and $\U(\Sigma_2^-\cap\Delta_{1, \ell}, I)$ are abelian, it is easy to see that
\[ [v, x_{\beta_1}(1)]\in\U(\Sigma_2^+\cap\Sigma_\ell^+\cap\Delta_1, I), \quad [w, x_{\beta_1}(1)]\in\U((\Sigma_1^+\cap\Sigma_\ell^+)\setminus\{\beta_1\}, I). \]
Every element of $\U(\Sigma_2^+\cap\Sigma_\ell^+\cap\Delta_1, I)$ (resp. $\U(\Sigma_1^+\cap\Sigma_\ell^+\setminus\{\beta_1\}, I)$) can be expressed as such a commutator for a suitable choice of $v$ (resp. $w$).
Indeed, set $v=x_\gamma(\xi_\gamma)\cdot v'$, $\gamma=-\alpha_1-\alpha_2$, $v' \in \U(\Sigma_1^- \cap \Sigma_2^- \cap \Delta_\ell \setminus \{\gamma\})$.
Using relation \eqref{eq:comm-ab-c} and the fact that $\U(\Sigma_2^-\cap\Delta_\ell)$ commutes with $\U(\Sigma_\ell^+)$ we get that:
\begin{multline*}
[v, x_{\beta_1}(1)] = [x_\gamma(\xi_\gamma)\cdot v', x_{\beta_1}(1)] = {}^{x_\gamma(\xi_\alpha)}[v', x_{\beta_1}(1)] \cdot [x_\gamma(\xi_\gamma), x_{\beta_1}(1)] = \\
= [v', x_{\beta_1}(1)]\cdot x_{\beta_1-\alpha_1-\alpha_2}(\xi_\gamma) = \ldots = \prod_{\mathclap{\alpha\in\Sigma_1^- \cap \Sigma_2^- \cap \Delta_\ell}} x_{\beta_1+\alpha}(\xi_\alpha).
 \end{multline*}
It remains to note that $\Sigma_2^+ \cap \Sigma_\ell^+ \cap \Delta_1 = \beta_1 + \Sigma_1^- \cap \Sigma_2^- \cap \Delta_\ell$. The same argument works for $[w, x_{\beta_1}(1)]$.
Direct calculation using the commutator formula shows that
\[ {}^v\!\U(\Sigma_1^+\cap\Sigma_\ell^+\setminus\{\beta_1\}, I) \equiv \U(\Sigma_1^+\cap\Sigma_\ell^+\setminus\{\beta_1\}, I) \bmod \U(\Sigma_\ell^+\cap\Delta_2, I). \]
Summing up the above arguments, we get that
\[ [\U(\Sigma_2^-, I)\cdot\U(\Delta_{2, \ell}^-, I), y] \equiv \U((\Sigma_{1, 2}^+\cap\Sigma_\ell^+)\setminus\{\beta_1\}) \bmod \U(\Sigma_\ell^+\cap\Delta_2, I), \]
hence the inclusion~\eqref{eq:ind-stat} follows from the induction hypothesis (applied to $\Delta_{1, 2} \cong \rD_{\ell-2}$). 

Finally, we have found $a\in\U(\Sigma_\ell^+\setminus B, I)$ and $b\in\U(\Delta_\ell^-, I)$
such that $$a\in[b, y]\cdot\prod_{\beta\in B}X_\beta\subset[\U(\Delta_\ell^-, I), y]\cdot\U(B, I).$$
Now the assertion of the lemma follows from~\cref{lemma:nikolov-weyl}.
\end{proof}

\begin{proof}[Proof of \cref{thm:spin-sln-prod}]
Set $L = \E(\Delta_\ell, R, I) \leq \E(\rD_\ell, R, I)$ and denote by $\sigma$ the automorphism of $\G(\rD_\ell, R)$ induced by the sole diagram automorphism of the Dynkin diagram of $\rD_\ell$.
By \cref{thm:DennisVaserstein} one has
\begin{multline*}
\E(\rD_\ell, R, I) = \EP_{\ell}(R, I)\cdot\U(\Sigma_{\ell-1}^- \cap \Sigma_{\ell}^-, I)\cdot\EP_{\ell-1}(R, I) = \\
= L \cdot\U(\Sigma_\ell^+, I)\cdot\U(\Sigma_{\ell-1}^- \cap \Sigma_{\ell}^-, I)\cdot \big(L \cdot \U(\Sigma_\ell^+, I) \big)^\sigma.
\end{multline*}  
Now using \cref{lemma:nikolov-type-dl}, one can find $y_1, y_2\in\G(\rD_\ell, R)$ and $w_1, w_2\in\widetilde{W}(\rD_\ell)$ such that
\begin{align*} & L \cdot \U(\Sigma_\ell^+, I) \subset L \cdot \U(\Delta_{\ell-1}^-, I) \cdot {}^{y_1}\!\U(\Delta_{\ell-1}^-, I) \cdot {}^{w_1}\!\U(\Delta_{\ell-1}^+, I), \\
& \U(\Sigma_{\ell-1}^-\cap\Sigma_\ell^-, I) \subset \U(\Delta_\ell^+, I) \cdot {}^{y_2}\!\U(\Delta_\ell^+, I) \cdot {}^{w_2}\!\U(\Delta_\ell^-, I). \end{align*} 
Thus $\E(\rD_\ell, R, I)$ is a product of at most $9$ conjugates of $L \cong \E(\rA_{\ell-1}, R, I)$. \end{proof}