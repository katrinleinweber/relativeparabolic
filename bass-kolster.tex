The next theorem is a relative version of the so called Bass---Kolster decomposition (cf.~\cite[Theorem~2.1]{St78}).
\begin{thm}\label{thm:BassKolster}
Let $\Phi$ be a classical root system of rank $\ell\geqslant2$, let $R$ be an arbitrary commutative ring and $I$ be an ideal, satisfying one of the following assumptions:
\[\begin{array}{l@{\quad}l@{\quad}l@{\quad}c}
\Phi = \rA_\ell, \ \ell\geqslant 2, & \sr(I) \leqslant \ell; \\
\Phi = \rC_\ell, \ \ell\geqslant 2, & \sr(I) \leqslant 2\ell-1; \\
\Phi = \rB_\ell, \rD_\ell, \ \ell\geqslant 3, & \asr(I) \leqslant \ell-1.
\end{array}\]
Then the principal congruence subgroup $\G(\Phi, R, I)$ admits the following relative version of Bass---Kolster decomposition:
\[ \G(\Phi, R, I)=  \U(\Phi^+, I) \cdot \U(\Phi^-, I) \cdot Z \cdot \U(\Sigma_1^-\setminus\{-\alpha_\mathrm{max}\}, I) \cdot \U(\Sigma_1, I) \cdot \G(\Delta_1, R, I), \]
where $Z = Z_{\alpha_\mathrm{max}}(I)=\left\{z_{-\alpha_\mathrm{max}}(r, 1)\ \middle|\ r\in I \right\}$.
\end{thm}
\begin{proof}

Let $g$ be an element of $\G(\Phi, R, I)$. Set $v=g \cdot v^+\in\Ums(n, I)$. 
Notice that in each case it suffices to find $g' \in \U(\Phi^-, I) \cdot \U(\Phi^+, I) \cdot g$ such that 
\begin{equation} \label{eq1} (g'\cdot v^+)_{1} = 1 + s \text{ and } (g'\cdot v^+)_{\varpi\ssub{1}-\alpha\ssub{max}} = s\ \text{for some}\ s\in I. \end{equation}
Indeed, set $g'' = z_{-\alpha\ssub{max}}(-s, 1) \cdot g'$.
Obviously, one has $(g''\cdot v^+)_1 = 1$, $(g''\cdot v^+)_{\varpi\ssub{1}-\alpha\ssub{max}}=0$ and the conclusion of the theorem follows from \cref{lemma:Chevalley-Matsumoto}.

\textsc{Case $\Phi=\rA_\ell$, $n=\ell + 1$.}
Since $\sr(I)\leqslant\ell$, one can find $a_1, \ldots, a_\ell\in I$ such that $(v_1+a_1v_{\ell+1}, \ldots, v_\ell+a_\ell v_{\ell+1})^t=(v'_1, \ldots, v'_\ell)^t$ is $I$-unimodular.
Then there are $b_1, \ldots, b_\ell\in I$ such that $b_1v'_1+\ldots b_\ell v'_\ell=v'-1\in I$. 
Thus the vector \[ v'' = \prod_{i=1}^\ell x_{\ell+1, i}(b_i) \cdot \prod_{i=1}^\ell x_{i, \ell+1}(a_i) \cdot v \]
satisfies the equalities~\eqref{eq1}.

\textsc{Case $\Phi=\rC_\ell$, $n=2\ell$.}
Notice that the column $(v_1, \ldots, v_{-2}, v_{-1}^2)^t$ is also $I$-unimodular.
Applying condition $\sr(I)\leq 2\ell-1$ we find $c_1, c_2, \ldots, c_{-2} \in I \cdot v_{-1}$ such that upper $2\ell -1$ components of $v'=(v_1 + c_1 v_{-1}, \ldots, v_{-2} + c_{-2}v_{-1}, v_{-1})^t$ form an $I$-unimodular column.
By the choice of $c_i$ we can find suitable $d\in I$ such that $h_1 \cdot v = v'$ for
\[ h_1 = x_{1, -1}(c_1 + d) \cdot \prod_{i=2}^{-2} x_{i, -1}(c_i) \in \U(\Sigma_1^+, I). \]

We can find $f_1, f_2, \ldots, f_{-2} \in R$ such that $f_1v'_1+\sum_{i=2}^{-2} f_i v'_i = 1$.
Set $\xi = v''_1-v''_{-1}-1 \in I$, 
\[ h_2 = x_{-1, 1}\biggl(\xi f_1 + \sum_{i=2}^\ell v_1' \xi^2 f_i f_{-i}\biggr) \cdot \prod_{i=2}^{-2} x_{-1, i}(\xi f_i) \in \U(\Sigma_1^-, I). \]
Direct computation shows that $v'' = h_2 \cdot v'$ satisfies equalities~\eqref{eq1}.

\textsc{Case $\Phi=\rD_\ell$, $n= 2\ell$.} 
By \cref{item:asrUnipD} we can find $h_1\in \U(\Sigma^+_\ell, I)$ such that the upper half $v'_+$ of $v'=h_1 \cdot v$ is $I$-unimodular.
Since $\sr(I)\leq \ell-1$ we can find $c_1$, $c_3, \ldots c_\ell \in I$ such that $(v''_1, v''_3, \ldots, v''_\ell) \in \Ums(\ell-1, I)$, where
\[ v''=h_2 \cdot x_{1, 2}(c_1) \cdot v', \quad h_2=\prod_{i=3}^\ell x_{i, 2}(c_i). \]
We can find $f_1, f_3, \ldots, f_\ell \in R$ such that $f_1v''_1+\sum_{i=3}^\ell f_i v''_{i} = 1$.
As before, set
\[ \xi = v''_1-v''_{-2}-1 \in I, \quad h_3 = x_{-2, 1}(\xi f_1) \cdot \prod_{i=3}^\ell x_{-2, i}(\xi f_i), \quad v'''=h_3 \cdot v''. \]
Clearly, $t_{1, 2}(c_1) \cdot h_1 \in \U(\Phi^+, I)$, $ h_3 \cdot h_2 \in \U(\Phi^-, I)$ and $v'''$ satisfies~\eqref{eq1}.

\textsc{Case $\Phi=\rB_\ell$, $n=2\ell+1$.} Subdivide $v\in \Ums(2\ell+1, I)$ as $v=(v_+, v_0, v_-)\in R^\ell\times R\times R^\ell$.
Denote by $J\leq I$ the ideal spanned by components of $v_-$.
Since $\sr(I/J)\leq \ell$ we can find $c_1, \dots, c_\ell\in I$ such that for $v' = h \cdot v$, $h = \prod_{i=1}^\ell x_{i, 0}(c_i) \in \U(\Phi^+, I)$
one has $\bar{v'}_+=(\bar{v'_1}, \ldots, \bar{v'_\ell}) \in \Ums(\ell, I/J)$ and, therefore, $(v'_+, v'_-) \in \Ums(2\ell, I)$.
Now the proof can be finished by repeating the argument for the case $\Phi=\rD_\ell$ (applied to the subset of long roots of $\rB_\ell$).
\end{proof}

It is easy to see that the proof of the above theorem is effective and gives an estimate of the total number of elementary root unipotents involved in the decomposition.
\begin{cor}\label{cor:bass-kolster-count}
In the assumptions and notation of \cref{thm:BassKolster} every element of $\G(\Phi, R, I)$ 
can be factored into a product of one element of $\G(\Delta_1, R, I)$ one element of $Z$ and at most $4|\Sigma_1|-1$ elements of $\mathcal{X}$.
\end{cor}
\begin{proof}
The assertion can be obtained by a careful analysis of the proof of the previous theorem.
Cases $\Phi=\rA_\ell, \rC_\ell$ are immediate.
In the case $\Phi=\rD_\ell$ from the proof of \cref{thm:BassKolster} one obtains that
\begin{equation*} \G(\Phi, R, I) =  \U(\Sigma_\ell^+, I) \cdot X_{\alpha_1}(I) \cdot \U(\Sigma_2^-\cap\Delta_1, I) \cdot X_{-\alpha\ssub{\mathrm{max}}}(I) \cdot Z \cdot \U(\Sigma_1^-, I) \cdot \U(\Sigma_1^+, I) \cdot \G(\Delta_1, R, I). \end{equation*}
We can present an element $g$ of $\U(\Sigma_\ell^+, I)$ as a product of $g_1 \in \U(\Sigma_{\{1, 2\}^+} \cap \Sigma_\ell^+)$ and $g_2\in \U(\Delta_{\{1, 2\}}\cap \Sigma_\ell^+)$.
An examination of the extended Dynkin diagram of $\rD_\ell$ implies that $g_2$ either centralizes or normalizes all factors of the above decomposition (except the last one)
and therefore can be moved to the right until it is consumed by $\G(\Delta_1, R, I)$.
On the other hand, $g_1$ is a product of at most $2\ell-3$ elementary unipotents, while the width of $\U(\Sigma_1^\pm, I)$ and $\U(\Sigma_2^-\cap\Delta_1)$ in elementary unipotents does not exceed $2\ell-2$ and $2\ell-4$, respectively.
Summing up these upper bounds we obtain
$$(2\ell-3) + 1 + (2\ell - 4) + 1 + 2\cdot (2\ell - 2) = 8\ell - 9 = 4|\Sigma_1| - 1.$$

The estimate in the case $\Phi=\rB_\ell$ can be obtained in a similar way. \end{proof}
\begin{cor}\label{cor:bass-kolster-iterated}
Assume that $\Phi$ and $I$ satisfy one of the following assumptions
\[\begin{array}{l@{\quad}l@{\quad}l}
\Phi=\rA_\ell, & \sr(I)\leqslant 2, & N'=3\left|\Phi^+\right|+2\ell - 5; \\
\Phi=\rC_\ell, & \sr(I)\leqslant 3, & N'=3\left|\Phi^+\right|+3\ell - 6; \\
\Phi=\rB_\ell, \rD_\ell, & \asr(I)\leqslant 2, & N'=4\left|\Phi^+\right| - 4.
\end{array}\]

Then every element of $\G(\Phi, R, I)$ can be decomposed into a product of one element of $\G(\langle\pm\alpha_\ell\rangle, R, I) \cong \SL(2, R, I)$ and at most $N'$ elements of $\mathcal{Z}(\Sigma_\ell)$:
\end{cor}
\begin{proof}
The assertion can be obtained by iteratively applying (for a total of $\ell-1$ times) the decomposition of \cref{thm:BassKolster}.
The improved estimate for $\Phi=\rA_\ell$ (resp. $\rC_\ell$) follows from the fact that it suffices to make only two (resp. three) additions to shorten the unimodular column in the first step of the proof of \cref{thm:BassKolster}.
\end{proof}
\begin{figure}[hb]\label{fig:bass-kolster}
\[\begin{tikzcd}[row sep=tiny]
\rD_4 \arrow[out=0, in=120]{rd}{\asr\leqslant3} & & & \\
\rA_4 \arrow{r}{\sr\leqslant4} & \rA_3 \arrow{r}{\sr\leqslant3} & \rA_2 \arrow{r}{\sr\leqslant2} & \rA_1 \\
\rC_4 \arrow{r}{\sr\leqslant7} & \rC_3 \arrow{r}{\sr\leqslant5} & \rC_2 \arrow[out=0, in=-120, swap]{ru}{\sr\leqslant3} \\
\rB_4 \arrow{r}{\asr\leqslant3} & \rB_3 \arrow[out=0, in=-120, swap]{ru}{\asr\leqslant2} & &
\end{tikzcd}\] \caption{Reductions used in the proof of \cref{cor:bass-kolster-iterated} and \cref{thm:SL2width}} \end{figure}
\begin{proof}[Proof of \cref{thm:SL2width}]
 As in the proof of the above corollary one has to iteratively apply \cref{thm:BassKolster}.
 To reduce the number of $\SL_2$-factors involved in the decompositoin one has to group into a single $\SL_2$-factor a pair of opposite root subgroups $X_{\alpha}(I)$, $X_{-\alpha}(I)$ (or $Z_{\pm\alpha}(I)$) appearing on each
 of the $3$ junctions between the positive and negative unipotent subgroups in the Bass---Kolster decomposition.
 Since a total of $\ell-1$ reductions are used, we get the estimate $N \leq N' - 3(\ell - 1) + 1$ and the assertion of \cref{thm:SL2width} follows.
\end{proof}
