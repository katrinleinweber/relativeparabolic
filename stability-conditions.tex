As we will be mainly concerned with applications to stability of $K_1$ of Chevalley groups, our treatment of stability conditions is restricted to the case of commutative $R$
notwithstanding with the fact that the majority of constructions related to stable range conditions can be defined in a more general non-commutative setting (see e.g.~\cite{Ba64, Va69, Va71}).

\subsection{Stable rank}
Recall the definition of \emph{stable rank} introduced by H.~Bass and L.~Vaserstein (see~\cite{Ba64, Va69}).
\begin{dfn} A row $a\in{}^n\!R$ is called \emph{$I$-unimodular} if it is congruent to $e_1$ modulo $I$ and its components $a_1,a_2,\ldots,a_n$ generate $R$ as an ideal. \end{dfn}
A column $b \in R^n$ will be called $I$-unimodular if its transpose $b^t$ is an $I$-unimodular row.
We denote the set of all $I$-unimodular rows (resp. columns) by $\Umd(n,R,I)$ (resp. $\Ums(n,R,I)$).
When $R=I$ we refer to $R$-unimodular rows and columns as simply unimodular.

It is not hard to show that for a $I$-unimodular row $a$ there exists an $I$-unimodular column $b$ such that $ab=1$ (see~\cite[\S~2]{Va69}).

An $I$-unimodular row $a=(a_1,\ldots a_{n+1})$ is called \emph{stable} if one can choose $b_1,\ldots,b_n\in I$ such that the row $(a_1+a_{n+1}b_1,\ldots,a_n+a_{n+1}b_n)$ is also $I$-unimodular. 
We say that $I$ satisfies stable range condition $\SR_n(I)$ if any $I$-unimodular row of length $n+1$ is stable.
It is easy to see that $\SR_n(I)$ implies $\SR_m(I)$ for any $m\geqslant n$. It is also clear that validity of $\SR_n(I)$ does not depend on the ambient ring $R$.

By definition, the {\it stable rank} of $I$ is the smallest natural $n$ such that $\SR_n(I)$ holds.
We denote the stable rank of $I$ by $\sr(I)$, we set $\sr(I)=\infty$ if no such $n$ exists.

The following proposition summarizes basic properties of stable ranks.
\begin{prop} \label{prop:sr_properties}
Let $R$ be arbitrary commutative unital ring and let $I\trianglelefteq R$ be an ideal.
\begin{enumerate}
\item For any ideal $J\subseteq I$ one has $\sr(R,I)\leqslant\sr(R)$, $\sr(R/J,I/J)\leqslant\sr(R,I)$.
\item One has $\sr\left(\prod\limits_{i=1}^n R_i\right)=\max\limits_{i=1}^n\left(\sr(R_i)\right).$
\item Let $J$ denote the Jacobson radical of $R$, then one has $\sr(R)=\sr(R/J)$ and $\sr(R, J) = 1$.
\end{enumerate}
\end{prop}
\begin{proof} See~\cite{Va71}, \cite[Theorem~2.3]{Va69}. 
\textbf{TODO: Add reference for the latter fact}.
%TODO:
\end{proof}
\begin{example}
Since the stable rank of a field equals $1$, one can conclude from the previous proposition that $\sr(R)=1$ for any semilocal ring $R$.
Other examples of rings of stable rank $1$ are provided by boolean rings, the ring of algebraic integers, the ring of entire functions, the disk-algebra.
\end{example}

\subsection{Absolute stable rank}
For a row $a=(a_1,\ldots,a_n)\in{}^n\!R$ denote by $\mathfrak{J}(a)$ the intersection of all maximal ideals of $R$ containing $a_1,\ldots,a_n$.
It is easy to see that a row $a\in R^n$ is unimodular if and only if $\mathfrak{J}(a)=R$. 
Clearly, for any $g\in\GL(n,R)$ one has $\mathfrak{J}(a\cdot g)=\mathfrak{J}(a)$.
\begin{dfn}
We say that a pair $(R,I)$ satisfies $\ASR_n(R,I)$ if it satisfies $\SR_n(R,I)$ and, moreover, 
for any row $a=(a_1,\ldots,a_{n+1})\in I^n$ there exist $b_1,\ldots,b_n\in I$ such that
\[\mathfrak{J}(a_1,\ldots,a_{n+1})=\mathfrak{J}(a_1+b_1a_{n+1},\ldots,a_n+b_na_{n+1}).\]
It is easy to see that $\ASR_m(R,I)$ implies $\ASR_n(R,I)$ for any $n\geqslant m$. 
By definition, the \emph{absolute relative stable rank} $\asr(R,I)$ is the smallest natural $n$ such that $\ASR_n(R, I)$ holds. 
Similarly, we set $\asr(R,I)=\infty$ if no such $n$ exists.
When $I=R$ we call $\asr(R, R)$ simply the \emph{absolute stable rank} of $R$ and denote it by $\asr(R)$.
\end{dfn}
The following reformulation of the absolute stable rank condition was noted by A.~Bak.
\begin{lemma} For a commutative ring $R$ the following statements are equivalent:
\begin{enumerate}
 \item\label{asr_Stein} $R$ satisfies $\ASR_n(R, R)$;
 \item\label{asr_Bak} For any \emph{unimodular} row $(a_1,\ldots,a_n,b,d)\in\Ums(n+2,R)$ there exist $c_1,\ldots,c_n\in R$ such that $(a_1+c_1d,\ldots,a_n+c_nd,b+td)$ is unimodular for all $t\in R$.
\end{enumerate}
\end{lemma}
\begin{proof}
Assume first that $\ASR_n(R)$ holds.
Then for the row $(a_1,\ldots,a_n,d)$ there exist $c_1,\ldots,c_n$ such that $\mathfrak{J}(a_1,\ldots,a_n,d)=\mathfrak{J}(a_1+c_1d,\ldots,a_n+c_nd)$. 
Therefore $a_1+c_1d,\ldots,a_n+c_nd$ together with $b$ generate the unit ideal.
Of course, one could replace $b$ with $b+td$ from the very start.

To show the converse take arbitrary row $(a_1,\ldots,a_n,d)\in{}^{n+1}\!R$ and consider the unimodular row $(a_1,\ldots,a_n,1,d)\in\Ums(n+2,R).$
There exist $c_1,\ldots,c_n$ such that \[(a'_1,\ldots,a'_n,1+td)=(a_1+c_1d,\ldots,a_n+c_nd,1+td)\] is unimodular for all $t\in R$.
Suppose that there exists a maximal ideal $\mathfrak{m}\trianglelefteq R$ such that $a'_1,\ldots,a'_n\in\mathfrak{m}$ but at least one of the elements $d$, $a_i$ is not in $\mathfrak{m}$.
Clearly, $d\notin\mathfrak{m}$ (otherwise $a_i=a'_i-c_id\in\mathfrak{m}$ contrary to the assumption).
On the other hand, $1+td\notin\mathfrak{m}$,
hence the inequality $1+\bar{t}\bar{d}\neq0$ in the residue field $R/\mathfrak{m}$ may not hold for arbitrary $\bar{t}$ 
(take e.g. $\bar{t}=-1/\bar{d}$).
This shows that no such $\mathfrak{m}$ may exist and, therefore, $\mathfrak{J}(a_1,\ldots,a_n,d)=\mathfrak{J}(a'_1,\ldots,a'_n)$.
\end{proof}

Let $R$ be a commutative ring. We denote by $\Max(R)$ its \emph{maximum spectrum} i.e. the set of maximal ideals of $R$ equipped with Zariski topology.
The following result gives an upper estimate for the absolute stable rank (see~\cite[Theorem~2.3]{EO}, \cite[Theorem~3.7]{MKV}).
\begin{thm}
Let $R$ be a commutative ring whose maximum spectrum $\Max(R)$ can be covered with a finite family of subsets $X_1\ldots, X_m$ such that the topological dimension of every subset $X_i$ does not exceed some number $d$. Then $\asr(A)\leqslant d+1$.
\end{thm}
Clearly one has
\[ \sr(R,I)\leqslant\asr(R,I)\leqslant\asr(R)\leqslant \dim(\Max(R))+1\leqslant\dim(\Spec(R))+1. \]

The next lemma is an immediate consequence of the definition of stable rank.
\begin{lemma}\label{lemma:srUnip} Let $v\in\Ums(n, R, I)$ be a unimodular column. Assume that $\sr(I)\leq n-m$ for some $1\leq m \leq n$.
Then one can find a matrix $a\in \M(n-m,m,I)$ such that the column $(e_{n-m},a)v\in \Ums(n-m, R, I)$. \end{lemma} 

Let $v\in R^{2\ell}$ be a vector of the standard vector representation of $\G(\rD_\ell, R)$.
It will be convenient for us to denote by $v^+$ and $v^-$ the upper and lower halves of $v$ i.e. $v^+=(v_1,\ldots, v_\ell)^t$, $v^-=(v_{-\ell},\ldots, v_{-1})^t$.
\begin{lemma}\label{lemma:asrUnip}
 Let $\Phi=\rD_\ell$ and assume that $\asr(R, I)\leq \ell -1$. Then for any $I$-unimodular column $v=(v^+, v^-)^t\in\Ums(2\ell, R, I)$
 there exist $g\in\U(\Sigma^+_\ell, I)$ such that $(g \cdot v)^+ \in \Ums(\ell, R, I)$.
\end{lemma}
\begin{proof} Denote by $J$ the ideal of $R$ spanned by entries of $v^-$. Clearly, $J \leq I$.
By Proposition~\ref{prop:sr_properties} $\sr(I/J) \leq \ell-1$ and therefore
the elementary group $\E(\rA_{\ell-1}, R/J, I/J)$ acts transitively on $\Ums(\ell, R/J, I/J)$ (see~\cite[Theorem~2.3c]{Va69}). 
This implies existence of $h\in \E(\Delta_\ell, R, I)$ such that for $v' = h \cdot v$ one has $(v')^+ \equiv e_1$ modulo $J$.

By hypothesis, we can find $c_2,\ldots, c_\ell\in I$ such that for $v''=v'\cdot \prod\limits_{i=2}^{\ell}t_{-1, -i}(c_i)$ one has
$\mathfrak{J}(v''_{-\ell},\ldots, v''_{-2}) = \mathfrak{J}(v'^-)$. Clearly, $v''_1$ is still congruent to $1$ modulo $J$, therefore 
$(v''_1, v''_{-\ell},\ldots, v''_{-1})$ is $I$-unimodular. Now, applying condition $\sr(I) \leq \ell-1$ once again we find
$d_1,d_3,\ldots, d_{\ell}\in I$ such that for $v'''=t_{-2,1}(d_1) \cdot \prod\limits_{i=3}^{\ell} t_{-2,-i}(d_i) \cdot v''$
the entries $(v'''_1, v'''_{-\ell},\ldots, v'''_{-3})$ form a $I$-unimodular column.

We can find $f_1, f_3,\ldots, f_\ell \in R$ such that $f_1v'''_1+\sum\limits_{i=3}^\ell f_i v'''_{-i} = 1$.
Set $\xi = v'''_1-v'''_2-1 \in I$, $v^4=t_{1,2}(\xi f_1) \cdot \prod\limits_{i=3}^\ell t_{-i,2}(\xi f_i) \cdot v'''$.
Clearly $v^4_2 = v^4_1-1$, therefore $v^{4+}$ is $I$-unimodular.
Summarizing the above, we have found $g\in \EP_\ell(R, I)$ such that $v^4=g \cdot v$
and assertion of the lemma immediately follows from Levi decomposition. \end{proof}

\begin{lemma} \label{lemma:uraction} 
Let $\Phi=\rA_\ell, \rD_\ell$. Denote by $\pi$ standard vector representation of $\G(\Phi, R)$ on $V=R^n$, $n=\ell+1,2\ell$.
Assume that one of the following conditions holds:
\begin{itemize}
 \item $\Phi=\rA_\ell$, $\Gamma=\{ k+1, \ldots, \ell+1\} \subset \Lambda(\pi)$ and $\sr(I)\leq k\leq \ell$;
 \item $\Phi=\rD_\ell$, $\Gamma=\{-l,\ldots, -2, -1\} \subset \Lambda(\pi)$ and $\asr(R, I)\leq \ell-1$. 
\end{itemize}
Then for any $I$-unimodular column $v\in \Ums(n, R, I)$ there exist $x\in \U(\Phi^+, I)$, $y\in \U(\Phi^-, I)$ such that $(yx \cdot v)_\lambda = 0$ for $\lambda\in \Gamma$.
\end{lemma}
\textbf{TODO: Write proof of this lemma}
%TODO: