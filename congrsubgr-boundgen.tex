We now give an application of the above decomposition to the problem of bounded generation.
\begin{dfn}
Given a group $G$ and a set of generators $X$ (for simplicity we assume $X=X^{-1}$ and $1\in X$), one says that the width of $G$ with respect to $X$ equals $N$, if $G=X^N$, i.e. every element of $G$ can be represented as a product of at most $N$ generators from $X$.
\end{dfn}
\begin{lemma}\label{lemma:srRI1}
If $\sr(R,I)=1$, the width of $\SL(2,R,I)$ with respect to $z_\alpha$ does not exceed $4$.
\end{lemma}
\begin{proof}
Let $A=\begin{psmallmatrix}a & b \\ c & d\end{psmallmatrix}\in\SL(2,R,I)$. The first column is $I$-unimodular, therefore there exists $z\in I$ such that $a+cz\in R^*$. Multiply $A$ by $x_{12}(z)$ from the left to obtain an invertible element in the upper left corner. Applying $x_{21}(-c/a)$ from the left and $x_{12}(-b/a)$ from the right, we obtain a diagonal matrix. Thus
\begin{multline*}
A=x_{12}(-z)\cdot x_{21}(c/a)\cdot
\begin{pmatrix} \varepsilon & 0 \\ 0 & 1/\varepsilon \end{pmatrix}
\cdot x_{12}(b/a)=\\
=x_{12}(-z)\cdot
\begin{pmatrix} \varepsilon & 0 \\ 0 & 1/\varepsilon \end{pmatrix}
\cdot x_{21}(y) \cdot x_{12}(b/a),
\end{multline*}
where all the coefficients in transvections lie in $I$ and $\varepsilon\in 1+I$. The above formula is a relative version of Gauss decomposition.
Note that
\begin{multline*}
\begin{pmatrix} \varepsilon & 0 \\ 0 & 1/\varepsilon \end{pmatrix} =
\begin{pmatrix} 1 & -1 \\ 0 & 1 \end{pmatrix}
\begin{pmatrix} 1 & 0 \\ 1-\varepsilon & 1 \end{pmatrix}
\begin{pmatrix} 1 & 1/\varepsilon \\ 0 & 1 \end{pmatrix}
\begin{pmatrix} 1 & 0 \\ \varepsilon^2-\varepsilon & 1 \end{pmatrix} =\\=
\begin{pmatrix} 1 & -1 \\ 0 & 1 \end{pmatrix}
\begin{pmatrix} 1 & 0 \\ 1-\varepsilon & 1 \end{pmatrix}
\begin{pmatrix} 1 & 1+z \\ 0 & 1 \end{pmatrix}
\begin{pmatrix} 1 & 1/\varepsilon-1-z \\ 0 & 1 \end{pmatrix}
\begin{pmatrix} 1 & 0 \\ \varepsilon^2-\varepsilon & 1 \end{pmatrix},
\end{multline*}
therefore
\[
A=z_{21}(1-\varepsilon,-1-z)\cdot x_{12}(1/\varepsilon-1-z)\cdot x_{21}(\varepsilon^2-\varepsilon+y)\cdot x_{12}(b/a). \qedhere
\]
\end{proof}
\begin{rem}
In the above lemma one can restrict himself to the generators of the form $x_\alpha(s)$ and $z_\alpha(s,1)$, $s\in I$, and have an estimate of $5$ factors. It might be useful since $\left\{z_\alpha(s,1)\mid s\in I\right\}=\X_\alpha(I)^{x_\alpha(1)}$ if a $1$-parameter subgroup.
\end{rem}
The following lemma is a corollary of Theorems~5.7 and 5.8 of \cite{LenMorStePrimitiveRoot}.
\begin{lemma}
Let $p$ be a rational prime, $c\perp d$ two coprime integers and $p\perp d$. Then under the assumption of the Generalized Riemann Hypothesis there exist infinitely many primes $q\equiv c\pmod{d}$, such that $p$ is a primitive root modulo $q$.
\end{lemma}
One can modify Lemma~6 of \cite{VavSmSuUnitrEng} (see also \cite{VseUnitrZ1p}) in the following way:
\begin{lemma}\label{lemma:Z1p}
Assume GRH. If $R=\mathbb{Z}[\sfrac{1}{p}]$ and $I\lhd R$, the width of $\SL(2,R,I)$ does not exceed $6$.
\end{lemma}
\begin{proof}
Write $I=(m)$ for some $m\in\mathbb{Z}$, $p\nmid m$. Fix $g\in\SL(2,R,I)$:
\[ g=\begin{pmatrix}x & y \\ z & w\end{pmatrix},\quad
\begin{matrix*}[l]
x=p^\alpha a,\quad & a,b,\alpha,\beta\in\mathbb{Z}, \\ y=p^\beta bm, & p\nmid a,b.
\end{matrix*}\]
\textsc{Case 1:} $\alpha\geqslant\beta$. Since $p^{\alpha-\beta}a\perp bm^2$ and $p\perp bm^2$, there exist infinitely many rational primes $q$ of the form $p^{\alpha-\beta}a+bm^2k$, such that $p$ is a primitive root modulo $q$. One may choose $q$ prime to $b$. Then
\[ g_1=g\cdot x_{21}(mk) =
\begin{pmatrix} p^\beta q & p^\beta bm \\ * & * \end{pmatrix}.\]
There exists $u\geqslant 1$ such that $p^u\equiv b\pmod q$, say $p^u=b+lq$. Then
\[ g_2 = g_1\cdot x_{12}(ml) =
\begin{pmatrix} p^\beta q & mp^{\beta+u} \\ * & * \end{pmatrix}. \]
$g_2\equiv 1\pmod m$, thus $p^\beta q=1+cm$ for some $c$.
\begin{align*}
g_3 = & g_2\cdot x_{21}\left(\dfrac{-c}{p^{\beta+u}}\right) =
\begin{pmatrix} 1 & mp^{\beta+u} \\ * & * \end{pmatrix}, \\
g_4 = & g_3\cdot x_{12}\left(-mp^{\beta+u}\right) =
\begin{pmatrix} 1 & 0 \\ * & * \end{pmatrix}, \\
g_5 = & g_4\cdot x_{21}\left(\dfrac{c}{p^{\beta+u}}\right) =
\begin{pmatrix} 1 & 0 \\ * & * \end{pmatrix}.
\end{align*}
Note that $g_5=g_2\cdot z_{12}\left(-mp^{\beta+u},c/p^{\beta+u}\right)$ and that the two other coefficients in the transvections are multiples of $m$. Thus in this case the length of $g$ does not exceed $4$: $g=x_{21}z_{12}x_{12}x_{21}$.

\textsc{Case 2:} $\alpha<\beta$. Since $\mathbb{Z}[\sfrac{1}{p}]/I$ is finite, there exists $k>0$ such that $p^k\equiv 1\pmod I$. One can choose $k>\beta-\alpha$. Then $k+\alpha>-k+\beta$ and
\[ g\cdot h\left(p^k\right) =
\begin{pmatrix} p^\alpha a & p^\beta bm \\ * & * \end{pmatrix}
\begin{pmatrix} p^k & 0 \\ 0 & p^{-k} \end{pmatrix}=
\begin{pmatrix} p^{k+\alpha} a & p^{-k+\beta} bm \\ * & * \end{pmatrix}.
\]
One can apply Case 1 for the latter, so
$g=x_{21}z_{12}x_{12}x_{21}h^{-1}$,
and $x_{12}h^{-1}$ can be processed in the same way, as in Lemma \ref{lemma:srRI1}:
$g=x_{21}z_{12}x_{12}\cdot z_{12}x_{21}x_{12}$.
\end{proof}
Let $\mathcal{O}_S$ be a Dedekind ring of arithmetic type, defined by a finite set $S$ of places of some global field $k$, and $I$ an ideal in $\mathcal{O}_S$. In case $k$ has a real embedding, then $\K_1(\Phi,\mathcal{O}_S,I)=1$ for any ideal $I\trianglelefteq\mathcal{O}_S$ and any root system $\Phi$ of rank $\geqslant2$ (see~\cite{BassMilnorSerre}). It is shown in \cite{TavgenThesis} that $E(\Phi,I)$ has finite width with respect to $\{x_\alpha(s)\mid\alpha\in\Phi,\ s\in I\}$. From this one immediately gets the bounded generation for $\G(\Phi,R,I)$.
\begin{lemma}
If $\rk\Phi\geqslant2$ and $k$ has a real embedding, then $\G(\Phi,\mathcal{O}_S,I)$ has finite width with respect to $z_\alpha$.
\end{lemma}
\begin{proof}
Because of the Bass---Kolster decomposition it suffices to express every element $g$ of $\SL(2,\mathcal{O}_S,I)$ as a product of a bounded number of generators in a group of rank $2$. Write
\[ g=\begin{pmatrix}1+a & b \\ c & 1+d\end{pmatrix}\in\SL(2,\mathcal{O}_S,I),\quad a,b,c,d\in I. \]

The condition $\det(g)=1$ implies $a+d=bc-ad\in I^2$.

Define the Vaserstein's congruence subgroup as
\[ G(I,I)=\left\{ \begin{pmatrix}1+a & b \\ c & 1+d\end{pmatrix}\in\SL(2,\mathcal{O}_S)\;\middle|\; a,d\in I^2,\ b,c\in I \right\}. \]
Note that $g\cdot z_{21}(a,1)\in G(I,I)$. Indeed,
\[
\begin{pmatrix}
1+a & b \\ c & 1+d
\end{pmatrix} \xmapsto{\cdot z_{21}(a,1)}
\begin{pmatrix}
1+ba-a^2 & b-a-ba-a^2 \\ c+a+ad-ac & 1+bc-ac
\end{pmatrix} \in G(I,I).
\]
Now for a matrix $g'=\begin{psmallmatrix}1+a & b \\ c & 1+d\end{psmallmatrix}\in G(I,I)$ the matrix $x_{21}(-c)\cdot g'\cdot x_{12}(-b)\in\G(2,\mathcal{O}_S,I^2)$. Since for root systems $\Phi\neq\rC_\ell$ with $\rk\Phi\geqslant2$ one has $\E(\Phi,\mathcal{O}_S,I^2)\leqslant\E(\Phi,I)$, we are done for all groups but symplectic.

\textbf{TODO:} What happens for $\rC_\ell$?
\end{proof}