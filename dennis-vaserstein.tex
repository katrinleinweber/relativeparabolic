Let $\Phi$ be an irreducible root system of rank $\ell$.
Let $r$, $s$ be two distinct indices $1\leq r,s \leq \ell$.
From Levi decomposition it follows that
\begin{multline*}
\U(\Phi^+, I)\cdot \U(\Phi^-, I) \cdot \E(\Delta_r, R, I) \cdot \EP_s(R, I) = \\
= \U(\Sigma_r, I)\cdot \U(\Sigma^-_r, I) \cdot \E(\Delta_r, R, I) \cdot \EP_s(R, I) = \hspace{5em} \\
\hspace{5em} = \EP_r(R, I) \cdot \E(\Delta_s, R, I) \cdot \U(\Sigma_s^-, I)\cdot \U(\Sigma_s, I) = \\
= \EP_r(R, I) \cdot \U(\Sigma^-_r\cap \Sigma^-_s, I) \cdot \EP_s(R, I).
\end{multline*}
Denote by $A_{rs}$ any of the equal subsets from the previous formula. 

The main result of this subsection is the following theorem which is a relative analogue of Dennis---Vaserstein decomposition~\cite[Theorem~2.5]{St78}.

\begin{thm}\label{thm:DennisVaserstein}
The relative elementary subgroup $\E(\Phi, R, I)$ coincides with the subset $A_{rs}$ under the following assumptions on $I$:
\[\begin{array}{l@{\quad}l@{\quad}l}
\Phi = \rA_\ell, \rB_\ell,\ \ell\geqslant 2, & (s, r)=(1, \ell), & \sr(I) \leqslant \ell-1; \\     
\Phi = \rD_\ell,\ \ell\geqslant 4,           & (s, r)=(1, \ell), & \sr(I) \leqslant \ell-2; \\ 
\Phi = \rE_\ell,\ \ell=6,7,             & (s, r)=(\ell, 2), & \sr(I) \leqslant \ell-3; \\ 
\Phi = \rE_\ell,\ \ell=6,7,             & (s, r)=(\ell, 1), & \asr(I)\leqslant \ell-2.
\end{array}\]
\end{thm}
The proof of Theorem~\ref{thm:DennisVaserstein} occupies the rest of this section.
Consider the usual conjugation action of $\E(\Phi, R)$ on $\E(\Phi, R, I)$. 
This action induces an action of $\E(\Phi, R)$ on the set $\mathfrak{S}$ of all subsets of $\E(\Phi, R, I)$.
On the other hand, $\E(\Phi, R, I)$ acts on $\mathfrak{S}$ by left multiplication.
Denote by $N_{rs}$ and $L_{rs}$ stabilizers of $A_{rs} \in \mathfrak{S}$ with respect to these actions.
\[ N_{rs} = \left\{ g\in \E(\Phi, R)\ \middle|\ g \cdot A_{rs} \cdot g^{-1} = A_{rs} \right\},\quad L_{rs}= \left\{ g\in \E(\Phi, R, I)\ \middle|\ g \cdot A_{rs} = A_{rs} \right\}. \]

It is easy to see that $N_{rs}$ normalizes $L_{rs}$. Indeed, for $g\in N_{rs}$, $h\in L_{rs}$ one has
$$h^g \cdot A_{rs} = g^{-1} \cdot h \cdot g \cdot A_{rs} \subseteq g^{-1} \cdot h \cdot A_{rs} \cdot g \subseteq A_{rs}^g \subseteq A_{rs}.$$

\begin{lemma}\label{lemma:dv_unipotent}
For any $1\leq i\leq n$ the following statements hold (together with their analogues for $\Phi^-$).
\begin{enumerate} 
\item $\U(\Phi^+, I) = \X_{\alpha\ssub{i}}(I)\cdot \U(\Phi^+\setminus\{\alpha\ssub{i}\}, I) = \U(\Phi^+\setminus\{\alpha\ssub{i}\}, I)\cdot \X_{\alpha\ssub{i}}(I)$.
\item For any $\xi\in R$ one has $\U(\Phi^+\setminus\{\alpha_i\}, I)^{x_{-\alpha\ssub{i}}(\xi)} \subseteq \U(\Phi^+, I)$.
\item $\U(\Phi^+, I)\cdot \U(\Phi^-, I) \subseteq \U(\Phi^+\setminus \{\alpha_i\}, I) \cdot \U(\Phi^-, I) \cdot \X_{\alpha\ssub{i}}(I) \cdot \X_{-\alpha\ssub{i}}(I)$.
\end{enumerate}
\end{lemma}
\begin{proof}
 The first two statements easily follow from Chevalley commutator formula while the third one is a formal consequence of the first two.
\end{proof}

The following lemma is a relative version of the main reduction used by M.~Stein in~\cite{St78} for the proof of the absolute Dennis--Vaserstein decomposition.
\begin{lemma}\label{lemma:Stein_reduction}
Assume that there exists a subset $\widetilde{L_r} \subseteq \E(\Delta_r, R, I)$ with the following properties:
\begin{enumerate}[label=(\alph*)] 
 \item\label{stein_cond1} One has $\U(\Sigma^+_r, I)\cdot \U(\Sigma^-_r, I) \cdot \E(\Delta_r, R, I) \subseteq \U(\Phi^+, I)\cdot \U(\Phi^-, I) \cdot \widetilde{L_r}.$
 \item\label{stein_cond2} One has $\X_{-\alpha\ssub{r}}(I)^{\widetilde{L_r}} \subseteq \EP_s(R, I).$
\end{enumerate}
Then $\X_{-\alpha_r}(I) \subseteq L_{rs}$ and $\EP_s \subseteq N_{rs}.$
\end{lemma}
\begin{proof} Set $A:=\U(\Phi^+, I)\cdot \U(\Phi^-, I) \cdot \widetilde{L_r} \cdot \EP_s(\Phi, R, I).$
From assumption~\ref{stein_cond1} of the lemma it follows that $A_{rs}=A$.
Observe first that Lemma~\ref{lemma:dv_unipotent} and condition~\ref{stein_cond2} together imply
\begin{multline*}
A \subseteq \U(\Phi^+\setminus \{\alpha_r\}, I) \cdot \U(\Phi^-, I) \cdot \X_{\alpha\ssub{r}}(I) \cdot \X_{-\alpha\ssub{r}}(I) \cdot \widetilde{L_r} \cdot \EP_s(R, I) \subseteq \\ 
\subseteq \U(\Phi^+\setminus\{\alpha_r\}, I) \cdot \U(\Phi^-, I) \cdot \X_{\alpha\ssub{r}}(I) \cdot \X_{-\alpha\ssub{r}}(I) \cdot \widetilde{L_r} \cdot \EP_s(R, I) \subseteq \\
\subseteq \U(\Phi^+\setminus\{\alpha_r\}, I) \cdot \U(\Phi^-, I) \cdot \widetilde{L_r} \cdot \U(\Sigma_r, I) \cdot \EP_s(R, I)  \cdot \EP_s(R, I) \subseteq \\
\subseteq \U(\Phi^+\setminus\{\alpha_r\}, I) \cdot \U(\Phi^-, I) \cdot \widetilde{L_r} \cdot \EP_s(R, I).
\end{multline*}
Now applying Lemma~\ref{lemma:dv_unipotent} we get the following inclusions:
\begin{align*}
& A^{\X_{-\alpha\ssub{r}}} \subseteq \U(\Phi^+, I) \cdot \U(\Phi^-, I) \cdot \widetilde{L_r} ^{\X_{-\alpha\ssub{r}}} \cdot \EP_s(R, I) \subseteq A; \\
& \X_{-\alpha\ssub{r}}(I) \cdot A \subseteq \U(\Phi^+, I) \cdot \X_{-\alpha\ssub{r}}(I) \cdot \U(\Phi^-, I) \cdot \widetilde{L_r} \cdot \EP_s(R, I) = A.
\end{align*}
To prove the second part of the statement observe first that $\EP_s$ is generated by $\X_{\alpha\ssub{i}}$ for $1\leqslant i\leqslant n$ and $\X_{-\alpha\ssub{i}}$ for $i\neq s$.
%TODO: Find proper reference for this fact
We have just shown that $\X_{-\alpha_r}\subseteq N_{rs}$.
On the other hand, inclusions $\X_{\alpha\ssub{k}} \subseteq N_{rs}$ for $\ 1\leqslant k\leqslant \ell$ and $\X_{-\alpha\ssub{k}} \subseteq N_{rs}$ for $k\neq r,s$ are obvious.
\end{proof}

\begin{proof}[Proof of Theorem~\ref{thm:DennisVaserstein}]
We first show that the assumed stability conditions on $I$ imply the assumptions of Lemma~\ref{lemma:Stein_reduction}.
Consider the following two subsets of $\Lambda(\pi)$:
$$\Gamma = \varpi_s- (\Sigma_s^+\cap \Delta_r),\quad \Gamma_0 = \{\lambda \in \Gamma \mid \lambda - \alpha_r \in \Lambda(\pi) \}.$$
With the choice of $s$ as in the assumption, $\Gamma$ is the set of weights of an irreducible representation of $\G(\Delta_r, R)$ corresponding to the same highest weight $\varpi_s$.
The subsystem $\Delta_r$ has type $\rA_{\ell-1}$ in all cases except the last one.
It is also clear that $|\Gamma_0|=1$ for $\Phi=\rA_\ell, \rB_\ell$, $|\Gamma_0|=2$ for $\Phi=\rD_\ell$, and $|\Gamma_0|=3$ in case $\Phi=\rE_\ell$ and $r=2$.
If $\Phi=\rE_\ell$, $r=1$, the subsystem $\Delta_r$ has type $\rD_{\ell-1}$ and $|\Gamma_0|=\ell-1$.

%TODO: PICK AS L_r!!!
Let $\widetilde{L_r}$ be the set of all elements $g$ of $\E(\Delta_r,R, I)$ such that $(g \cdot v^+)_\lambda = 0$ for $\lambda\in\Gamma_0$.
In any of  the specified cases the assumption on $I$ allows us to apply Lemma~\ref{lemma:uraction} to the subsystem $\Delta_r$ and find
$x\in\U(\Delta_r\cap\Phi^+, I)$, $y\in \U(\Delta_r\cap\Phi^-, I)$ such that $yx\cdot g \in \widetilde{L_r}$.
This proves the first condition of Lemma~\ref{lemma:Stein_reduction}, indeed:
\[ \U(\Sigma^+_r, I) \U(\Sigma^-_r, I) \cdot g = \U(\Sigma^+_r, I) x^{-1} \cdot \U(\Sigma^-_r, I)^{x^{-1}} y^{-1} (yxg) \subseteq \U(\Phi^+, I) \U(\Phi^-, I) \widetilde{L_r}. \]
To prove the second assumption of Lemma~\ref{lemma:Stein_reduction} notice that by the definition of $\Gamma_0$ for any $s\in I$, $ g\in\widetilde{L_r}$ one has $x_{-\alpha_r}(s) \cdot g \cdot v^+ = g \cdot v^+$ hence
\[ \X_{-\alpha\ssub{r}}(I)^{\widetilde{L_r}} \subseteq \U(\Phi^-, I) \cap \Stab(v^+) \subseteq \E(\Delta_s, R, I) \subseteq \EP_s(R, I). \]

From now on we can assume that the conclusion of Lemma~\ref{lemma:Stein_reduction} holds and $\EP_s$ normalizes $A_{rs}$.
Notice that in view of Theorem~\ref{thm:Stepanov} it suffices to show that the following two families of elements are contained in $L_{rs}$:
\begin{itemize} \item $z_{\alpha}(s, \xi)$, $s\in I$, $\xi \in R$, $\alpha\in\Sigma^-_s$;
\item $x_{\beta}(s)$, $s \in I$, $\beta \in \Phi$. \end{itemize}
Since $\EP_s \subseteq N_{rs}$, the inclusions for the first family of elements follow from those for the second one.
We already know that $\U(\Phi^+, I) \subseteq L_{rs}$, hence it only suffices to show $\U(\Phi^-, I) \subseteq L_{rs}$.

Notice that the Weyl group $W(\Delta_s)$ acts transitively on $\Delta_s$, therefore in view of relation~\ref{rel:R3} the subgroup
$W_s := \langle w_\alpha(1) \mid \alpha\in\Delta_s\rangle \leqslant \EP_s$ acts transitively on the set of root subgroups $\X_\alpha(I)$, $\alpha\in \Delta_s$
and we get that that $\U(\Delta_s \cap \Phi^-, I)\subseteq L_{rs}$.

%Now denote by $\widetilde{\alpha}$ the maximal root of $\Phi$.
By our assumptions on $\Phi$ we have $m_s(\alpha\ssub{\mathrm{max}})=1$, hence every two roots $\alpha, \beta \in \Sigma^-_s$ of the same length
have the same $s$-shape and
%(i.\,e. $\shape(\{s\}, \alpha = \shape(\{s\}, \beta)$).
hence by Lemma~\ref{lemma:abs} are interchanged by the action of $W(\Delta_s)$. % $\alpha$ and $\beta$ if their length is equal.
%By the same argument as above we get that $W(\Delta_s)$.
Since $\X_{\alpha\ssub{s}} \subseteq L_{rs}$ by the same argument as above, we get $\U({\Sigma^-_s}^>, I)\subseteq L_{rs}$.
This completes the proof of the theorem for $\Phi\neq \rB_\ell$. 

In the case $\Phi=\rB_\ell$ we have to show additionally that $\X_{-\alpha^<_{\mathrm{max}}}(I) \leqslant L_{rs}$, where $\alpha^<_{\mathrm{max}}$
denotes the maximal short root of $\rB_\ell$.
By Chevalley commutator formula we have
\[ x_{-{\alpha^<_{\mathrm{max}}}}(s) = x_{-1,0}(s) = x_{1,-2}(s) \cdot x_{-1,-2}(-s)^{x_{-2, 0}(-1)} \cdot x_{-1, 2}(\xi),\ s\in I. \]
On the other hand, $x_{-2,0}(1) \in \EP_s$, and we have already shown $\X_{-1,-2}(I), \X_{-1, 2}(I) \subseteq L_{rs}$.
This concludes the proof of the theorem.
\end{proof}
