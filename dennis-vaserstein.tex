Let $\Phi$ be an irreducible root system of rank $\ell$.
In what follows, $r$ and $s$ are two distinct integers such that $1 \leq r, s \leq \ell$.
From Levi decomposition~\eqref{rel:Levi-decomp} it follows that the following four subsets of $\E(\Phi, R, I)$ are equal:
\begin{multline*}
A_{rs} := \U(\Phi^+, I)\cdot \U(\Phi^-, I) \cdot \E(\Delta_r, R, I) \cdot \EP_s(R, I) = \\
= \U(\Sigma_r, I)\cdot \U(\Sigma^-_r, I) \cdot \E(\Delta_r, R, I) \cdot \EP_s(R, I) = \hspace{5em} \\
\hspace{5em} = \EP_r(R, I) \cdot \E(\Delta_s, R, I) \cdot \U(\Sigma_s^-, I)\cdot \U(\Sigma_s, I) = \\
= \EP_r(R, I) \cdot \U(\Sigma^-_r \cap \Sigma^-_s, I) \cdot \EP_s(R, I).
\end{multline*}

%The relative elementary subgroup $\E(\Phi, R, I)$ admits Dennis---Vaserstein decomposition, i.\,e. $\E(\Phi, R, I) = A_{rs}$ under the following assumptions on $I$ and $\Phi$.
%\[\begin{array}{l@{\qquad}c@{\quad}c@{\qquad}l}
%\Phi                                 & s    & r      & \text{ring condition} \\ \hline\vphantom{\Bigl(}
%\rA_\ell,\ \ell\geqslant 2           & 1    & \ell   & \sr(I) \leqslant \ell-1 \\    
%\rB_\ell,\ \ell\geqslant 2           & 1    & \ell   & \sr(I) \leqslant \ell-1 \\
%\rC_\ell,\ \ell \geqslant 2          & 1    & \ell   & \sr(I) \leqslant \ell-1 \\
%\rD_\ell,\ \ell\geqslant 4           & 1    & \ell   & \sr(I) \leqslant \ell-2 \\ 
%\rD_\ell,\ \ell\geqslant 4           & \ell & \ell-1 & \sr(I) \leqslant 2      \\
%\rE_\ell,\ \ell=6,7                  & \ell & 2      & \sr(I) \leqslant \ell-3 \\ 
%\rE_\ell,\ \ell=6,7                  & \ell & 1      & \asr(I)\leqslant \ell-2 \\ 
%\rF_4                                & 4    & 1      & \asr(I)\leqslant 3 \end{array}\]
Denote by $N_{rs}$ the submonoid of $\E(\Phi, R)$ consisting of elements which normalize the subset $A_{rs}$, i.\,e.
\[ N_{rs} = \left\{ g\in \E(\Phi, R)\ \middle|\ g \cdot A_{rs} \cdot g^{-1} \subseteq A_{rs} \right\}. \]

\begin{lemma}\label{lemma:dv-normal} 
For every $\alpha \in \Delta_{\{r,s\}} \cup (\Phi^+ \setminus (\Sigma^+_r \cap \Sigma^+_s))$ one has $X_\alpha(R) \subseteq N_{rs}$. \end{lemma}
\begin{proof}
Notice that for every $i$ the group $\EP_i(R, I)$ is normalized by $\EP_i(R)$, hence a fortiori it is normalized by $X_\alpha(R)$, $\alpha \in S_i^+$.
Since $\U(\Sigma_r^- \cap \Sigma_s^-, I)$ is normalized by $\E(\Delta_{\{r,s\}}, R)$ we obtain the assertion of the lemma for $\alpha \in \Delta_{\{r, s\}}$.

Now if $\alpha$ lies in $\Phi^+ \setminus \Sigma_r \cap \Sigma_s$ it is contained either in $\Delta_r$ or in $\Delta_s$.
Assume, for example, the latter. By Levi decomposition we have $\U(\Sigma_r^- \cap \Sigma_s^-, I)^{X_\alpha(R)} \subseteq \U(\Sigma_s^-, I) \subseteq A_{rs}$
which again implies the assertion of the lemma.
\end{proof}

\begin{proof}[Proof of \cref{thm:DennisVaserstein}]
First of all, notice that in the case $\Phi=\rG_2$ the assertion of the theorem formally follows from \cref{thm:srRI1} and identities~\ref{rel:h-comm}--\ref{rel:h-w}.
Since $A_{rs} = \E(\Phi, R, I)$ implies $A_{sr} = \E(\Phi, R, I)$, it suffices to consider only the following possibilities for $\Phi$, $s$, $r$.
\begin{table}[htb]
\[\begin{array}{l @{\qquad} l @{\qquad} c @{\quad} c @{\quad} c @{\qquad} c @{\qquad} c @{\qquad} c}
\Phi                                 & s    &r      & |\Lambda(\pi)| & \text{type of $\pi$} & \text{type of $\Delta_r$} & |\Lambda(\pi')|& |\Gamma|  \\ \hline\vphantom{\Bigl(}
\rA_\ell,\ \ell\geqslant 2           & 1    &\ell   & \ell+1         & \text{natural}       & \rA_{\ell-1}              & \ell           & 1  \\     
\rB_\ell,\ \ell\geqslant 2           & 1    &\ell   & 2\ell+1        & \text{natural}       & \rA_{\ell-1}              & \ell           & 1  \\     
\rC_\ell,\ \ell\geqslant 2           & 1    &\ell   & 2\ell          & \text{natural}       & \rA_{\ell-1}              & \ell           & 1  \\
\rD_\ell,\ \ell\geqslant 4           & 1    &\ell   & 2\ell          & \text{natural}       & \rA_{\ell-1}              & \ell           & 2  \\ 
\rD_\ell,\ \ell\geqslant 4           & \ell &\ell-1 & 2^{\ell-1}     & \text{half-spinor}   & \rA_{\ell-1}              & \ell           & \ell-2  \\
\rE_\ell,\ \ell=6,7                  & \ell &2      & 27, 56         & \text{minimal}       & \rA_{\ell-1}              & \ell           & 3       \\ 
\rE_\ell,\ \ell=6,7                  & \ell &1      & 27, 56         & \text{minimal}       & \rD_{\ell-1}              & 2(\ell-1)      & \ell-1  \\
\rF_4                                & 4    &1     & 26             & \text{minimal}       & \rC_3                     & 6              & 3\end{array}\]
 \caption{List of the cases considered in the proof of \cref{thm:DennisVaserstein}.} \label{table:dv-reps}
\end{table}

Denote by $L_{rs}$ the set consisting of all the elements $g\in \E(\Phi, R, I)$ such that $g \cdot A_{rs} \subseteq A_{rs}$.
It is easy to see that $L_{rs}$ contains $\EP_r(R, I)$ and is normalized by $N_{rs}$, i.\,e. ${L_{rs}}^{N\ssub{rs}}\subseteq L_{rs}$. Indeed, for $g\in N_{rs}$, $h\in L_{rs}$ one has
\begin{equation}\label{rel:NnormL} h^g \cdot A_{rs} = g^{-1} \cdot h \cdot g \cdot A_{rs} \subseteq g^{-1} \cdot h \cdot A_{rs} \cdot g \subseteq {A_{rs}}^g \subseteq A_{rs}.\end{equation}

The first step of the proof is to show that $N_{rs}$ contains sufficiently many elements. 
This is accomplished in~\cref{lemma:Stein_reduction} below, where we invoke stability conditions for the first time.
We show that $N_{rs}$ contains the root subgroup $X_{-\alpha\ssub{r}}(R)$, which together with \cref{lemma:dv-normal} and \cref{item-egen} immediately implies that $N_{rs}$ contains the Levi subgroup $\E(\Delta_s, R)$ and hence the extended Weyl group $\widetilde{W}(\Delta_s)$.

The next goal is to demonstrate the inclusion $\U(\Sigma_s, R) \subseteq N_{rs}$ (which implies $\EP_s(R) \subseteq N_{rs}$).
All the possibilities for $\Phi$, $s$ and $r$ fall into one of the following three cases:
\begin{enumerate}
 \item \textit{Case $m_s(\alpha_{\mathrm{max}})=1$ and $\Phi$ is simply laced.}
  In this case all the roots of $\Sigma^+_s$ have the same $s$-shape and length, hence the assertion follows from \cref{item-trans2}, \eqref{rel:NnormL} and the fact that we already have the inclusion $X_{\alpha_s}(R) \subseteq N_{rs}$.
 \item \textit{Case $\Phi = \rB_\ell$.} Using the argument from the previous case we get that $N_{rs}$ contains the subgroup generated by the root subgroups $X_\alpha(R)$, where $\alpha$ varies over long roots of $\Sigma_s^+$. 
  It remains to prove the inclusion $X_{1,0}(R) \subset N_{rs}$.
  Specializing identity~\eqref{rel:CCF} we get \begin{equation*} \label{rel:CCF-specBC} x_{1,0}(ab) = [x_{1, 2}(a), x_{2, 0}(b)] \cdot x_{1,-2}(-a b^2) \end{equation*}
  Since $X_{2, 0}(R)$ and $X_{1,-2}(R)$ are contained in $N_{rs}$, the inclusion $X_{1,0}(R) \subseteq N_{rs}$ follows.
 \item \textit{Case $\Phi = \rC_\ell, \rF_4$.}
 We settle these remaining cases in \cref{lemma:DVST} where we invoke the stability conditions one more time. 
 In fact, we prove a stronger result, namely that $N_{rs}$ contains $X_{-\alpha\ssub{s}}(R)$ and, consequently, $N_{rs} = \E(\Phi, R)$.
\end{enumerate}

In view of \cref{prop:Stepanov} to prove the theorem it suffices to show that $L_{rs}$ contains the generating set $\mathcal{Z}(\Sigma_s^-)$ of $\E(\Phi, R, I)$.
From $\mathcal{Z}(\Sigma_s^-) \subseteq \mathcal{X}^{\EP_s(R)}$ and \eqref{rel:NnormL} it follows that we only have to show $\mathcal{X} \subseteq L_{rs}$.

The last inclusion follows from the same transitivity argument as above and the fact that $X_\alpha(I) \subseteq \EP_r(R, I) \subseteq L_{rs}$ for $\alpha \in \Delta_r \cup \Sigma_r$.
For $\Phi=\rB_\ell$ one has to once again apply Chevalley commutator formula~\eqref{rel:CCF}:
$$X_{0,1}(I) \subseteq X_{2,1}(I) \cdot X_{2,1}(I)^{x_{0,2}(1)} \cdot X_{-2,1}(I) \subseteq L_{rs} \cdot L_{rs}^{N_{rs}} \cdot L_{rs} \subseteq L_{rs}. \qedhere$$ 
\end{proof}

\begin{lemma}\label{lemma:dv_unipotent} For any $1\leq i\leq n$ the following statements hold. 
\begin{thmlist} \item \label{item-dvu1} $\U(\Phi^+, I) = X_{\alpha\ssub{i}}(I)\cdot \U(\Phi^+\setminus\{\alpha\ssub{i}\}, I) = \U(\Phi^+\setminus\{\alpha\ssub{i}\}, I)\cdot X_{\alpha\ssub{i}}(I)$.
\item \label{item-dvu2} For any $\xi\in R$ one has $\U(\Phi^+\setminus\{\alpha_i\}, I)^{x_{-\alpha\ssub{i}}(\xi)} \subseteq \U(\Phi^+, I)$.
\item \label{item-dvu3} $\U(\Phi^+, I)\cdot \U(\Phi^-, I) \subseteq \U(\Phi^+\setminus \{\alpha_i\}, I) \cdot \U(\Phi^-, I) \cdot X_{\alpha\ssub{i}}(I) \cdot X_{-\alpha\ssub{i}}(I)$.
\end{thmlist} \end{lemma}
\begin{proof} The first two statements easily follow from Chevalley commutator formula~\eqref{rel:CCF} while the third one is a formal consequence of the first two. \end{proof}

\begin{lemma}\label{lemma:Stein_reduction}
Under the assumptions of \cref{thm:DennisVaserstein} one has $X_{-\alpha_r}(R) \subseteq N_{rs}.$
\end{lemma}
\begin{proof}
Let $\pi$ be the irreducible representation of $\G(\Phi, R)$ with the highest weight $\varpi_s$ (see Table~\ref{table:dv-reps}).
Notice that $\Delta_r$ is an irreducible classical root system of type $\rA_\ell$, $\rC_\ell$ or $\rD_\ell$.
Denote by $(\pi', V')$ the irreducible component of the restriction of $\pi$ to $\G(\Delta_r, R)$ containing the highest weight vector $v^+$ of $\pi$.
In all the cases under consideration, $\pi'$ is isomorphic to the natural representation of $\G(\Delta, R)$.
Set $\Gamma = \{\lambda \in \Lambda(\pi') \mid \lambda - \alpha_r \in \Lambda(\pi) \}.$

The set $\Gamma$ can be visualized using weight diagrams in the following manner.
After removing all bonds marked $r$ the weight diagram of $\pi$ splits into several connected components.
The subset $\Lambda(\pi') \subseteq \Lambda(\pi)$ corresponds to the component of the diagram containing the vertex marked $\varpi_s$.
Clearly, $\Gamma$ consists of the weights of $\Lambda(\pi')$ incident to the removed bonds.

From a consideration of the weight diagrams (see~\cite{PSV98}) we determine the number of elements in $\Lambda(\pi')$ and $\Gamma$ (see~Table~\ref{table:dv-reps}).

Consider the subset $B \subset \E(\Delta_r, R, I)$ consisting of elements $g$ such that $(g \cdot v^+)_\lambda = 0$ for all $\lambda\in\Gamma$.
Set $A:=\U(\Phi^+, I)\cdot \U(\Phi^-, I) \cdot B \cdot \EP_s(R, I).$

Applying \cref{lemma:uraction} to the subsystem $\Delta_r$ we find
$x\in\U(\Delta_r^+, I)$, $y\in \U(\Delta_r^-, I)$ such that $yx\cdot g \in B$.
Consequenty, this shows that $A = A_{rs}$, indeed:
\begin{equation*} \U(\Sigma^+_r, I) \cdot \U(\Sigma^-_r, I) \cdot g = \U(\Sigma^+_r, I) x^{-1} \cdot \U(\Sigma^-_r, I)^{x^{-1}} y^{-1} (yxg) \subseteq \U(\Phi^+, I) \cdot \U(\Phi^-, I) \cdot B. \end{equation*}

Notice that by the definition of $\Gamma$ and \cref{lemma:Matsumoto} for any $s\in I$, $ g\in B$ one has $x_{-\alpha_r}(s) \cdot g \cdot v^+ = g \cdot v^+$.
Consequently, one has
\[ X_{-\alpha\ssub{r}}(I)^{B} \subseteq \U(\Phi^-, I) \cap \Stab(v^+) \subseteq \U(\Delta_s^-, I) \subseteq \EP_s(R, I), \]
which implies the following inclusion
\begin{multline*}
X_{\alpha_r}(I) \cdot X_{-\alpha_r}(I) \cdot B \cdot \EP_s(R, I) \subseteq 
X_{\alpha_r}(I) \cdot B \cdot \EP_s(R, I) \cdot \EP_s(R, I) \subseteq \\
\subseteq B \cdot \U(\Sigma_r, I) \cdot \EP_s(R, I) =
B \cdot \EP_s(R, I).
\end{multline*}
Together with the third statement of \cref{lemma:dv_unipotent} this shows that
\begin{equation*} \label{rel:sred}
A = \U(\Phi^+\setminus\{\alpha_r\}, I) \cdot \U(\Phi^-, I) \cdot B \cdot \EP_s(R, I).
\end{equation*}
Finally, since $[B, X_{-\alpha_r}(R)] \subseteq \U(\Sigma_r^-, R) \cap \E(\Phi, R, I) = \U(\Sigma_r^-, I)$ we obtain the assertion of the lemma, indeed
\begin{equation*} \label{rel:ninv} A^{X_{-\alpha\ssub{r}}(R)} = \U(\Phi^+, I) \cdot \U(\Phi^-, I) \cdot B ^{X_{-\alpha\ssub{r}}(R)} \cdot \EP_s(R, I) = A. \qedhere \end{equation*}
\end{proof}

\begin{lemma}\label{lemma:DVST}
Assume that one of the following holds:
\begin{lemlist}
 \item \label{lemma:DVcaseCl} $\Phi=\rC_\ell$ and $\sr(I) \leq \ell-1$;
 \item \label{lemma:DVcaseF4} $\Phi=\rF_4$ and $\asr(I) \leq 3$.
\end{lemlist}
Then one has $X_{-\alpha_s}(R) \subseteq N_{rs}$.
\end{lemma}
\begin{proof}
Consider the subset $C \subseteq \EP_s(R, I)$ consisting of elements satisfying the following condition:
\begin{itemize}
 \item \textsc{Case $\Phi=\rC_\ell$.} Matrix entries $(g_{i,2})$, $i=2,\ldots, \ell$ form an $I$-unimodular column of height $\ell-1$.
 \item \textsc{Case $\Phi=\rF_4$.} Matrix entries $(g_{\lambda, \lambda}, \ldots, g_{\lambda, \lambda - \alpha_3 - \alpha_2 - \alpha_1})$, where $\lambda = \varpi_4 - \alpha_4$ (see~\cite[Fig.~26]{PSV98}),
                                     form an $I$-unimodular column of height $4$. \end{itemize}

Set $A' = \EP_\ell(R, I) \cdot \U(\Sigma_s^- \cap \Sigma_r^-, I) \cdot C$.
By \cref{item:asrUnipC} (in the case $\Phi=\rC_\ell$) or \cref{cor:embeddingBD} (in the case $\Phi=\rF_4$) applied to the subgroup $\G(\Delta_s, R, I)$ 
we find for every $g \in \EP_s(R, I)$ an element $x \in \U(\Sigma_r \cap \Delta_s, I)$ such that $xg \in C$.  
Notice that one immediately gets the equality $A_{rs} = A'$ from this.
Indeed for $g\in \EP_s(R, I)$ one has
\begin{equation*} \EP_r(R, I) \cdot \U(\Sigma_s^- \cap \Sigma_r^-, I) \cdot g \subseteq 
 \EP_r(R, I)x^{-1}  \cdot \U(\Sigma_s^-, I) \cdot xg \subseteq A'. \end{equation*}

By the very definition of $C$, for every $g \in C$ one can choose $y \in \U(\Sigma_s \cap \Delta_r, I)$ such that $(y \cdot g)_{\varpi_s,\varpi_s - \alpha_s} = 0$.
Consequently, for every $g\in C$ one has
\begin{multline*}
 \EP_r(R, I) \cdot \U(\Sigma_s^- \cap \Sigma_r^-, I) \cdot g \subseteq \EP_r(R, I) y^{-1} \cdot \U(\Sigma_r^-\cap \Sigma_s^-, I)^{y^{-1}} \cdot y g \subseteq \\
  \subseteq \EP_r(R, I) \cdot \U(\Sigma_r^-, I) \cdot y g
\end{multline*}
Notice that the matrix entry $(yg)_{\varpi_s,\varpi_s}$ is invertible.
From the choice of $y$ it follows that for every $\xi\in R$ the element $g_1:=(yg)^{x_{-\alpha_s}(\xi)}$
satisfies the assumptions of \cref{lemma:Chevalley-Matsumoto} and therefore can be rewritten as $g_1 = uh$ for some $u \in \U(\Sigma_s^-, I)$, $h \in \EP_1(R, I)$.
Consequently, one has
\begin{multline*} {A_{rs}}^{X_{-\alpha_{s}}(R)} \subseteq \EP_r(R, I)^{X_{-\alpha_{s}}(R)} \cdot \U(\Sigma_r^-, I)^{X_{-\alpha_{s}}(R)} \cdot \U(\Sigma_s^-, I) \cdot \EP_s(R, I) \subseteq \\
 \subseteq \EP_\ell(R, I) \cdot \U(\Phi^-, I) \cdot \EP_s(R, I) \subseteq A_{rs} \end{multline*}
 as claimed.
\end{proof}