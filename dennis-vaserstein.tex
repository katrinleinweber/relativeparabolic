Throughout the present section we denote by $\EP_s(R, I)$ the subgroup $\E(\Delta_s, R, I) \cdot \U(\Sigma_s, I)$, $1 \leq s \leq n$.
Set $\EP_s := \EP_s(R, R) = \E(\Delta_s, R) \cdot \U(\Sigma_s, R)$. 

\subsection{Main reduction.}
Let $i$, $j$ be two distinct indices $1\leq i,j \leq n$. From Levi decomposition it follows that the following four subsets of $\E(\Phi, R, I)$ are equal to each other:
\begin{multline}\nonumber \U(\Phi, I)\cdot \U(\Phi^-, I) \cdot \E(\Delta_i, R, I) \cdot \EP_j(R, I) = \U(\Sigma_i, I)\cdot \U(\Sigma^-_i, I) \cdot \E(\Delta_i, R, I) \cdot \EP_j(R, I) = \\
= \EP_i(R, I) \cdot \E(\Delta_j, R, I) \cdot \U(\Sigma_j^-, I)\cdot \U(\Sigma_j, I) = \EP_i(R, I) \cdot \U(\Sigma^-_i\cap \Sigma^-_j, I) \cdot \EP_j(R, I). \end{multline}
Denote by $A_{ij}$ any of the equal subsets from the previous formula. 
Clearly, $A_{ij}$ is invariant with respect to conjugation by root subgroups $X_{\alpha\ssub{k}}$ for $1\leq k\leq n$ and $X_{-\alpha\ssub{k}}$ for $k\neq i,j$.

\begin{lemma}\label{lemma:dv_unipotent} For any $1\leq i\leq n$ the following statements hold. \begin{enumerate} 
\item $\U(\Phi^\pm, I) = X_{\pm\alpha\ssub{i}}(I)\cdot \U(\Phi^\pm\setminus\{\pm\alpha\ssub{i}\}, I) = \U(\Phi^\pm\setminus\{\pm\alpha\ssub{i}\}, I)\cdot X_{\pm\alpha\ssub{i}}(I).$
\item For any $\xi\in R$ one has $x_{\mp\alpha\ssub{i}}(\xi)\cdot \U(\Phi^\pm\setminus\{\alpha_i\}, I)\cdot x_{\mp\alpha\ssub{i}}(\xi)^{-1} \subseteq \U(\Phi^\pm, I).$
\item $\U(\Sigma_i, I)\cdot \U(\Sigma^-_i, I) \subseteq \U(\Sigma_i\setminus \{\alpha_i\}, I) \cdot \U(\Phi^-, I) \cdot X_{\alpha\ssub{i}}(I) \cdot X_{-\alpha\ssub{i}}(I)$.
\end{enumerate} \end{lemma}
\begin{proof}
 The first two statements follow from Chevalley commutator formula. The last statement is a consequence of the first two.
\end{proof}

\begin{lemma}\label{lemma:conj_lemma} Let $G$ be a group, $H\leq G$ be a subgroup and $A, B\subseteq G$ be two subsets.
Assume that $A$ is invariant with respect to left multiplication by elements of $B$ and conjugation by elements of the subgroup $H$.
Then $A$ is invariant with respect to left multiplication by elements of the set $B^H = \{b^h \vert b\in B, h\in H\}$. \end{lemma}
\begin{proof} $b^h \cdot A \subseteq h^{-1}\cdot b \cdot h \cdot A \subseteq h^{-1} \cdot b \cdot A \cdot h \subseteq h^{-1} \cdot A \cdot h \subseteq A. \qedhere$ \end{proof}

\begin{lemma}\label{Stein_reduction}
Let $\Phi_l = \rA_l, \rD_l, \rE_6, \rE_7$. If $\Phi\neq \rE_7$ set $s:=1$ otherwise set $s:=7$. 
Let $r$ be any other natural number not equal to $s$.
Assume that there exists a subset $\widetilde{L_r} \subseteq \E(\Delta_r, R, I)$ satisfying the following two conditions:
\begin{enumerate}[label=(\alph*)] 
 \item\label{stein_cond1} One has $\U(\Phi^+, I)\cdot \U(\Phi^-, I) \cdot \widetilde{L_r} = \U(\Phi^+, I)\cdot \U(\Phi^-, I) \cdot \E(\Delta_r, R, I).$
 \item\label{stein_cond2} There is an inclusion $X_{-\alpha\ssub{r}}(I)^{\widetilde{L_r}} \subseteq \EP_s(R, I).$
\end{enumerate}
Then $A_{rs} = \E(\Phi, R, I)$. \end{lemma}
\begin{proof} Set $A:=\U(\Phi^+, I)\cdot \U(\Phi^-, I) \cdot \widetilde{L_r} \cdot \EP_s(\Phi, R, I).$
From condition~\ref{stein_cond1} of the lemma it follows that $A_{rs}=A$.

The first step of the proof is to show that $A_{rs}$ is invariant with respect to conjugation by the root subgroup $X_{-\alpha\ssub{r}}$ 
and left multiplication by elements of $X_{-\alpha\ssub{r}}(I)$. To prove this, observe first that from Lemma~\ref{lemma:dv_unipotent} and condition~\ref{stein_cond2} it follows that
\begin{multline}\nonumber 
A \subseteq \U(\Sigma_r\setminus \{\alpha_r\}, I) \cdot \U(\Phi^-, I) \cdot X_{\alpha\ssub{r}}(I) \cdot X_{-\alpha\ssub{r}}(I) \cdot \widetilde{L_r} \cdot \EP_s(R, I) \subseteq \\ 
\subseteq \U(\Sigma_r\setminus\{\alpha_r\}, I) \cdot \U(\Phi^-, I) \cdot X_{\alpha\ssub{r}}(I) \cdot X_{-\alpha\ssub{r}}(I) \cdot \widetilde{L_r} \cdot \EP_s(R, I) \subseteq \\
\subseteq \U(\Sigma_r\setminus\{\alpha_r\}, I) \cdot \U(\Phi^-, I) \cdot \widetilde{L_r} \cdot \U(\Sigma_r, I) \cdot \EP_s(R, I)  \cdot \EP_s(R, I) \subseteq \\
\subseteq \U(\Sigma_r\setminus\{\alpha_r\}, I) \cdot \U(\Phi^-, I) \cdot \widetilde{L_r} \cdot \EP_s(R, I). \end{multline}
Now it remains to apply Lemma~\ref{lemma:dv_unipotent} once again:
\begin{equation}\nonumber A^{X_{-\alpha\ssub{r}}} \subseteq \U(\Phi^+, I) \cdot \U(\Phi^-, I) \cdot \widetilde{L_r} ^{X_{-\alpha\ssub{r}}} \cdot \EP_s(R, I) \subseteq A. \end{equation}
\begin{equation}\nonumber X_{-\alpha\ssub{r}}(I) \cdot A \subseteq \U(\Phi^+, I) \cdot X_{-\alpha\ssub{r}}(I) \cdot \U(\Phi^-, I) \cdot \widetilde{L_r} \cdot \EP_s(R, I) \subseteq A. \end{equation}

Notice that the subgroup $\EP_s$ is generated by root subgroups $X_{\alpha\ssub{i}}$ for $1\leq i\leq n$ and $X_{-\alpha\ssub{i}}$ for $i\neq s$,
therefore in fact we have just proved that $A$ is invariant with respect to conjugation by all elements of $\EP_s$.

Now we are ready to prove the needed equality. By virtue of Theorem~\ref{theorem:Stepanov} it suffices to show that
$A$ is invariant with respect to left multiplication by elements of the following two families:
\begin{itemize} \item $z_{\alpha}(s, \xi)$, $s\in I$, $\xi \in R$, $\alpha\in\Sigma^-_s$;
 \item $x_{\beta}(s)$, $s \in I$, $\beta \in \Phi$. \end{itemize}
Thanks to Lemma~\ref{lemma:conj_lemma} it suffices to check the required statement only for the second family of elements.
Denote by $S$ the set of all roots $\alpha$ of $\Phi$ for which one has $X_{\alpha}(I)\cdot A_{rs}\subseteq A_{rs}$.
It remains to prove that $S$ coincides with $\Phi$. We already know that $\Phi^+ \subseteq S$.

Notice that the Weyl group $W(\Delta_s)$ acts transitively on $\Delta_s$, therefore in view of~\ref{rel:R3} the subgroup
$W_s := \langle w_\alpha(1) \mid \alpha\in\Delta_s\rangle \leq \EP_s$ acts transitively on the set of root subgroups $X_\alpha(I)$, $\alpha\in \Delta_s$.
Since $-\alpha_r\in S$ this implies that that $\Delta_s\subseteq S$.

Now denote by $\widetilde{\alpha}$ the maximal root of $\Phi$. Our assumptions on $\Phi$ guarantee that $m_s(\alpha)=1$, 
hence every two roots $\alpha, \beta \in \Sigma^-_s$ have the same $s$-shape (i.\,e. $\shape(\{s\}, \alpha = \shape(\{s\}, \beta)$).
By Lemma~\ref{lemma:abs} $W(\Delta_s)$ interchanges $\alpha$ and $\beta$. %if their length is equal.
Since $\alpha_s \in S$ the argument similar to the one above implies $\Sigma^-_s\subseteq S$. This completes the proof of the lemma. \end{proof}