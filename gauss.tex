\begin{thm}
Assume that for every terminal subsystem $\Delta$ the corresponding relative elementary subgroup admits the following decomposition:
\[ \E(\Delta,R,I)=\Hh(\Delta,R,I)\U^+(\Delta,I)\ldots\U^\pm(\Delta,I) \]
with $N$ triangular factors. Then the ambient relative elementary subgroup admits a decomposition
\[ \E(\Phi,R,I)=\Hh(\Phi,R,I)\U^+(\Phi,I)\ldots\U^\pm(\Phi,I) \]
of the same length $N$.
\end{thm}
\begin{proof}
Consider a set $Y=\Hh(\Phi,R,I)\U^+(\Phi,I)\ldots\U^\pm(\Phi,I)$. To show that $Y=\E(\Phi,R,I)$ one has to show that
\begin{enumerate}
\item $XY\subseteq Y$ for some set $X$ that generates $E(\Phi,R,I)$ as a \emph{normal} subgroup of $\E(\Phi,R)$;
\item $Y^{\E(\Phi,R)}\subseteq Y$, so that $Y$ is indeed normalized by $E(\Phi,R)$.
\end{enumerate}
We will take $X=\left\{x_\alpha(\xi)\mid\alpha\in\Pi,\ \xi\in I \right\}$. The set $X^{\E(\Phi,R)}$ contains elements $x_\alpha(\xi)$ for all $\alpha\in\Phi$ (since every root is conjugated to a fundamental one under the action of Weyl group, and the torus normalizer is contained in $\E(\Phi,R)$). Therefore $\E(\Phi,I)\subseteq\langle X\rangle^{\E(\Phi,R)}$ Now obviously $XY\subseteq Y$, since $\Hh(\Phi,R,I)$ normalizes each of $\X_\alpha(I)$, and for any $\alpha\in\Phi^+$ one has $\X_\alpha(I)\in\U(\Phi,I)$.

As for the second part, we use the fact that $\E(\Phi,R)$ is generated by $\left\{x_\alpha(\xi)\mid\alpha\in\pm\Pi,\ \xi\in R\right\}$. Fix a root $\alpha\in\pm\Pi$. It falls into one of the terminal subsystems, say $\Delta$. We expand $Y$ as follows:
\[ Y=\Hh(\Phi,R,I)\cdot\U(\Delta,I)\ldots\U^\pm(\Delta,I)\cdot\U(\Sigma,I)\ldots\U^\pm(\Sigma,I). \]
Here $\Sigma=\Phi^+\setminus\Delta$. Now
\begin{multline*}
x_\alpha(\xi)\cdot Y\cdot x_\alpha(-\xi) = x_\alpha(\xi)\cdot\E(\Delta,R,I)\cdot\U(\Sigma,I)\ldots\U^\pm(\Sigma,I)\cdot x_\alpha(-\xi) =\\
= x_\alpha(\xi)\cdot\E(\Delta,R,I)\cdot x_\alpha(-\xi)\cdot\U(\Sigma,I)\ldots\U^\pm(\Sigma,I) = Y.\qedhere
\end{multline*}
\end{proof}