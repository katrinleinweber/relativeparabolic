The proof of Gauss decomposition presented below is similar to the absolute case (cf.~\cite[Theorem~5.1]{Sm12}).
Over a local ring ring this result follows from~\cite[Proposition~2.8]{Abe69} (see also \cite[Proposition~2.3]{AbeSuzNormalSubgroups} for the case of semilocal rings).

\begin{prop}\label{thm:Gauss}
Let $\Phi$ be a reduced irreducible root system of rank $\ell$ and let $\Delta_1, \Delta_\ell$ be
two reductive subsystems of $\Phi$ corresponding to the endnodes of the Dynkin diagram of $\Phi$.
Suppose that both relative elementary subgroups $\E(\Delta_i, R, I)$, $i=1,\ell$ admit triangular decomposition with $N$ triangular factors:
\[ \E(\Delta_i, R, I) = \Hh(\Delta_i, R, I) \cdot \U(\Delta^+_i, I) \cdot \U(\Delta^-_i, I) \cdot \ldots \cdot \U(\Delta^\pm_i, I),\quad i=1,\ell. \]
Then the group $\E(\Phi, R, I)$ admits the decomposition with the same number of factors:
\[ \E(\Phi, R, I) = \Hh(\Phi,R,I) \cdot \U(\Phi^+,I) \cdot \U(\Phi^-, I) \cdot \ldots \cdot \U(\Phi^\pm,I). \]
\end{prop}
\begin{proof}
Denote by $Y$ the product of subgroups in the right-hand side of the above equality.
To show that $Y=\E(\Phi,R,I)$ it suffices to check that
\begin{enumerate}
\item $Y$ is normalized by $\E(\Phi,R)$, i.\,e. $Y^{\E(\Phi,R)}\subseteq Y$;
\item there exists a set $X$ generating $\E(\Phi,R,I)$ as a \emph{normal} subgroup of $\E(\Phi,R)$ such that $XY\subseteq Y$.
\end{enumerate}
To prove the first assertion it suffices to show that $Y^{x_\alpha(\xi)} \subseteq Y$ for any $\alpha\in \pm \Pi$, $\xi\in R$.
Fix a root $\alpha\in\pm\Pi$. Clearly, $\alpha \in \Delta_i$ for $i=1$ or $i=\ell$ and we can expand $Y$ as follows:
\[ Y=\Hh(\Phi,R,I) \cdot \U(\Delta_i^+,I) \cdot \ldots \cdot \U^\pm(\Delta_i,I) \cdot \U(\Sigma_i^+,I) \cdot \ldots \cdot \U(\Sigma_i^\pm,I). \]
For every  $h \in \Hh(\Phi, R, I)$ one has for some $s\in I$
\[ x_\alpha(\xi)\cdot h = h\cdot x_\alpha((1+s)\xi)=h\cdot x_\alpha(\xi)\,x_\alpha(s\xi). \]

Therefore by the assumption of the theorem
\begin{multline*}
Y^{x_\alpha(\xi)} \subseteq \Hh(\Phi,R,I) \cdot x_\alpha(-\xi) X_{\alpha}(I) \cdot \E(\Delta_i,R,I)\cdot x_\alpha(\xi) \cdot \U(\Sigma^+_i,I) \cdot \ldots \cdot \U(\Sigma^\pm_i,I) =\\
= \Hh(\Phi, R, I) \cdot \E(\Delta_i,R,I) \cdot \U(\Sigma^+_i,I) \cdot \ldots \cdot \U(\Sigma^\pm_i,I) = Y.
\end{multline*}

Now set $X=\left\{x_\alpha(\xi)\ \middle|\ \alpha\in\Pi,\ \xi\in I \right\}$. 
Every root is a conjugate of some fundamental root under the action of $W(\Phi)$.
Since $\E(\Phi,R)$ contains the normalizer of the torus, the set $X^{\E(\Phi,R)}$ contains all the generators of the group $\E(\Phi, I) = \langle x_\alpha(\xi),\  \alpha\in\Phi,\ \xi\in I \rangle$.
Finally, the inclusion $XY \subseteq Y$ follows from the fact that $\Hh(\Phi,R,I)$ normalizes every root subgroup $X_\alpha(I)$.
\end{proof}

\begin{thm}\label{thm:srRI1}
Let $\Phi$ be a root system, let $I$ be an ideal of arbitrary commutative ring $R$ such that $\sr(I)=1$.
Then the relative elementary Chevalley group admits Gauss decomposition
\[ \E(\Phi,R,I) = \Hh(\Phi,R,I) \cdot \U(\Phi,I) \cdot \U(\Phi^-,I) \cdot \U(\Phi,I). \]
\end{thm}
\begin{proof}
In view of the above proposition it suffices to prove that Gauss decomposition of length $3$ holds for $\Phi=\rA_1$.

Let $A=\begin{psmallmatrix}a & b \\ c & d\end{psmallmatrix}$ be an element of $\SL(2, R, I)$.
The first column of $A$ is $I$-unimodular, therefore there exists $z\in I$ such that $a+cz\in R^*$.
Multiplying $A$ on the left by $x_{12}(z)$ we get matrix $A'=x_{12}(z)\cdot A=\begin{psmallmatrix}a' & b' \\ c & d\end{psmallmatrix}$ with invertible top-left corner element $a'$.
After multiplying $A'$ on the left by $x_{21}(-c/a')$ and on the right by $x_{12}(-b'/a')$ we get a diagonal matrix. 
Thus we obtain relative Gauss decomposition of $A$
\begin{equation}\nonumber
A=x_{12}(-z)\cdot x_{21}(c/a')\cdot
\begin{pmatrix} \varepsilon & 0 \\ 0 & 1/\varepsilon \end{pmatrix}
\cdot x_{12}(b'/a')=x_{12}(-z)\cdot
\begin{pmatrix} \varepsilon & 0 \\ 0 & 1/\varepsilon \end{pmatrix}
\cdot x_{21}(y) \cdot x_{12}(b'/a'),
\end{equation}
where $\varepsilon\in 1+I$ and $y\in I$. \end{proof}