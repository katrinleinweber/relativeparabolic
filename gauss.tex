\begin{thm}\label{thm:Gauss}
Let $\Phi$ be a reduced irreducible root system of rank $\ell$ and let $\Delta_1, \Delta_\ell$ be
two reductive subsystems of $\Phi$ corresponding to the endnodes of the Dynkin diagram of $\Phi$.
Suppose that both relative elementary subgroups $\E(\Delta_i, R, I)$, $i=1,\ell$ admit triangular decomposition with $N$ triangular factors:
\[ \E(\Delta_i, R, I) = \Hh(\Delta_i, R, I) \cdot \U(\Delta^+_i, I) \cdot \U(\Delta^-_i, I) \cdot \ldots \cdot \U(\Delta^\pm_i, I),\quad i=1,\ell. \]
Then the group $\E(\Phi, R, I)$ admits the decomposition with the same number of factors:
\[ \E(\Phi, R, I) = \Hh(\Phi,R,I) \cdot \U(\Phi^+,I) \cdot \U(\Phi^-, I) \cdot \ldots \cdot \U(\Phi^\pm,I). \]
\end{thm}
\begin{proof}
Denote by $Y$ the product of subgroups in the right-hand side of the above equality.
To show that $Y=\E(\Phi,R,I)$ it suffices to check that
\begin{enumerate}
\item $Y$ is normalized by $\E(\Phi,R)$, i.\,e. $Y^{\E(\Phi,R)}\subseteq Y$;
\item there exists a set $X$ generating $\E(\Phi,R,I)$ as a \emph{normal} subgroup of $\E(\Phi,R)$ such that $XY\subseteq Y$.
\end{enumerate}
To prove the first assertion it suffices to show that $Y^{x_\alpha(\xi)} \subseteq Y$ for any $\alpha\in \pm \Pi$, $\xi\in R$.
Fix a root $\alpha\in\pm\Pi$. Clearly, $\alpha \in \Delta_i$ for $i=1$ or $i=\ell$ and we can expand $Y$ as
\[ Y=\Hh(\Phi,R,I) \cdot \U(\Delta_i^+,I) \cdot \ldots \cdot \U^\pm(\Delta_i,I) \cdot \U(\Sigma_i^+,I) \cdot \ldots \cdot \U(\Sigma_i^\pm,I). \]
For every  $h \in \Hh(\Phi, R, I)$ one has $x_\alpha(\xi)\cdot h = h\cdot x_\alpha(\zeta)$ for some $\zeta\in I$, therefore by the assumption of the theorem
\begin{multline*}
Y^{x_\alpha(\xi)} \subseteq \Hh(\Phi,R,I) \cdot \X_{\alpha}(I) \cdot \E(\Delta_i,R,I)\cdot x_\alpha(\xi) \cdot \U(\Sigma^+_i,I) \cdot \ldots \cdot \U(\Sigma^\pm_i,I) =\\
= \Hh(\Phi, R, I) \cdot \E(\Delta_i,R,I) \cdot \U(\Sigma^+_i,I) \cdot \ldots \cdot \U(\Sigma^\pm_i,I) = Y.
\end{multline*}

Now set $X=\left\{x_\alpha(\xi)\ \middle|\ \alpha\in\Pi,\ \xi\in I \right\}$. 
Every root is a conjugate of some fundamental root under the action of $W(\Phi)$.
Since $\E(\Phi,R)$ contains the normalizer of the torus, the set $X^{\E(\Phi,R)}$ contains all the generators of the group $\E(\Phi, I) = \langle x_\alpha(\xi),\  \alpha\in\Phi,\ \xi\in I \rangle$.
Finally, the inclusion $XY \subseteq Y$ follows from the fact that $\Hh(\Phi,R,I)$ normalizes every root subgroup $\X_\alpha(I)$.
\end{proof}